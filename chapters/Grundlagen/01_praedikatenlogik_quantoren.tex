\chapter{Logik}
\definition{Aussage}
Eine Aussage ist ein Satz, von dem es Sinn macht, zu fragen, ob er wahr oder falsch ist.

\section{Logische Junktoren}
Wir verknüpfen mehrere Aussagen zu größeren aussagelogischen Formeln mithilfe von logischen Junktoren:
  \paragraph{Negation:}
  $\neg A$

  \paragraph{Konjuktion:}
  $A \wedge B$

  \paragraph{Disjunktion:}
  $A \vee B$

	\par \medskip

Mit diesen grundlegenden Junkoren kann man alle Verknüpfungen darstellen. Um Schreibarbeit zu sparen gibt es verkürzende Schreibweisen:

\paragraph{Implikation:}
$A\Rightarrow B \equiv \neg(A\wedge \neg B)$
\paragraph{Äquivalenz:}
$A\Leftrightarrow B \equiv (A\wedge B)\vee (\neg A\wedge \neg B)$

\vspace{1em}
\begin{center}
	\renewcommand{\arraystretch}{1.2}
  \begin{tabular}{c c|c c c c c}
    $A$ & $B$ & $\neg A$ & $A \wedge B$ & $A \vee B$ & $A \Rightarrow B$ & $A \Leftrightarrow B$\\
    \hline  f & f & w & f & f & w & w \\
            f & w & w & f & w & w & f \\
            w & f & f & f & w & f & f \\
            w & w & f & w & w & w & w
  \end{tabular}
\end{center}



\section{Prädikatenlogik und Quantoren}
Ein Prädikat ist ein Ausdruck, der die Form einer Aussage hat, aber Variablen enthält.
Eine Aussage wird daraus erst, wenn wir angeben, für welche $m$ das Prädikat gelten soll.

Sei $M$ eine Menge und $P(m)$ für jedes $m\in M$ eine Aussage. Wir beschreiben die Aussage mit dem \emph{Allquantor}:
\begin{equation*}
  \forall m\in M: P(m)
\end{equation*}
d.h. $P(m)$ soll für \emph{jedes} Element $m$ aus $M$ gelten.
\par\medskip
Mit dem \emph{Existenzquantor} bekommt das Prädikat eine andere Bedeutung:
\begin{equation*}
  \exists m\in M: P(m)
\end{equation*}
d.h. es soll mindestens ein $m\in M$ existieren, für das $P(m)$ gilt.

\paragraph{Beispiel}
$M=\N, P(m)$: \glqq $m$ ist eine gerade Zahl.\grqq

$(\forall m\in M: P(m))$ ist falsch. \\
$(\exists m\in M: P(m))$ ist jedoch wahr.

\subsection{Verneinung von Aussagen}
Verneinung von quantifizieren Prädikat-Aussagen:
\glqq Prädikat verneinen und Quantoren tauschen.\grqq
\begin{equation*}
  \neg(\forall m\in M: P(m)) \equiv  \exists m\in M: \neg P(m)
\end{equation*}
\subsection{Reihenfolge der Quantoren}
Bei Quantoren kommt es auf die Reihenfolge an:
\begin{align*}
  \forall n\in\N \quad\exists m\in \N &: m\geq n \quad\text{ist wahr}\\
  \exists n\in\N \quad\forall m\in \N &: m\geq n \quad\text{ist falsch}\\
\end{align*}
