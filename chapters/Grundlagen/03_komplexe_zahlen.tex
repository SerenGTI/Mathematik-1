\chapter{Komplexe Zahlen}
Wir definieren $\C$ als Menge $\C\coloneqq\R\times\R$, d.h. wir definieren die komplexen Zahlen als zusammengesetzte Zahlen, also als die Menge der geordneten Paare von reellen Zahlen.
Wobei wir folgende Abbildungen mit $\C\times\C\rightarrow\C$ auf $\C$ festlegen:
\begin{description}
	\item[Addition] $(a,b) + (c,d) \coloneqq (a+b,c+d)$
	\item[Multiplikation] $(a,b) * (c,d) \coloneqq (ac-bd,ad+bc)$
\end{description}

\bemerkung
Die Menge der reellen Zahlen kann als Teilmenge von $\C$ aufgefasst werden. $\R\subset\C$ indem man die injektive Abbildung $\R\rightarrow\C, a\mapsto (a,0)$ benutzt. Die oben definierten Verknüpfungen schränken sich dann auf die Verknüpfungen in $\R$ ein:
\begin{itemize}
  \item $(a,0)+(b,0)=(a+b,0)$
  \item $(a,0)*(b,0)=(a* b - 0, a* 0+ b* 0) = (a* b,0)$
\end{itemize}
In diesem Sinne ist $\C$ eine \emph{Erweiterung} des Körpers $\R$.

\definition{Imaginäre Einheit}
Wir führen die imaginäre Einheit ein.
$\i\coloneqq (0,1)$ damit gilt:
\begin{equation*}
  (0,1)*(0,1) = (0*0 -1*1,0*1+0*1)=(-1,0)=\i^2=-1
\end{equation*}
Es gilt also $\i^2=-1$, daher schreibt man auch $\i=\sqrt{-1}$. Die Zahlen $(0,y)=y*\i, y\in\R$ heißten imaginäre Zahlen.
Wir können uns wegen $\C=\R\times\R$ komplexe Zahlen als Punkte bzw. Vektoren in der \emph{Gauß'schen Zahlenebene} vorstellen.

\begin{satz}{}
  Für jede komplexe Zahl $(a,b)\in\C$ gilt:
  \begin{equation*}
    (a,b)=a+b*\i
  \end{equation*}
\end{satz}

\beweis Durch Ausrechnen der rechten Seite:
\begin{align*}
  a+b\i &= (a,0)+(b,0)*(0,1)\\
  &=(a,0)+(b*0-0*1,b*1+0*0)\\
  &=(a,0)+(0,b)=(a,b)
\end{align*}

\bemerkung
Wie man leicht nachrechnet, gelten wie in $\R$ die Kommutativ-, Assoziativ- und Distributivgesetze.

\definition{Konjugiert komplexe Zahl}
Sei $z=a+b\i\in\C$. Dann heißt $\overline z$ die konjugiert komplexe Zahl $\overline z=a-b\i$ von $z$.

\begin{satz}{Eigenschaften der konjugiert komplexen Zahl}
  Seien $z,w\in\C$ dann gilt:
  \begin{enumerate}
    \item $\overline{z+w}=\overline z+\overline w$
    \item $\overline{z* w}=\overline z * \overline w$
    \item $\frac 1 2 (z+\overline z)=\Re(z)$
    \item $\frac 1 2 (z-\overline z)=\Im(z)$
    \item $z* \overline z > 0 \in\R$ falls $z\neq0$
  \end{enumerate}
\end{satz}

\definition{Betrag einer komplexen Zahl}
Mit der komplexen Zahl $z=a+b\i$ und $a,b\in\R$ gilt für den Betrag von $z$:
\begin{align*}
  |z|&=\sqrt{z*\overline z}=\sqrt{a^2+b^2}\\
  |z|&=|\overline z|
\end{align*}
Insbesondere lässt sich das multiplikative Inverse wie folgt ausdrücken:
\begin{equation*}
  z^{-1}=\frac 1 z=\frac{\overline z}{z*\overline z}=\frac{a-b*\i}{a^2+b^2}
\end{equation*}

\section{Polarkoordinaten-Darstellung}
