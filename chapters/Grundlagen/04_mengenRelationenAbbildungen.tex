\chapter{Mengen, Relationen und Abbildungen}

\section{Mengen}
Eine Menge ist eine wohldefinierte Gesamtheit von Objekten, den Elementen der Menge.
\begin{equation*}
  \text{z.B. }\Q=\set{\frac{p}{q}}{p\in\Z, q\in\N}
\end{equation*}

\begin{definition}{Teilmenge}
	Eine Menge $M_1$ ist \emph{Teilmenge} von $M$, wenn
	\begin{align*}
	  &\forall x\in M_1 : x\in M\\
	  &\Rightarrow M_1 \subseteq M
	\end{align*}
	Für jede Menge M gilt $\emptyset \subseteq M$ und $M\subseteq M$.\\
	Gilt $M_1\subseteq M$ und $M_1\neq M$ ist $M_1$ eine \emph{echte Teilmenge} von $M$, d.h. $M_1 \subset M$ oder $M_1\subsetneq M$
\end{definition}

\renewcommand{\arraystretch}{1.4}
\begin{tabular}{>{\bfseries\scshape}r l}
	Potenzmenge & $\P(M)=\Pot(M) \text{ ist die Menge aller Teilmengen von } M.$\\
	Schnittmenge & $M_s = M_1 \cap M_2; \quad M_s \coloneqq \set{m\in M_1}{m\in M_2}$\\
	Vereinigung & $M_v = M_1 \cup M_2; \quad M_v \coloneqq \set{m}{m\in M_2 \vee m\in M_2}$\\
	Differenz & $M_1\setminus M_2 \coloneqq \set{m\in M_1}{m\not\in M_2}$\\
	Kartesisches Produkt & $M_1 \times M_2 \coloneqq \set{(m_1, m_2)}{m_1\in M_1 \wedge m_2\in M_2}$
\end{tabular}
\medskip

Zwei Mengen $M_1$ und $M_2$ heißen \emph{disjunkt}, falls $M_1\cap M_2 = \emptyset$

\section{Relationen}
\begin{definition}{Relation}
	Eine Relation zwischen zwei Mengen $M$ und $N$ ist eine Teilmenge von $M\times N$.
	\begin{equation*}
	  R\subseteq M_1\times M_2
	\end{equation*}
	ist $(x, y) \in R$, steht $x$ mit $y$ in Relation $\rightarrow x\sim y$.
\end{definition}

$R \subseteq M\times M$ heißt:

\medskip
\renewcommand{\arraystretch}{1.4}
\begin{tabular}{>{\bfseries\scshape}r l}
	reflexiv & falls $\forall x\in M : (x,x)\in R$\\
	symmetrisch & falls $\forall x,y \in M : (x,y)\in M \Rightarrow (y,x) \in R$\\
	antisymmetrisch & falls $\forall x,y \in R : (x,y)\in M \wedge (y,x)\in R \Rightarrow x=y$\\
	transitiv & falls $\forall x,y,z \in M : (x,y)\in R \wedge (y,z)\in R \Rightarrow (x,z)\in R$
\end{tabular}
\medskip

\begin{definition}{Äquivalenzrelation}
	Eine Relation heißt Äquivalenzrelation, wenn sie reflexiv, symmetrisch und transitiv ist.
\end{definition}


\begin{definition}{Ordnungsrelation}
	Eine Relation heißt Ordnungsrelation, wenn sie reflexiv, antisymmetrisch und transitiv ist.
\end{definition}


\section{Abbildungen}
\begin{definition}{Abbildung}
	Seien $M$ und $N$ zwei Mengen. Eine Zuordnungsvorschift, die jedem Element $x\in M$ ein Element $f(x)\in N$ zuweist, heißt Abbildung oder Funktion von $M$ nach $N$.
	\begin{equation*}
	  f:M\rightarrow N, x\mapsto f(x)
	\end{equation*}
	$M$: Definitionsbereich, $N$: Wertebereich
\end{definition}


\begin{definition}{Bild und Urbild}
	Sei $f:M\mapsto N$ eine Abbildung. Wir definieren
	\begin{itemize}
	  \item für $x\in M$ heißt $f(x)\in N$ das \emph{Bild} von $x$
	  \item für eine Teilmenge $A\subseteq M$ heißt $f(A)=\set{f(x)}{x\in A}$ das \emph{Bild der Teilmenge  $A$}
	  \item für eine Teilmenge $B\subseteq N$ heißt $f^{-1}(B) = \set{x\in M}{f(x)\in B}$ das \emph{Urbild} von $B$
	\end{itemize}
\end{definition}


\begin{definition}{Graph einer Abbildung}
	Sei $f:M\rightarrow N$ eine Abbildung. Der Graph von $f$ ist eine Teilmenge des Werte- und Definitionsbereichs
	\begin{equation*}
		\mathrm{Graph}(f)=\set{(x,f(x))}{x\in M}\subseteq M\times N
	\end{equation*}
\end{definition}

Fasst man eine Funktion als eine Relation auf, so ist der Graph das selbe wie $R$.
\begin{equation*}
	\mathrm{Graph}(f)=R\subseteq M\times N
\end{equation*}

\bemerkung Für Funktionen $f:\R\rightarrow\R$ ist der Graph eine Teilmenge der Ebene $\R^2$.

\begin{definition}{Verkettung}
	Seien $f:M\rightarrow N$ und $g:N\rightarrow P$ Abbildungen. Dann ist die Verkettung:
	\begin{align*}
	  &g\circ f:M\rightarrow P\\
	  &g\circ f(x)\coloneqq g(f(x))
	\end{align*}
\end{definition}


\begin{definition}{Identität}
	Für jede Menge $M$ ist
	\begin{equation*}
	  \mathrm{id}_M:M\rightarrow M, x\mapsto x
	\end{equation*}
	die identische Abbildung auf $M$.
\end{definition}


\subsection{Abbildungseigenschaften}
Sei $f:M\rightarrow N$ eine Abbildung. Dann heißt $f$:

\medskip
\renewcommand{\arraystretch}{1.4}
\begin{tabular}{>{\bfseries\scshape}r l}
	injektiv & wenn jedes Element $y\in N$ \emph{höchstens ein Urbild} hat.\\
	surjektiv & wenn jedes Element $y\in N$ \emph{mindestens ein Urbild} hat. $\forall y\in N\; \exists x\in M : f(x) = y$\\
	bijektiv & wenn jedes Element $y\in N$ \emph{genau ein Urbild} hat. $\forall y\in N\; \exists! x\in M : f(x) = y$
\end{tabular}
\medskip

\bemerkung
\begin{enumerate}
  \item Bijektivität gilt genau dann, wenn es eine Umkehrabbildung $f^{-1}$ gibt:
  \begin{align*}
    f:M\rightarrow N && f^{-1}:N\rightarrow M\\
    f\left(f^{-1}(x)\right) \quad\text{mit }x\in N && f^{-1}\left(f(x)\right)=x \quad\text{mit }x\in M
  \end{align*}
  \item Man kann jede Abbildung surjektiv machen, indem man den Wertebereich durch das Bild von $f$ ersetzt: $N\coloneqq f(M)$
\end{enumerate}

\section{Mächtigkeit von Mengen}
Die Mächtigkeit einer Menge ist die Anzahl ihrer Elemente. Man schreibt $|M|$ für die Mächtigkeit von $M$.

Zwei Mengen $A$ und $B$ sind gleich mächtig, wenn es eine bijektive Abbildung $f:A\rightarrow B$ gibt.

Eine Menge heißt \emph{abzählbar unendlich}, falls $|A|=|\N|$ d.h. falls es eine bijektive Abbildung $f:A\rightarrow \N$ gibt.

Sie heißt \emph{überabzählbar unendlich}, falls $|A|>|\N|$.

Es gilt immer auch für unendliche Mengen, dass $|M| < |\Pot(M)|$.

Für endliche Mengen gilt $|\Pot(M)| = 2^{|M|}$



\section{Zahlenmengen}
\begin{definition}{Natürliche Zahlen}
	Die natürlichen Zahlen sind eine Menge $\N$, auf der eine Abbildung $f:\N\rightarrow\N$ erklärt ist, die folgende Eigenschaften hat, wobei $f(n)$ der \emph{Nachfolger} von $n$ heißt.
	\begin{description}
	  \item[$\N 1$] Es gibt genau ein Element in $\N$, das nicht Nachfolger eines anderen Elements ist.
	  \item[$\N 2$] $f$ ist injektiv
	  \item[$\N 3$] Ist $M\subseteq \N$ eine Teilmenge, die folgende Eigenschaften hat:
	  \begin{enumerate}
	    \item $1\in M$
	    \item Falls $m\in M$ und $f(m)\in M$
	  \end{enumerate}
	  Dann gilt: $M = \N$

	  D.h. $M\subseteq \N : 1\in M \wedge (m\in M \Rightarrow f(m)\in M) \Rightarrow M=\N$
	\end{description}
\end{definition}


Man kann zeigen, dass die natürlichen Zahlen durch diese Eigenschaften (die \textsc{Peano}-Axiome) gekennzeichnet sind. Das heißt, dass es im wesentlichen nur eine solche Menge mit einer solchen Abbildung $f$ gibt, nämlich $\N$.

Das Axiom $\N 3$ heißt auch Induktionsaxiom. Aus ihm folgt:

\begin{satz}{Vollständige Induktion}
  Sei $A(n)$ für jede natürliche Zahl $n \in\N$ eine Aussage, für die gilt:
  \begin{itemize}
    \item $A(1)$ ist wahr
    \item $\forall n\in\N : A(n) \Rightarrow A(n+1)$
  \end{itemize}
  dann ist $A(n)$ für alle $n\in\N$ wahr.
\end{satz}
