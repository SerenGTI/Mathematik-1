\chapter{Grundlegende Rechenmethoden}
\section{Summen- und Produktzeichen}
\begin{equation*}
  \sum\limits_{k=m}^n a_k\coloneqq a_m + a_{m+1} + \ldots + a_n
\end{equation*}
Bei der Summe ist $k$ der Summationsindex, $m$ die untere und $n$ die obere Summationsgrenze
\begin{equation*}
  \prod\limits_{k=m}^n a_k\coloneqq a_m * a_{m+1} * \ldots * a_n
\end{equation*}

\bemerkung
\begin{itemize}
  \item Ist die obere Summationsgrenze kleiner als die untere, so handelt es sich um eine \emph{leere Summe}, ihr Wert ist 0.
  \item Entsprechend ist der Wert des \emph{leeren Produkts} 1.
\end{itemize}
