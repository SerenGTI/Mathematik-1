\chapter{Lineare Algebra}
\section{Vektorräume}
\paragraph{Beispiel}
\begin{align*}
  \R^2&=\R\times\R=\set{(x,y)}{x,y\in\R}\\
  \R^3&=\set{(x,y,z)}{x,y,z\in\R}\\
  &\vdots\\
  \R^n&=\set{(x_1,x_2,\ldots,x_n)}{x_1,\ldots,x_n\in\R}\\
\end{align*}
Wir schreiben die Elemente von $\R^n$ auch als sogenannte Spaltenvektoren:
\begin{equation*}
  \begin{pmatrix}
    x_1\\
    x_2\\
    \vdots\\
    x_n
  \end{pmatrix} \text{ anstatt von } (x_1,x_2,\ldots,x_n)
\end{equation*}

Mit der komponentenweisen Addition, der Vektoraddition:
\begin{equation*}
  \begin{pmatrix}
    x_1\\
    \vdots\\
    x_n
  \end{pmatrix}
  +
  \begin{pmatrix}
    y_1\\
    \vdots\\
    y_n
  \end{pmatrix}
  =
  \begin{pmatrix}
    x_1+y_1\\
    \vdots\\
    x_n+y_n
  \end{pmatrix}
\end{equation*}
wird $\R^n$ zu einer abel'schen Gruppe mit dem Nullvektor als neutrales Element und dem negierten Vektor als inverses Element bezüglich der Addition.

In der Vektorrechnung nennt man Zahlen (z.B. Elemente aus $\R,\C,\Q$) \emph{Skalare}, um Zahlen und Vektoren deutlich zu unterscheiden.

Sei $x\coloneq \begin{pmatrix}
  x_1\\
  \vdots\\
  x_n
\end{pmatrix} \in \R$ und $\lambda\in\R$. Dann ist die \emph{skalare Multiplikation} $x\cdot\lambda$ definiert durch $x\cdot\lambda \coloneqq \begin{pmatrix}
  \lambda\cdot x_1\\
  \vdots\\
  \lambda\cdot x_n
\end{pmatrix}$

Die beiden Operationen Vektoraddition und skalare Multiplikation sind kennzeichnend für einen Vektorraum.
