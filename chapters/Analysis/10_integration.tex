\chapter{Integration}
\section{Integration von Funktionen einer Variable}
\glqq Integration wird überall dort benötigt, wo ändernde Ursachen sich zu einer Gesamtwirkung summieren.\grqq Geometrisch kann man das (noch zu definierende) Integral z.B. über eine stetige Funktion $f:[a,b]\rightarrow \R$ interpretieren als Fläche unter dem Graphen, wo Anteile unter der $x$-Achse als negativ gerechnet werden.

\begin{center}
	\begin{easyfunction}{0}{10}{-3}{3}{0.7}

		\fill [green, opacity=0.3, domain=0.52:5, variable=\x]
      (0.52, 0)
      -- plot ({\x}, {0.05*(\x-5)*(\x-5)*(\x-5)-\x+5})
      -- (5, 0);

		\fill [red, opacity=0.3, domain=5:9.47, variable=\x]
      (0.52, 0)
      -- plot ({\x}, {0.05*(\x-5)*(\x-5)*(\x-5)-\x+5})
      -- (9.47, 0)
      -- cycle;

		\draw[->] (0,0) -- (10.2,0) node[right] {$x$};
		\draw[->] (0,-2.7) -- (0,2.7) node[above] {$f(x)$};

		\draw (0.52,0) node (a) [fill = white,rectangle,inner sep = 0pt,minimum size = 0pt,minimum height=4pt,draw, label={below:$a$}] {};
		\draw (9.47,0) node (b) [fill = white,rectangle,inner sep = 0pt,minimum size = 0pt,minimum height=4pt,draw, label={below:$b$}] {};

		\begin{scope}
			\clip(0,-3) rectangle (10,3);

			\draw[line width=0.5mm, scale=1, domain=0:10, smooth, variable=\x, blue] plot ({\x},{0.05*(\x-5)*(\x-5)*(\x-5)-\x+5}) node[below right] {};
		\end{scope}
	\end{easyfunction}
\end{center}
