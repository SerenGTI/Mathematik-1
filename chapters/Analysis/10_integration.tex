\chapter{Integration}
\section{Integration von Funktionen einer Variable}
\glqq Integration wird überall dort benötigt, wo ändernde Ursachen sich zu einer Gesamtwirkung summieren.\grqq Geometrisch kann man das (noch zu definierende) Integral z.B. über eine stetige Funktion $f:[a,b]\rightarrow \R$ interpretieren als Fläche unter dem Graphen, wo Anteile unter der $x$-Achse als negativ gerechnet werden.

\begin{center}
	\begin{easyfunction}{0}{10}{-3}{3}{0.7}

		\fill [green, opacity=0.3, domain=0.52:5, variable=\x]
      (0.52, 0)
      -- plot ({\x}, {0.05*(\x-5)*(\x-5)*(\x-5)-\x+5})
      -- (5, 0);

		\fill [red, opacity=0.3, domain=5:9.47, variable=\x]
      (0.52, 0)
      -- plot ({\x}, {0.05*(\x-5)*(\x-5)*(\x-5)-\x+5})
      -- (9.47, 0)
      -- cycle;

		\draw[->] (0,0) -- (10.2,0) node[right] {$x$};
		\draw[->] (0,-2.7) -- (0,2.7) node[above] {$f(x)$};

		\draw (0.52,0) node (a) [fill = white,rectangle,inner sep = 0pt,minimum size = 0pt,minimum height=4pt,draw, label={below:$a$}] {};
		\draw (9.47,0) node (b) [fill = white,rectangle,inner sep = 0pt,minimum size = 0pt,minimum height=4pt,draw, label={below:$b$}] {};

		\begin{scope}
			\clip(0,-3) rectangle (10,3);

			\draw[line width=0.5mm, scale=1, domain=0:10, smooth, variable=\x, blue] plot ({\x},{0.05*(\x-5)*(\x-5)*(\x-5)-\x+5}) node[below right] {};
		\end{scope}
	\end{easyfunction}
\end{center}

Der Grundgedanke ber der Integration ist es, den Graph von $f$ durch Rechtecke anzunähern. Es gibt verschiedene Begriffe von Integration, z.B.
\begin{itemize}
	\item Das Cauchy-Integral für Regelfunktionen
	\item Das Riemann-Integral
	\item Das Lebesque-Integral
\end{itemize}
Diese verschiedenen Integralbegriffe unterscheiden sich in ihrer Leistungsfähigkeit, d.h. in den Klassen von Funktionen für die die entsprechenden Integrale definiert sind:
$$
	\begin{Bmatrix}
		\text{Cauchy-}\\
		\text{integrierbare}\\
		\text{Funktionen}
\end{Bmatrix}\subsetneq
\begin{Bmatrix}
	\text{Riemann-}\\
	\text{integrierbare}\\
	\text{Funktionen}
\end{Bmatrix}\subsetneq
\underset{\text{(nicht alle Funktionen)}}{\begin{Bmatrix}
	\text{Lebesque-}\\
	\text{integrierbare}\\
	\text{Funktionen}
\end{Bmatrix}}
$$

\section{Das Cauchy-Integral}
\begin{definition}{Intervall-Zerlegung}
	Die endliche Punktmenge aus $n+1$ reellen Zahlen $Z=\simpleset{x_0,x_1,\ldots,x_n}$ heißt Zerlegung von $[a,b]$, wenn gilt
		$a=x_0<x_1<x_2<\ldots<x_n=b$.
\end{definition}


\begin{definition}{Treppenfunktionen}
	Eine (beschränkte) Funktion $f:[a,b]\rightarrow \R$ heißt Treppenfunktion, falls eine Zerlegung $Z$ existiert, so dass gilt:
	\begin{equation*}
		f(x)=f_j\enspace\text{ für }x\in(x_{j-1},x_j)
	\end{equation*}
	Für reelle Zahlen $f_1,\ldots,f_n\in\R$.
\end{definition}

Diese Funktionen sind also auf dem offenen Teilintervall $(x_{j-1},x_j)$ konstant. An den Sprungstellen $\simpleset{x_0,\ldots,x_n}$ dürfen sie beliebige Werte annehmen. Beispiel:

\begin{center}
	\begin{easyfunction}{0}{3}{0}{10}{1}
		\draw[->] (0,0) -- (9.2,0) node[right] {$x$};
		\draw[->] (0,0) -- (0,2.7) node[above] {$f(x)$};
		%\makegrid

		\draw (1,0) node (a) [fill = white,rectangle,inner sep = 0pt,minimum size = 0pt,minimum height=4pt,draw, label={below:$a$}] {};
		\draw (3,0) node (x1) [fill = white,rectangle,inner sep = 0pt,minimum size = 0pt,minimum height=4pt,draw, label={below:$x_1$}] {};
		\draw (5,0) node (x2) [fill = white,rectangle,inner sep = 0pt,minimum size = 0pt,minimum height=4pt,draw, label={below:$x_2$}] {};
		\draw (7.5,0) node (x2) [fill = white,rectangle,inner sep = 0pt,minimum size = 0pt,minimum height=4pt,draw, label={below:$x_3$}] {};
		\draw (8.5,0) node (b) [fill = white,rectangle,inner sep = 0pt,minimum size = 0pt,minimum height=4pt,draw, label={below:$b$}] {};

		\begin{scope}
			\clip(0,0) rectangle (10,3);

			\draw[line width=0.5mm,scale=1,domain=1:3,smooth,variable=\x,red] plot ({\x},{0.2})
				node[below right] {};
			\draw[line width=0.5mm,scale=1,domain=3:5,smooth,variable=\x,red] plot ({\x},{1.5})
				node[below right] {};
			\draw[line width=0.5mm,scale=1,domain=5:7.5,smooth,variable=\x,red] plot ({\x},{0.7})
				node[below right] {};
			\draw[line width=0.5mm,scale=1,domain=7.5:8.5,smooth,variable=\x,red] plot ({\x},{2.3})
				node[below right] {};
		\end{scope}


		\draw (1,0.2) node (fa) [fill = red,circle,inner sep = 0pt,minimum size = 4pt] {};
		\draw (3,0.2) node (fx11) [draw,red,fill = white,circle,inner sep = 0pt,minimum size = 4pt] {};
		\draw (3,2) node (fx12) [fill = red,circle,inner sep = 0pt,minimum size = 4pt] {};
		\draw (3,1.5) node (fx13) [draw,red,fill = white,circle,inner sep = 0pt,minimum size = 4pt] {};

		\draw (5,0.7) node (fx21) [draw,red,fill = white,circle,inner sep = 0pt,minimum size = 4pt] {};
		\draw (5,1.5) node (fx22) [fill = red,circle,inner sep = 0pt,minimum size = 4pt] {};
		\draw (7.5,0.7) node (fx31) [fill = red,circle,inner sep = 0pt,minimum size = 4pt] {};
		\draw (7.5,2.3) node (fx32) [draw,red,fill = white,circle,inner sep = 0pt,minimum size = 4pt] {};
		\draw (8.5,2.3) node (fb) [fill = red,circle,inner sep = 0pt,minimum size = 4pt] {};
	\end{easyfunction}
\end{center}

\paragraph{Bemerkung:}
\begin{itemize}
	\item Die Summe von zwei Treppenfunktionen ist wiederum eine Treppenfunktion. Verwende dafür die Vereinigung der beiden Zerlegungen.
	\item Für $\lambda\in\R$ und $f$ eine Treppenfunktion, ist $\lambda f$ eine Treppenfunktion.
\end{itemize}
\begin{lemma}{}
	Die Menge der Treppenfunktionen über einem Intervall $[a,b]$ bildet einen reellen Vektorraum, nämlich einen Untervektorraum von der Menge $\simpleset{f:[a,b]\rightarrow \R}$ aller reellen Funktionen auf $[a,b]$.
\end{lemma}

Völlig natürlich ist jetzt die Definition des Integrals von Treppenfunktionen, wir summieren die Flächeninhalte der Rechtecke unter dem Graphen auf.
\begin{definition}{}
	Das Integral über eine Treppenfunktion $f:[a,b]\rightarrow \R$ ist die reelle Zahl
	\begin{equation*}
		\int\limits_a^bf(x)\intd x\coloneqq \sum\limits_{j=1}^n f_j*(x_j-x_{j-1})
	\end{equation*}
\end{definition}
Wir bemerken, dass das Integral nicht von der Wahl der Zerlegung abhängt.

\begin{satz}{Eigenschaften des Integrals}
	Seien $f,g$ Treppenfunktionen über dem Intervall $[a,b]$ und $c\in\R$. Es gilt dann:
	\begin{enumerate}
		\item Integral einer Konstanten:
		\begin{equation*}
			\int\limits_a^b c\intd x = c*(b-a)
		\end{equation*}
		\item Linearität des Integrals:
		\begin{itemize}
			\item $\displaystyle\int\limits_a^b c*f(x)\intd x=c*\int\limits_a^b f(x)\intd x$
			\item $\displaystyle\int\limits_a^b f(x)+g(x) \intd x = \int\limits_a^b f(x) \intd x+\int\limits_a^b g(x) \intd x$
		\end{itemize}
		\item Grenzaufteilung:
		\begin{equation*}
			\int\limits_a^b f(x)\intd x=\int\limits_a^c f(x)\intd x+\int\limits_c^b f(x)\intd x \quad (a\leq c\leq b)
		\end{equation*}
		\item \textsc{Schwarz}'sche Ungleichung:
		\begin{equation*}
			\int\limits_a^b f(x)g(x)\intd x
				\leq \left(\int\limits_a^b f^2(x)\intd x\right)\left(\int\limits_a^b g^2(x)\intd x\right)
		\end{equation*}
		\item Monotonie des Integrals:
		\begin{itemize}
			\item $\displaystyle\int\limits_a^b f(x)\intd x\leq \int\limits_a^b g(x)\intd x \quad (f(x)\leq g(x) \enspace\forall x\in[a,b])$
			\item $\displaystyle\left|\int\limits_a^b f(x)\intd x\right|\leq \int\limits_a^b |f(x)|\intd x\leq \sup_{x\in[a,b]}|f(x)|*(b-a)$
		\end{itemize}
	\end{enumerate}
\end{satz}

Wir werden nun die Klasse der Funktionen einführen, für die wir das Integral erklären. Der Grundgedanke ist, Funktionen durch Treppenfunktionen anzunähern.

\begin{center}
	\begin{easyfunction}{0}{3}{0}{3}{0.9}

		\draw [green, line width=0.2mm] (0.2,0) rectangle (3.6,1.7);

		\draw[->] (0,0) -- (4,0) node[right] {$x$};
		\draw[->] (0,0) -- (0,3) node[above] {$f(x)$};

		\begin{scope}
			\clip(0,0) rectangle (4,3);

			\draw[line width=0.5mm,scale=1,domain=0.2:3.6,smooth,variable=\x,red] plot ({\x},{sin(deg(2*(\x-1.1)))+1.5})
				node[below right] {};
		\end{scope}
	\end{easyfunction}
	\begin{easyfunction}{0}{3}{0}{3}{0.9}

		\draw [green, line width=0.2mm] (0.2,0) rectangle (1.33,1.5);
		\draw [green, line width=0.2mm] (1.33,0) rectangle (2.46,2.5);
		\draw [green, line width=0.2mm] (2.46,0) rectangle (3.6,1.5);

		\draw[->] (0,0) -- (4,0) node[right] {$x$};
		\draw[->] (0,0) -- (0,3) node[above] {$f(x)$};
		%\makegrid



		\begin{scope}
			\clip(0,0) rectangle (4,3);

			\draw[line width=0.5mm,scale=1,domain=0.2:3.6,smooth,variable=\x,red] plot ({\x},{sin(deg(2*(\x-1.1)))+1.5})
				node[below right] {};
		\end{scope}
	\end{easyfunction}
	\begin{easyfunction}{0}{3}{0}{3}{0.9}
		\draw [green, line width=0.2mm] (0.2,0) rectangle (0.76,0.7);
		\draw [green, line width=0.2mm] (0.76,0) rectangle (1.32,1.8);
		\draw [green, line width=0.2mm] (1.32,0) rectangle (1.88,2.5);
		\draw [green, line width=0.2mm] (1.88,0) rectangle (2.44,2.5);
		\draw [green, line width=0.2mm] (2.44,0) rectangle (3,1.8);
		\draw [green, line width=0.2mm] (3,0) rectangle (3.6,0.7);

		\draw[->] (0,0) -- (4,0) node[right] {$x$};
		\draw[->] (0,0) -- (0,3) node[above] {$f(x)$};

		\begin{scope}
			\clip(0,0) rectangle (4,3);

			\draw[line width=0.5mm,scale=1,domain=0.2:3.6,smooth,variable=\x,red] plot ({\x},{sin(deg(2*(\x-1.1)))+1.5})
				node[below right] {};
		\end{scope}
	\end{easyfunction}
\end{center}

Die entscheidende Idee dabei ist, gleichmäßige Konvergenz von Treppenfunktionen gegen eine Funktion $f$ zu fordern.
\paragraph{Erinnerung:}
Eine Folge von Treppenfunktionen $(t_n):[a,b]\rightarrow \R$ konvergiert gleichmäßig gegen $f:[a,b]\rightarrow \R$, wenn gilt:
\begin{equation*}
	\forall\epsilon>0\enspace\exists N\in\N \enspace\underbrace{\forall x\in[a,b]:|t_n(x)-f(x)|}_{\sup_{x\in[a,b]}|t_n(x)-f(x)|}<\epsilon
\end{equation*}
Der Fehler kann so durch $\epsilon*(b-a)$ beschränkt werden.

\begin{center}
	\begin{easyfunction}{0}{3}{0}{10}{1}
		\draw[->] (0,0) -- (8,0) node[right] {$x$};
		\draw[->] (0,0) -- (0,5) node[above] {$f(x)$};
		%\makegrid



		\fill [green, opacity=0.3, domain=1:7, variable=\x]
      (1, 0)
      -- plot ({\x},{1.5*sin(deg(\x-1))+3})
      -- (7, 0);
		\fill [white, domain=1:7, variable=\x]
      (1, 0)
      -- plot ({\x},{1.5*sin(deg(\x-1))+2})
      -- (7, 0);

		\draw (1,0) node (a) [fill = white,rectangle,inner sep = 0pt,minimum size = 0pt,minimum height=4pt,draw, label={below:$a$}] {};
		\draw (7,0) node (b) [fill = white,rectangle,inner sep = 0pt,minimum size = 0pt,minimum height=4pt,draw, label={below:$b$}] {};


		\begin{scope}
			\clip(0,0) rectangle (10,6);

			\draw[line width=0.5mm,scale=1,domain=1:7,smooth,variable=\x,red] plot ({\x},{1.5*sin(deg(\x-1))+2.5})
				node[below right] {};

			\draw[line width=0.2mm,scale=1,domain=1:7,smooth,variable=\x,green] plot ({\x},{1.5*sin(deg(\x-1))+3})
				node[below right] {};
			\draw[line width=0.2mm,scale=1,domain=1:7,smooth,variable=\x,green] plot ({\x},{1.5*sin(deg(\x-1))+2})
				node[below right] {};

		\end{scope}


		\draw (0.8,3) node(ep1) [draw, rectangle, inner sep=0pt, minimum width=8pt, minimum height=0pt] {};
		\draw (0.8,2) node(ep2) [draw, rectangle, inner sep=0pt, minimum width=8pt, minimum height=0pt] {};
		\draw (0.8,2.5) node(ep) [draw=none, green, rectangle, inner sep=0pt, minimum size=0pt, label={left:$2\epsilon$}] {};


		\draw[->]
		(ep1) -- (ep2);
		\draw[->]
		(ep2) -- (ep1);

		\tikzset{
	    every node/.style={
					minimum size=0.5mm,
					rectangle,
					fill=blue,
					inner sep=0pt,
					outer sep=0pt,
					blue
	    }
		}

		\draw (1,3) node (1) [] {};

		\draw (1.5,3) node (2) [] {};
		\draw (1.5,3.5) node (3) [] {};

		\draw (2,3.5) node (4) [] {};
		\draw (2,4) node (5) [] {};

		\draw (3.2,4) node (6) [] {};
		\draw (3.2,3.5) node (7) [] {};

		\draw (3.7,3.5) node (8) [] {};
		\draw (3.7,2.7) node (9) [] {};

		\draw (4.2,2.7) node (10) [] {};
		\draw (4.2,2) node (11) [] {};

		\draw (4.7,2) node (12) [] {};
		\draw (4.7,1.4) node (13) [] {};

		\draw (5.2,1.4) node (14) [] {};
		\draw (5.2,1) node (15) [] {};

		\draw (6.3,1) node (16) [] {};
		\draw (6.3,1.5) node (17) [] {};

		\draw (6.8,1.5) node (18) [] {};
		\draw (6.8,2) node (19) [] {};

		\draw (7,2) node (20) [] {};

		\draw[blue, line width=0.5mm]
		(1) -- (2) (2) -- (3) (3) -- (4) (4) -- (5) (5) -- (6) (6) -- (7) (7) -- (8) (8) -- (9) (9) -- (10) (10) -- (11) (11) -- (12) (12) -- (13) (13) -- (14) (14) -- (15) (15) -- (16) (16) -- (17) (17) -- (18) (18) -- (19) (19) -- (20);
	\end{easyfunction}
\end{center}

Wir führen nun die sogenannten Regelfunktionen ein als die Klasse von Funktionen, die sich als gleichmäßige Limites von Treppenfunktionen darstellen lassen:
\begin{definition}{Regelfunktionen}
	Die Funktion $f:[a,b]\rightarrow \R$ heißt Regelfunktion, falls es eine Folge von Treppenfunktionen $(t_n):[a,b]\rightarrow \R$ gibt, die gleichmäßig gegen $f$ konvergiert.
\end{definition}

\paragraph{Bezeichnung:}
Konvergiert $t_n$ gleichmäßig gegen $f$, so schreiben wir auch
\begin{equation*}
	t_n\overset{glm.}\longrightarrow f \text{ bzw. } t_n\overset{glm.}{\underset{n\to\infty}\longrightarrow} f
\end{equation*}


Welche Funktionen sind Regelfunktionen?
\begin{satz}{Hauptsatz über Regelfunktionen}
	Die Funktion $f:[a,b]\rightarrow\R$ ist eine Regelfunktion genau dann, wenn $f$ in jedem Punkt von $(a,b)$ einen links- oder einen rechtsseitigen Grenzwert, und in $a$ einen rechtsseitigen sowie in $b$ einen linksseiten Grenzwert besitzt.
\end{satz}
\paragraph{Folgerung:}
\begin{itemize}
	\item Insbesondere sind stetige Funktionen Regelfunktionen.
	\item Stückweise stetige Funktionen (insbesondere Treppenfunktionen) und beschränkte monotone Funktionen sind Regelfunktionen.
	\item Produkt und Betrag von Regelfunktionen sind wiederum Regelfunktionen.
\end{itemize}
\paragraph{Bemerkung:}
Die Menge der Regelfunktionen bildet einen reellen Vektorraum.

\begin{definition}{Integral für Regelfunktionen}
	Sei $f:[a,b]\rightarrow\R$ eine Regelfunktion, d.h. es gibt eine Folge von Treppenfunktionen $t_n:[a,b]\rightarrow\R$ mit $t_n\overset{glm.}{\underset{n\to\infty}\longrightarrow} f$.
	Dann setzen wir das Integral der Regelfunktion als:
	\begin{equation*}
		\int\limits_a^b f(x)\intd x\coloneqq \lim\limits_{n\to\infty}\int\limits_a^b t_n(x)\intd x
	\end{equation*}
\end{definition}

Damit diese Definition sinnvoll ist, müssen wir noch die Wohldefiniertheit beweisen. Das heißt die Existenz des Grenzwerts auf der rechten Seite und die Unabhängigkeit von der Wahl der Folge $(t_n)$.

\begin{lemma}{}
	Der Grenzwert in der obigen Definition existiert und ist unabhängig von der Wahl der Approximierenden Folge von Treppenfunktionen.
\end{lemma}

\begin{beweis}{}
	\begin{enumerate}
		\item Wir zeigen dass
		\begin{equation*}
			I_n\coloneqq \int_a^b t_n(x)\intd x
		\end{equation*}
		eine Cauchy-Folge ist. Sei $\epsilon>0$, dann gilt:
		\begin{align*}
			|I_n-I_m|&=\left|\int_a^b t_n(x)\intd x-\int_a^b t_m(x)\intd x\right|\\
			&=\left|\int_a^b t_n(x)-t_m(x)\intd x\right|\\
			&\leq \sup_{x\in[a,b]}|t_n(x)-t_m(x)|*(b-a)\\
			&=\epsilon*(b-a)
		\end{align*}
		Hier haben wir ausgenutzt, dass $(t_n)$ in folgendem Sinne eine Cauchyfolge ist, es gilt:
		\begin{equation*}
			\forall \epsilon>0 \enspace\exists N\in\N:\sup_{x\in[a,b]}|t_n(x)-t_m(x)|<\epsilon
		\end{equation*}
		Dies zeigt die Existenz des Grenzwerts.
	\end{enumerate}
\end{beweis}
