\chapter{Zahlenreihen}
Zahlenreihen sind Folgen, die durch aufsummieren einer anderen Folge $(a_n)$ entstehen.

\begin{definition}{Reihe}
	Sei $(a_n)_{n\in\N}$ eine reelle oder komplexe Folge. Dann heißt der Ausdruck
	\begin{equation*}
		\sum\limits_{i=1}^\infty a_i
	\end{equation*}
	unendliche Reihe (unendliche Summe). Die $a_i$ heißen die Glieder der Reihe. Unter einer Partialsumme versteht man die endliche Summe
	\begin{equation*}
		b_n\coloneqq\sum\limits_{i=1}^n a_i
	\end{equation*}
	Konvergiert die Folge der Partialsummen $(b_n)$, gegen einen Grenzwert $s$, so sagt man die Reihe $\textstyle\sum_{i=1}^\infty a_i$ konvergiert. Dann setzt man
	\begin{equation*}
		\sum\limits_{i=1}^\infty a_i = s = \lim\limits_{n\to\infty}b_n
	\end{equation*}
	Besitzt $(b_n)$ keinen Grenzwert, so sagt man die Reihe ist divergent.
\end{definition}
\paragraph{Beispiele:}
\begin{enumerate}
	\item $a_k=\frac{1}{k(k-1)}$, wir betrachten die Reihe $\sum\limits_{i=2}^\infty a_i$.

	Wir wollen untersuchen, ob diese Reihe konvergiert:
	\begin{equation*}
		\sum\limits_{i=2}^\infty a_i=\frac12 + \frac{1}{2*3} + \frac{1}{3*4} + \ldots
	\end{equation*}
	Es gilt: $\frac{1}{k(k-1)}=\frac{1}{k-1}-\frac{1}{k}$, daher gilt:
	\begin{align*}
		b_n&=a_2+a_3+\ldots+a_n\\
		&=(\frac{1}{1}-\frac{1}{2})+(\frac{1}{2}-\frac{1}{3})+\ldots+(\frac{1}{n-1}-\frac{1}{n})\\
		&=\frac{1}{1}-\frac{1}{n}
	\end{align*}
	Also gilt: $\sum\limits_{i=2}^\infty a_i = 1$.

	\item Die sogenannte harmonische Reihe:
	$\sum\limits_{k=1}^\infty \frac 1k$
	Betrachte:
	\begin{equation*}
		\sum\limits_{k=1}^\infty \frac 1k=1+\frac12+\underbrace{\frac13+\frac14}_{\geq \frac12}+\underbrace{\frac15+\frac16+\frac17+\frac18}_{\geq\frac12}+\ldots
	\end{equation*}
	Dies zeigt, dass die Folge der Partialsummen nicht beschränkt ist. Die Reihe ist divergent.

	\item Die geometrische Reihe:
	$\sum\limits_{k=0}^\infty z^k \enspace(z\in\C)$

	\begin{align*}
		\sum\limits_{k=0}^n z^k &=1+z+z^2+z^3+\ldots+z^n\\
		&=\frac{1-z^{n+1}}{1-z}
	\end{align*}
	Die obige Formel liefert uns für die Konvergenz der geometrischen Reihe:
	\begin{itemize}
		\item falls $|z|<1$ gilt $\sum_{k=0}^n z^k=\frac1{1-z}$
		\item falls $|z|\geq 1$ ist die geometrische Reihe divergent.
	\end{itemize}
\end{enumerate}
Wir beschäftigen uns im Folgenden mit Kriterien für die Konvergenz von Reihen.
\begin{satz}{Notwendiges Kriterium für die Konvergenz}
	Ist die Reihe $\sum_{k=0}^\infty a_k$ konvergent, dann ist die Folge der Reihenglieder $(a_n)$ eine Nullfolge $(a_n\rightarrow0)$.
\end{satz}
\beweis
Da eine konvergente Reihe vorliegt, ist die Folge der Partialsummen $b_n=\sum_{k=0}^n a_k$ eine Cauchyfolge, d.h. $\forall \epsilon>0\enspace\exists n_0\in\N:|b_n-b_m|<\epsilon\enspace\forall n,m\geq n_0$.
\begin{equation*}
	|b_m-b_n|=\left|\sum_{k=0}^m a_k-\sum_{k=0}^n a_k\right|=\left|\sum_{k=n+1}^m a_k\right| \quad\forall m,n\geq n_0\wedge m\geq n
\end{equation*}
Im Spezialfall $m=n+1$ folgt $|b_m-b_n|=|a_{n+1}|<\epsilon$. Dies zeigt, dass $(a_k)$ eine Nullfolge ist.\hfill$\Box$
\paragraph{Bemerkungen:}
\begin{itemize}
	\item Das Notwendigkeitskriterium ist nicht hinreichend für die Konvergenz der Reihe, denn zum Beispiel divergiert die harmonische Reihe.
	\item Die Bedingung im Beweis oben ist das Cauchykriterium für Reihen. Dieses ist eine hinreichende Bedingung.
\end{itemize}

\begin{lemma}{Konvergenzkriterium}
	Eine Reihe mit nichtnegativen reellen Gliedern, bei der die Folge der Partialsummen beschränkt ist, konvergiert. Denn dann ist die Reihe monoton steigend.
\end{lemma}





\section{Alternierende Reihen}
Wir betrachten nun alternierende Reihen:
\begin{definition}{Alternierende Reihen}
	Eine Reihe heißt alternierend, wenn die Reihenglieder abwechselnd nichtnegativ ($\geq 0$) und nichtpositiv ($\leq 0$) sind.
\end{definition}
Ein hinreichendes Kriterium für die Konvergenz einer alternierenden Reihe:
\begin{satz}{Leibnitzkriterium}
	Sei $(a_n)_{n\in\N}$ eine reelle Zahlenfolge, für die $a_k\geq 0$ gilt. Und es gilt $a_k\geq a_{k+1}$ (monoton fallend) und $\lim\limits_{k\to\infty} a_k = 0$. D.h. $(a_n)$ ist eine nichtnegative monoton fallende Nullfolge. Dann ist die alternierende Reihe
	\begin{equation*}
		\sum\limits_{k=0}^\infty (-1)^k*a_k
	\end{equation*}
	konvergent.
\end{satz}
\beweis
Für jedes $\epsilon > 0$ existiert ein $n_0\in\N$ sodass $a_k<\epsilon \enspace\forall k\geq n_0$ gilt.
Wir schätzen den Abstand der Partialsummen ab. Seien $m,n\geq n_0$ und $m\geq n$, dann gilt, falls $m-n$ ungerade und $n$ ungerade ist:
\begin{align*}
	|b_m-b_n|&=|(-1)^{n+1}a_{n+1}+(-1)^{n+2}a_{n+2}+\ldots+(-1)^{m-1}a_{m-1}+(-1)^{m}a_{m}|\\
	&=a_{n+1}-\underbrace{(a_{n+2}-a_{n+3})}_{\geq 0}+\ldots+\underbrace{(a_{m-1}-a_m)}_{\geq 0}\\
	&\leq a_{n+1}\text{, falls $m-n$ ungerade}
\end{align*}
Und analog falls $n$ gerade. Dies zeigt die Konvergenz der Reihe nach dem Chauchykriterium.

\paragraph{Beispiel:}
Dieses Kriterium lässt sich auf die alternierende harmonische Reihe anwenden.
\begin{equation*}
	\sum\limits_{k=1}^\infty \frac{(-1)^{k-1}}{k}
\end{equation*}
konvergiert gegen $\ln2$.

\paragraph{Bemerkung:}
Bei unendlichen Reihen gilt im Allgemeinen kein Kommutativ- oder Assoziativgesetz.
Der Grenzwert und auch das Konvergenzverhalten kann sich bei Umordnung und Um-Klammerung der Reihenglieder ändern:
\begin{align*}
	&\sum\limits_{k=1}^\infty \frac{(-1)^{k-1}}{k} = 1-\frac12+\frac13-\frac14+\frac15-\ldots
	\intertext{Wir ordnen die Reihe um:}
	&1-\frac12+\frac13-\frac14+\underbrace{\left(\frac15+\frac17\right)}_{\geq \frac14}-\frac16+\underbrace{\left(\frac19+\frac{1}{11}+\frac{1}{13}+\frac{1}{15}\right)}_{\geq \frac14}-\ldots \longrightarrow 0
\end{align*}
Diese Reihe divergiert nun!





\section{Absolute Konvergenz}
\begin{definition}{Eigenschaft der absoluten Konvergenz}
	Eine Zahlenreihe $\sum_{k=0}^\infty a_k$ heißt absolut konvergent, falls sogar die Reihe $\sum_{k=0}^\infty |a_k|$ konvergiert.
\end{definition}
\begin{satz}{}
	Jede absolut konvergente Reihe ist konvergent.
\end{satz}
\beweis
Wir beweisen dies mit dem Cauchykriterium für Reihen:

Da $\sum_{k=0}^\infty |a_k|$ konvergiert, gibt es zu jedem $\epsilon > 0$ ein $n_0\in\N$ sodass $\forall m,n\geq n_0$ mit $m\geq n$ und Anwedung der Dreiecksungleichung gilt:

\begin{equation*}
	\left|\sum\limits_{k=n+1}^m a_k\right|\leq\sum\limits_{k=n+1}^m |a_k|<\epsilon
\end{equation*}
\hfill$\Box$

\paragraph{Bemerkung:}
Die Umkehrung der Aussage gilt nicht, denn wir wissen, dass die harmonische Reihe divergiert obwohl die alternierende harmonische Reihe konvergiert.




\subsection{Kriterien für absolute Konvergenz}
Wir werden im Folgenden einige Kriterien für absolute Konvergenz von Reihen kennenlernen.
\begin{definition}{}
	Ist $\sum_{k=1}^\infty a_k$ eine reelle Zahlenreihe und ist $\sum_{k=1}^\infty b_k$ eine konvergente, reelle Reihe, so dass:
	\begin{equation*}
		|a_k|\leq|b_k| \quad \forall k\in\N
	\end{equation*}
	dann nennt man die Reihe $\sum_{k=1}^\infty b_k$ eine konvergente Majorante.
\end{definition}

\begin{satz}{Majorantenkriterium}
	Hat eine reelle oder komplexe Zahlenreihe $\sum_{k=1}^\infty a_k$ eine konvergente Majorante, $\sum_{k=1}^\infty b_k$, dann konvergiert die Reihe $\sum_{k=1}^\infty a_k$ absolut!
\end{satz}
\beweis
Sei $s_n\coloneqq \sum_{k=1}^n a_k$ die $n$-te Partialsumme der Reihe $\sum^\infty a_k$. Dann gibt es zu jedem $\epsilon>0$ ein $n_0\in\N$, sodass für alle $m\geq n\geq n_0$ gilt:
\begin{equation*}
	|s_m-s_n|=\sum_{k=n+1}^m |a_k|\leq \sum_{k=n+1}^m b_k <\epsilon \quad\Box
\end{equation*}

\par
Es gibt auch ein Minorantenkriterium:

Sei $a_k\geq c_k\geq 0$ für alle $k\in\N$ und ist $c_k$ divergent, so ist $a_k$ ebenfalls divergent.

\paragraph{Beispiele:}
\begin{itemize}
	\item Wir betrachten die Reihe $\sum_{k=1}^\infty \frac1{k^2}$:

	Da $\frac1{k^2}\leq\frac{1}{k*(k-1)}$ für $k\geq 2$, ist die Reihe $\sum_{k=1}^\infty \frac{1}{k*(k-1)}$ eine Majorante, für die wir die Konvergenz bereits gezeigt haben. Also konvergiert auch $\sum_{k=1}^\infty \frac1{k^2}$.

	\item Für die Reihe $\sum_{k=1}^\infty \frac{1}{\sqrt k}$ ist $\sum_{k=1}^\infty \frac{1}{k}$ eine divergente Minorante, also divergiert auch $\sum_{k=1}^\infty \frac{1}{\sqrt k}$.

	\item Die Reihe $\sum_{k=1}^\infty \frac{1}{2^k+3^k}$. Wegen $\frac{1}{2^k+3^k}<\frac{1}{3^k}=\left(\frac13\right)^k$ ist die geometrische Reihe mit $z=\frac13$ eine konvergente Majorante.
\end{itemize}

Wir formulieren nun zwei weitere hinreichende Kriterien für (sogar absolute) Konvergenz. Das Wurzelkriterium und das Quotientenkriterium.

\begin{satz}{Wurzelkriterium}
	Die Reihe $\sum_{k=1}^\infty a_k$ ist absolut konvergent, falls ein $q\in\R$ mit $0\leq q<1$ und ein $k_0\in\N$ existiert so dass
	\begin{equation*}
		\sqrt[k]{|a_k|}\leq q
	\end{equation*}
	für alle $k\geq k_0$ gilt.
\end{satz}
\beweis
Es gilt $|a_k|\leq q^k$ und $\sum_{k=1}^\infty q^k$ ist konvergent (geometrische Reihe). \hfill $\Box$


\begin{satz}{Quotientenkriterium}
	Sei $a_k\neq 0$ für $k\geq k_0$. Die Reihe $\sum_{k=1}^\infty a_k$ ist absolut konvergent, falls ein $q\in\R$ mit $0\leq q<1$ und ein $k_0\in\N$ existiert so dass
	\begin{equation*}
		\left|\frac{a_{k+1}}{a_k}\right|\leq q
	\end{equation*}
	für alle $k\geq k_0$ gilt.
\end{satz}
\beweis
Aus der Ungleichung folgt, dass
\begin{equation*}
	|a_{k+1}|\leq q*|a_k|\leq q^2*|a_{k-1}|\leq \ldots \leq q^{k+1-k_0}|a_{k_0}|
\end{equation*}
Daher ist
\begin{equation*}
	\sum_{k=k_0}^\infty |a_k|\leq \frac{|a_{k_0}|}{q^{k_0}}* \sum_{k=k_0}^\infty q^k
\end{equation*}
und wir haben die geometrische Reihe als konvergente Majorante gefunden.\hfill$\Box$

\paragraph{Bemerkungen:}
Falls für alle $k\in\N$ gilt:
\begin{itemize}
	\item $\sqrt[k]{|a_k|} \geq 1$, so divergiert die Reihe $\sum a_k$.
	\item $\left|\frac{a_{k+1}}{a_k}\right|\geq 1$, so divergiert die Reihe $\sum a_k$.
	\item $\lim\limits_{k\to\infty} \sqrt[k]{|a_k|} = q< 1$, so konvergiert die Reihe absolut.
	\item $\lim\limits_{k\to\infty} \sqrt[k]{|a_k|} = q> 1$, so divergiert die Reihe.
	\item $\lim\limits_{k\to\infty} \left|\frac{a_{k+1}}{a_k}\right| = q< 1$, so konvergiert die Reihe absolut.
	\item $\lim\limits_{k\to\infty} \left|\frac{a_{k+1}}{a_k}\right| = q> 1$, so divergiert die Reihe.
\end{itemize}
Ist $q=1$ so ist die Bedingung nicht erfüllt, die Reihe kann sowohl konvergieren als auch divergieren, das Kriterium macht keine Aussage über das Konvergenzverhalten.

\paragraph{Beispiele:}
\begin{itemize}
	\item Eine wichtige Reihe in der Mathematik ist die sogenannte Exponentialreihe:
	\begin{equation*}
		\exp(x)=\sum\limits_{k=0}^\infty \frac{x^k}{k!}
	\end{equation*}
	für ein $x\in\C$. Es gilt
	\begin{equation*}
		\sum\limits_{k=0}^\infty \frac{x^k}{k!} = \frac{1}{1}+\frac{x}{1}+\frac{x^2}{2}+\frac{x^3}{6}+\frac{x^4}{24}+\frac{x^5}{120}+\ldots
	\end{equation*}
	Um die Konvergenz der Exponentialreihe zu beweisen, wenden wir das Quotientenkriterium an:
	\begin{equation*}
		\left|\frac{a_{k+1}}{a_k}\right|=\left|\frac{\frac{x^{k+1}}{(k+1)!}}{\frac{x^k}{k!}}\right|=\left|\frac{x^k}{k+1}\right|\longrightarrow 0
	\end{equation*}
	Damit ist das Quotientenkriterium anwendbar und wir haben gezeigt, dass die Exponentialreihe konvergiert.

	\item Die harmonische Reihe $\sum_{k=1}^\infty \frac 1k$:
	Das Quotientenkriterium führt hier auf
	\begin{equation*}
		\left|\frac{a_{n+1}}{a_n}\right|\left|\frac{\frac{1}{k+1}}{\frac 1k}\right| = \left|\frac{k}{k+1}\right|
	\end{equation*}
	Dieser Term ist zwar kleiner als $1$, das Quotientenkriterium kann aber trotzdem nicht angewendet werden, da $\lim_{k\to\infty}\left|\frac{k}{k+1}\right|=1$!

	Das Wurzelkriterium führt ebenfalls auf $\lim\sqrt[k]{\frac 1k}=1$. Es macht ebenfalls keine Aussage.

	Mit den Kriterien kann weder auf Konvergenz noch auf Divergenz geschlossen werden.
\end{itemize}

\paragraph{Bemerkung:}
Wir hatten gesehen, dass der Wert (und auch das Konvergenzverhalten selbst) einer Reihe sich ändern kann, wenn man die Reihenglieder umordnet. Allerdings gilt im Fall von absoluter Konvergenz:

\begin{satz}{Umordnungssatz}
	Sei $\sum_{k=1}^\infty a_k$ absolut Konvergent und $\tau:\N\rightarrow\N$ eine Bijektion.
	Dann konvergiert auch die umgeordnete Reihe
	\begin{equation*}
		\sum\limits_{k=1}^\infty a_{\tau(k)}
	\end{equation*}
	Es gilt weiter, dass die Grenzwerte gleich sind:
	\begin{equation*}
		\sum_{k=1}^\infty a_k=\sum\limits_{k=1}^\infty a_{\tau(k)}
	\end{equation*}
\end{satz}

\subsection{Cauchyprodukt von Reihen}
Das Cauchyprodukt erlaubt es, das Produkt von zwei absolut konvergenten Reihen wieder als absolut konvergente Reihe darzustellen. Die Idee dabei ist, die Summanden nach folgendem Schema diagonal aufzusummieren:
\begin{equation*}
	\begin{array}{ccccc}
		\color{red}a_0b_0 & \color{orange}a_1b_0 & \color{blue}a_2b_0 & a_3b_0 & \cdots\\
		\color{orange}a_0b_1 & \color{blue}a_1b_1 & a_2b_1 & a_3b_1 & \cdots\\
		\color{blue}a_0b_2 & a_1b_2 & a_2b_2 & a_3b_2 & \cdots\\
		a_0b_3 & a_1b_3 & a_2b_3 & a_3b_3 & \cdots\\
		\vdots & \vdots & \vdots & \vdots & \ddots
	\end{array}
\end{equation*}
Es soll also gelten:
\begin{align*}
	\left(\sum\limits_{k=0}^\infty a_k\right)*\left(\sum\limits_{k=0}^\infty b_k\right)
	= &({\color{red} a_0b_0})+({\color{orange} a_1b_0+a_0b_1})+({\color{blue} a_2b_0+a_1b_1+a_0b_2})+\ldots\\
	&\ldots+\sum\limits_{j=0}^n a_{n-j}b_j+\ldots
\end{align*}

\begin{satz}{Cauchyprodukt}
	Seien $\sum_{k=0}^\infty a_k$ und $\sum_{k=0}^\infty b_k$ zwei absolut konvergente Reihen. Und sei $(c_n)$
	\begin{equation*}
		c_n=\sum\limits_{j=0}^n a_{n-j}b_j
	\end{equation*}
	Dann konvergiert das Cauchyprodukt $\sum_{k=0}^\infty c_k$ absolut. Außerdem gilt für die Grenzwerte:
	\begin{equation*}
		\sum\limits_{k=0}^\infty c_k = \left(\sum\limits_{k=0}^\infty a_k\right)*\left(\sum\limits_{k=0}^\infty b_k\right)
	\end{equation*}
\end{satz}

\paragraph{Anwendung: Funktionalgleichung der Exponentialfunktion}
Wir betrachten die beiden absolut konvergenten Reihen mit $x,y\in\C$:
\begin{itemize}
	\item $e^x=\exp(x)=\sum\limits_{k=0}^\infty \frac{x^k}{k!}$
	\item $e^y=\exp(y)=\sum\limits_{k=0}^\infty \frac{y^k}{k!}$
\end{itemize}
Das Cauchyprodukt der beiden Reihen ist
\begin{align*}
	\sum\limits_{j=0}^n c_n \text{ mit } c_n&=\sum\limits_{j=0}^n a_{n-j}b_j\\
	&=\sum\limits_{j=0}^n \left(\frac{x^{n-j}}{(n-j!)}*\frac{y^j}{j!}\right)\\
	&=\sum\limits_{j=0}^n \frac{n!}{n!(n-j)!j!}*x^{n-j}*y^j\\
	&=\frac{1}{n!} \sum\limits_{j=0}^n \binom{n}{j} *x^{n-j}*y^j\\
	&=\frac{1}{n!} (x+y)^n
\end{align*}
Das Cauchyprodukt ist dann
\begin{equation*}
	\sum\limits_{j=0}^n \frac{(x+y)^n}{n!}
\end{equation*}
Dies zeigt: $e^x*e^y=e^{x+y}$\hfill$\Box$
