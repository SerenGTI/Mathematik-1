\chapter{Zahlenreihen}
Zahlenreihen sind Folgen, die durch aufsummieren einer anderen Folge $(a_n)$ entstehen.

\begin{definition}{Reihe}
	Sei $(a_n)_{n\in\N}$ eine reelle oder komplexe Folge. Dann heißt der Ausdruck
	\begin{equation*}
		\sum\limits_{i=1}^\infty a_i
	\end{equation*}
	unendliche Reihe (unendliche Summe). Die $a_i$ heißen die Glieder der Reihe. Unter einer Partialsumme versteht man die endliche Summe
	\begin{equation*}
		b_n\coloneqq\sum\limits_{i=1}^n a_i
	\end{equation*}
	Konvergiert die Folge der Partialsummen $(b_n)$, gegen einen Grenzwert $s$, so sagt man die Reihe $\textstyle\sum_{i=1}^\infty a_i$ konvergiert. Dann setzt man
	\begin{equation*}
		\sum\limits_{i=1}^\infty a_i = s = \lim\limits_{n\to\infty}b_n
	\end{equation*}
	Besitzt $(b_n)$ keinen Grenzwert, so sagt man die Reihe ist divergent.
\end{definition}
\paragraph{Beispiele:}
\begin{enumerate}
	\item $a_k=\frac{1}{k(k-1)}$, wir betrachten die Reihe $\sum\limits_{i=2}^\infty a_i$.

	Wir wollen untersuchen, ob diese Reihe konvergiert:
	\begin{equation*}
		\sum\limits_{i=2}^\infty a_i=\frac12 + \frac{1}{2*3} + \frac{1}{3*4} + \ldots
	\end{equation*}
	Es gilt: $\frac{1}{k(k-1)}=\frac{1}{k-1}-\frac{1}{k}$, daher gilt:
	\begin{align*}
		b_n&=a_2+a_3+\ldots+a_n\\
		&=(\frac{1}{1}-\frac{1}{2})+(\frac{1}{2}-\frac{1}{3})+\ldots+(\frac{1}{n-1}-\frac{1}{n})\\
		&=\frac{1}{1}-\frac{1}{n}
	\end{align*}
	Also gilt: $\sum\limits_{i=2}^\infty a_i = 1$.

	\item Die sogenannte harmonische Reihe:
	$\sum\limits_{k=1}^\infty \frac 1k$
	Betrachte:
	\begin{equation*}
		\sum\limits_{k=1}^\infty \frac 1k=1+\frac12+\underbrace{\frac13+\frac14}_{\geq \frac12}+\underbrace{\frac15+\frac16+\frac17+\frac18}_{\geq\frac12}+\ldots
	\end{equation*}
	Dies zeigt, dass die Folge der Partialsummen nicht beschränkt ist. Die Reihe ist divergent.

	\item Die geometrische Reihe:
	$\sum\limits_{k=0}^\infty z^k \enspace(z\in\C)$

	\begin{align*}
		\sum\limits_{k=0}^n z^k &=1+z+z^2+z^3+\ldots+z^n\\
		&=\frac{1-z^{n+1}}{1-z}
	\end{align*}
	Die obige Formel liefert uns für die Konvergenz der geometrischen Reihe:
	\begin{itemize}
		\item falls $|z|<1$ gilt $\sum_{k=0}^n z^k=\frac1{1-z}$
		\item falls $|z|\geq 1$ ist die geometrische Reihe divergent.
	\end{itemize}
\end{enumerate}
Wir beschäftigen uns im Folgenden mit Kriterien für die Konvergenz von Reihen.
\begin{satz}{Notwendiges Kriterium für die Konvergenz}
	Ist die Reihe $\sum_{k=0}^\infty a_k$ konvergent, dann ist die Folge der Reihenglieder $(a_n)$ eine Nullfolge $(a_n\rightarrow0)$.
\end{satz}
\beweis
Da eine konvergente Reihe vorliegt, ist die Folge der Partialsummen $b_n=\sum_{k=0}^n a_k$ eine Cauchyfolge, d.h. $\forall \epsilon>0\enspace\exists n_0\in\N:|b_n-b_m|<\epsilon\enspace\forall n,m\geq n_0$.
\begin{equation*}
	|b_m-b_n|=\left|\sum_{k=0}^m a_k-\sum_{k=0}^n a_k\right|=\left|\sum_{k=n+1}^m a_k\right| \quad\forall m,n\geq n_0\wedge m\geq n
\end{equation*}
Im Spezialfall $m=n+1$ folgt $|b_m-b_n|=|a_{n+1}|<\epsilon$. Dies zeigt, dass $(a_k)$ eine Nullfolge ist.\hfill$\Box$
\paragraph{Bemerkungen:}
\begin{itemize}
	\item Das Notwendigkeitskriterium ist nicht hinreichend für die Konvergenz der Reihe, denn zum Beispiel divergiert die harmonische Reihe.
	\item Die Bedingung im Beweis oben ist das Cauchykriterium für Reihen. Dieses ist eine hinreichende Bedingung.
\end{itemize}
