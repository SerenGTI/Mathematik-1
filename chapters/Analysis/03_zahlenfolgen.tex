\chapter{Zahlenfolgen und Zahlenreihen}
Wir betrachten in diesem Abschnitt den Spezialfall, dass der metrische Raum $X$ gleich $\R$ oder $\C$ ist. Wir verwenden dabei die Metrik zwischen zwei Zahlen $\varrho(x,y)=|x-y|$. Wir sprechen in diesem Fall von Zahlenfolgen.

\section{Rechnen mit Grenzwerten}
\begin{satz}{}
	Seien $(x_n)$ und $(y_n)$ zwei konvergente Folgen in den reellen Zahlen $\R$ mit
	\begin{equation*}
		x\coloneqq \lim\limits_{n\to\infty} x_n \text{ und } y\coloneqq \lim\limits_{n\to\infty} y_n
	\end{equation*}
	für die für alle $n\in\N$ gilt $x_n\leq y_n$, dann folgt daraus $x\leq y$.
\end{satz}
\beweis
Angenommen $x > y$:\\
Wähle $\epsilon =\frac{x-y}2$, dann gibt es ein $n\in\N$, so dass
\begin{equation*}
	|x-x_n|<\epsilon \text{ und } |y_n-y|<\epsilon
\end{equation*}
gilt. Es folgt daraus
\begin{equation*}
	x-x_n<\epsilon \Leftrightarrow x-\epsilon<x_n \text{ und } y-y_n<\epsilon \Leftrightarrow y_n<\epsilon + y
\end{equation*}
Insgesamt erhält man
\begin{align*}
	x-\epsilon &< x_n\leq y_n <y+\epsilon\\
	x-y&<2\epsilon
\end{align*}
Widerspruch!

\begin{satz}{Rechenregeln für Limites}
	Seien $(x_n), (y_n)$ konvergente Folgen in $\R$ oder $\C$. Sei außerdem $\alpha\in\R$ oder $\C$, dann gilt

	\begin{itemize}
		\item $\left|\lim\limits_{n\to\infty}x_n\right|=\lim\limits_{n\to\infty}|x_n|$
		\item $\lim\limits_{n\to\infty}(\alpha * x_n) = \alpha \lim\limits_{n\to\infty}x_n$
		\item $\lim\limits_{n\to\infty}(x_n+y_n)=\lim\limits_{n\to\infty}x_n+\lim\limits_{n\to\infty}y_n$
		\item $\lim\limits_{n\to\infty}(x_n*y_n)=\lim\limits_{n\to\infty}x_n*\lim\limits_{n\to\infty}y_n$
		\item und falls $x_n\geq 0$, gilt mit $p\in\N$:
		\begin{itemize}
			\item $ \lim\limits_{n\to\infty}\sqrt[p]{x_n}=\sqrt[p]{\lim\limits_{n\to\infty}x_n}$
			\item $\lim\limits_{n\to\infty}x_n^p = \left(\lim\limits_{n\to\infty}x_n\right)^p$
		\end{itemize}
	\end{itemize}
\end{satz}
\beweis
Nur für die Aussage über die Addition:
$$\lim\limits_{n\to\infty}(x_n+y_m)=\lim\limits_{n\to\infty}x_n+\lim\limits_{n\to\infty}y_n$$

Es ist $\lim\limits_{n\to\infty}(x_n+y_m) = x+y$ zu zeigen, wobei $x_n\to x$ und $y_n\to y$ gilt.

Sei $\epsilon>0$. Es gibt ein $n_0\in\N$, so dass $|x_n-x|<\frac\epsilon2$ und $|y_n-y|<\frac\epsilon2$ für alle $n\geq n_0$.

Dann gilt mit der Dreiecksungleichung: $$\left|(x_n+y_n)-(x+y)\right|=\left|(x_n-x)+(y_n-y)\right|\leq |x_n-x|+|y_n-y|<\frac\epsilon2+\frac\epsilon2=\epsilon$$


\section{Konvergenzkriterien}
\begin{satz}{Einschließungskriterium}
	Seien $(x_n), (y_n), (z_n)$ reelle Zahlenfolgen. Weiterhin gelte $x_n\leq z_n\leq y_n$ für alle $n\in\N$ und $\lim\limits_{n\to\infty} x_n=x=\lim\limits_{n\to\infty} y_n$.

	Dann konvergiert auch $z_n$ mit $\lim\limits_{n\to\infty} z_n = x$.
\end{satz}
\beweis
Sei $\epsilon>0$. Wähle $n_0\in\N$ so, dass $|x_n-x|<\epsilon$ und $|y_n-y|<\epsilon$. Dann gilt für alle natürlichen Zahlen $n\geq n_0$:
\begin{equation*}
	|x-z_n|=\begin{cases}
		z_n-x\leq y_n-x \leq |y_n-x| <\epsilon \text{, falls $x<z_n$}\\
		x-z_n\leq x-x_n \leq |x-x_n| <\epsilon \text{, falls $x\leq z_n$}
	\end{cases}
\end{equation*}
\hfill $\Box$

\begin{lemma}{}\label{lemma:abschaetzung}
	Es gilt: $(1+x)^n>\frac{n^2}{4}x^2$, für $x>0, n\geq 2$
\end{lemma}
\beweis
Für $n\geq2$ gilt:
\begin{align*}
	(1+x)^n=\sum\limits_{i=0}^n\binom{n}{i}x^i>\binom{n}{2}x^2&=\frac{n(n+1)}{2}x^2\\
	&=\left(\frac{n^2}{2}-\frac n2 \right)x^2\\
	&=\left(\frac{n^2}4 + \underbrace{\frac{n^2-2n}{4}}_{\geq 0}\right)x^2\\
	&\geq \frac{n^2}4 x^2
\end{align*}

\paragraph{Beispiel:} zur Berechnung von Grenzwerten mit dem Einschließungskriterium.
\begin{itemize}
	\item Mit Hilfe dieser Ungleichung zeigen wir, dass folgende Aussage gilt:
	$$\lim\limits_{n\to\infty}\sqrt[n]{n}=1$$

	Wir setzen in die Formel aus \autoref{lemma:abschaetzung} $x=\sqrt[n]{n}-1\geq 0$ ein.

	Wir erhalten
	\begin{align*}
		n&>\frac{n^2}{4}*(\sqrt[n]{n}-1)^2\\
		\sqrt{n}&>\frac n2(\sqrt[n]{n}-1)\\
		\frac2{\sqrt{n}}+1&>\sqrt[n]{n}>1
	\end{align*}
	Es gilt $\lim\limits_{n\to\infty}\frac2{\sqrt{n}}+1 = 1$ und mit dem Einschließungskriterium folgt $\lim\limits_{n\to\infty}\sqrt[n]{n}=1$.

	\item Ähnlich kann man zeigen, dass $\lim\limits_{n\to\infty}\sqrt[n]{a}=1$ für $a>1$ ist.

	Dazu setzen wir in die Formel von oben $x=\sqrt[n]{a}-1$ ein.

	\begin{align*}
		a&>\frac{n^2}{4}*(\sqrt[n]{a}-1)^2\\
		\sqrt{a}&>\frac n2*(\sqrt[n]{a}-1)\\
		\frac{2\sqrt{a}}{n}+1&>\sqrt[n]{a}>1
	\end{align*}
	Es gilt wieder: $\lim\limits_{n\to\infty}\frac{2\sqrt{a}}{n}+1=1$, daraus folgt die Behauptung.
\end{itemize}

\paragraph{Bemerung:} Gilt auch für $0<a\leq 1$.

\subsection{Monotone Folgen}
\begin{definition}{}
	Eine reelle Zahlenfolge $(x_n)_{n\in\N}$ heißt
	\begin{description}
		\item[streng monoton wachsend], falls $x_n<x_{n+1}$
		\item[streng monoton fallend], falls $x_n>x_{n+1}$
		\item[monoton wachsend], falls $x_n\leq x_{n+1}$
		\item[monoton fallend], falls $x_n\geq x_{n+1}$
	\end{description}
	jeweils für alle $n\in\N$ gilt.
\end{definition}

\begin{satz}{}
	Beschränkte, monotone Folgen sind konvergent.
\end{satz}
\beweis
Wir zeigen die Aussage für monoton wachsende Folgen:

Gelte also $x_n\geq x_{n+1}<c$ für alle $n\in\N$. Dann existiert auch eine kleinste obere Schranke
\begin{equation*}
	c_{\text{min}}\coloneqq \sup\set{x_n}{n\in\N}
\end{equation*}
(Allgemein ist das Supremum $\sup M$ einer Menge $M\subseteq\R$ von reellen Zahlen die kleinste reelle Zahl $s$ für die gilt: $s\geq m\quad\forall m\in M$.

Gibt es keine obere Schranke, dann existiert auch kein Supremum. Es folgt aus der Vollständigkeit der reellen Zahlen, dass jede beschränkte Menge von reellen Zahlen ein Supremum besitzt.)

\par\medskip

Zu jedem $\epsilon>0$ existiert ein $n_0\in\N$, so dass $c_{\text{min}}-\epsilon<x_{n_0}$. Sonst wäre $c_{\text{min}}$ nicht die kleinste obere Schranke.

Dann gilt aber wegen der Monotonie der Folge, dass $c_{\text{min}}-\epsilon<x_n$ für alle $n\geq n_0$. Dann folgt $|c_{\text{min}}-x_n|=c_{\text{min}}-x_n<\epsilon$.


\begin{definition}{Teilfolgen}
	Sei $(n_k)_{k\in\N}$ eine streng monoton wachsende Folge von natürlichen Zahlen ($n_k\in\N$) und sei $(a_n)_{n\in\N}$ eine Folge in einem metrischen Raum, $a_n\in X$. Dann ist $(a_{n_k})_{k\in\N}$ eine Teilfolge der Folge $(a_n)$
\end{definition}
\paragraph{Beispiele:}
Sei $a_n=(-1)^n$. Dann ist $a_n$ divergent, aber die Teilfolgen
\begin{itemize}
	\item $a_{2n}=(-1)^{2n}=1$
	\item $a_{2n+1}=(-1)^{2n+1}=-1$
\end{itemize}
sind konvergent, sogar konstant. Andere Teilfolgen sind z.B. $n_k=3k$ (ebenfalls divergent).

\begin{definition}{Häufungspunkt}
	Sei $(a_n)_{n\in\N}$ eine Folge reeller oder auch komplexer Zahlen. Ein Element $a\in X$ heißt Häufungspunkt der Folge $(a_n)$, falls es eine gegen $a$ konvergente Teilfolge von $(a_n)$ gibt.
\end{definition}
\paragraph{Beispiel:}
$-1$ und $1$ sind die Häufungspunkte von $a_n=(-1)^n$.

\paragraph{Bemerkung:}
Es gilt der Satz von Bolzano-Weierstraß: Jede beschränkte, reelle oder komplexe Zahlenfolge hat eine konvergente Teilfolge.

\subsection{Cauchy'sches Konvergenzkriterium}
Wir kommen nun zum Cauchy'sches Konvergenzkriterium für Folgen. Im Unterschied zur Definition von Konvergenz, in der der Grenzwert vorkommt, ein \glqq inneres\grqq Kriterium für Konvergenz, d.h. um dieses Kriterium für Konvergenz entscheiden zu können, muss man den Grenzwert nicht kennen.

\begin{satz}{Cauchy'sches Konvergenzkriterium}
	Eine Folge reeller bzw. komplexer Zahlen $(a_n)_{n\in\N}$ konvergiert genau dann, wenn sie eine sog. Cauchyfolge ist. D.h. falls für alle $\epsilon>0$ ein $n_0\in\N$ existiert, sodass
	\begin{equation*}
		|a_n-a_m|<\epsilon \quad\forall n,m\geq n_0
	\end{equation*}
	gilt. (Der Abstand zweier Folgenglieder ist kleiner als Epsilon)
\end{satz}
\beweis
(nur eine Beweisskizze)

\begin{description}
	\item[\glqq$\Rightarrow$\grqq] Sei $(a_n)_{n\in\N}$ eine konvergente Folge und $\epsilon>0$ mit $\lim a_n=a$. Dann gibt es ein $n_0$, sodass $|a_n-a|<\sfrac\epsilon2$ für alle $n\geq n_0$ gilt. Dann gilt für je zwei $n,m\in\N$ mit $n,m\geq n_0$
	\begin{equation*}
		|a_n-a_m|\geq |a_n-a|+|a_m-a|<\frac\epsilon2+\frac\epsilon2=\epsilon
	\end{equation*}

	\item[\glqq$\Leftarrow$\grqq] Üblicherweise wird die Vollständigkeit eines metrischen Raums so definiert:

	\glqq Ein metrischer Raum ist vollständig, wenn in ihm jede Cauchyfolge konvergiert.\grqq

	Führt man die reellen Zahlen nicht axiomatisch ein, sondern gibt dafür ein Modell an (z.B. unendliche Dezimalbrüche), so kann man diese Aussage auch beweisen.\hfill$\Box$
\end{description}
