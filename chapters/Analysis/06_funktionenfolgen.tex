\chapter{Funktionenfolgen und -Reihen}
Wir betrachten hier Folgen
\begin{equation*}
	f_n:D\rightarrow \R \text{ oder } \C, x\mapsto f_n(x)
\end{equation*}
mit $D\subseteq \R$ oder $\C$. Wir wollen uns natürlich mit der Frage der Konvergenz von Funktionenfolgen und -reihen befassen. Es gibt zwei Arten von Konvergenz für Folgen von Funktionen:

\begin{definition}{Punktweise Konvergenz}
	Sei $D\subseteq \R,\C$ der Definitionsbereich der Funktionen $f$ und $f_n, n\in\N$ wobei $f_n:D\rightarrow \R,\C$ und $f:D\rightarrow \R,\C$.

	Die Folge $(f_n)$ heißt punktweise konvergent, falls für alle $x\in D$ gilt: $f_n(x)\to f(x)$.

	Oder mit anderen Worten:
	\begin{equation*}
		\forall x\in D\enspace \forall \epsilon\geq0 \enspace\exists n_0\in\N : |f_n(x)-f(x)|<\epsilon
	\end{equation*}
	Man schreibt dann auch $f_n(x)\to f(x)\enspace \forall x\in D$ für punktweise Konvergenz.
\end{definition}
\begin{definition}{Gleichmäßige Konvergenz}
	Die Funktionenfolge $(f_n)$ heißt gleichmäßig konvergent gegen die Funktion $f$, falls gilt:
	\begin{equation*}
		\forall\epsilon\geq0\enspace\exists n_0\in\N\enspace\forall x\in D\enspace   : |f_n(x)-f(x)|<\epsilon
	\end{equation*}
	Man schreibt dann auch $f_n\overset{glm}\longrightarrow f$ für gleichmäßige Konvergenz.
\end{definition}

\paragraph{Beispiele:}
\begin{itemize}
	\item \begin{multicols}{2}
	Sei $f_n(x)=1+\frac 1n * x$ mit dem Definitionsbereich $D=[-1;1]$. Sei $f(x)=1$ für $f:D\rightarrow \R$: Diese Funktion konvergiert gleichmäßig gegen $f\equiv 1$, denn

	\begin{equation*}
		|f_n(x)-f(x)|\leq |f_n(x)-1|
	\end{equation*}

	\columnbreak
		\begin{center}
			\begin{easyfunction}{-1}{1}{0}{2}{1}
				\draw[->] (-1.2,0) -- (1.2,0) node[right] {$x$};
				\easyfunctionxscale{-1}{1}
				\draw[->] (0,0) -- (0,2.2) node[above] {$f(x)$};

				\draw[line width=0.2mm,scale=1,domain=-1:1,smooth,variable=\x,blue] plot ({\x},{\x+1})
						node[above right] {$f_1(x)$};
				\draw[line width=0.2mm,scale=1,domain=-1:1,smooth,variable=\x,blue] plot ({\x},{0.5*\x+1})
						node[above right] {$f_2(x)$};
				\draw[line width=0.2mm,scale=1,domain=-1:1,smooth,variable=\x,blue] plot ({\x},{0.25*\x+1})
						node[right] {$f_4(x)$};
				\draw[line width=0.5mm,scale=1,domain=-1:1,smooth,variable=\x,red] plot ({\x},{1})
						node[below right] {$f(x)$};
			\end{easyfunction}
		\end{center}
	\end{multicols}

	\item	%\begin{multicols}{2}
		$D=[-1;1], f_n=x^{2n}$

		Die Funktionenfolge konvergiert punktweise gegen die Funktion
		\begin{equation*}
			f:D\rightarrow \R, f(x)=\begin{cases}
				1, \text{ falls }|x|=1\\
				0, \text{ sonst}
			\end{cases}
		\end{equation*}

		%\columnbreak

		\begin{center}
			\begin{easyfunction}{-1}{1}{0}{2}{1}
				\draw[->] (-1.2,0) -- (1.2,0) node[right] {$x$};
				\easyfunctionxscale{-1}{1}
				\draw[->] (0,0) -- (0,1.2) node[above] {$f(x)$};

				\draw[line width=0.2mm,scale=1,domain=-1:1,smooth,variable=\x,blue] plot ({\x},{\x*\x})
						node[above right] {};
				\draw[line width=0.2mm,scale=1,domain=-1:1,smooth,variable=\x,blue] plot ({\x},{\x*\x*\x*\x})
						node[above right] {};
				\draw[line width=0.2mm,scale=1,domain=-1:1,smooth,variable=\x,blue] plot ({\x},{\x*\x*\x*\x*\x*\x})
						node[right] {};
			\end{easyfunction}
		\end{center}
	% \end{multicols}
	Wir betrachten, dass die Grenzfunktion $f$ nicht stetig ist. Die gleichmäßige Konvergenz ist eine stärkere Eigenschaft als punktweise Konvergenz. Aus gleichmäßiger Konvergenz folgt punktweise Konvergez aber nicht umgekehrt.
	Gleichmäßigkeit der Funktionenfolge garantiert die Stetigkeit der Limesfunktion.
\end{itemize}

\begin{satz}{Stetigkeit der Limesfunktion}
	konvergiert eine Folge stetiger Funktionen gleichmäßig, dann ist die Limesfunktion stetig.
\end{satz}
\beweis
Angenommen $f_n\overset{glm}\longrightarrow f$. Sei $\epsilon > 0, x_0\in\R$. Wähle $n_0\in\N$ so, dass
\begin{equation*}
	|f_{n_0}(x)-f(x)|<\frac \epsilon3
\end{equation*}
für alle $x\in D$ gilt.

Wähle $\delta>0$ so, dass $|f_{n_0}(x)-f_{n_0}(x_0)|<\frac\epsilon3$ für alle $x\in D$ mit $|x-x_0|<\delta$ ist. Dann gilt:
\begin{align*}
	|f(x)-f(x_0)|&\leq \underbrace{|f(x)-f_{n_0}(x)|}_{<\sfrac \epsilon3}+\underbrace{|f_{n_0}(x)-f_{n_0}(x_0)|}_{<\sfrac \epsilon3}+\underbrace{|f_{n_0}(x_0)-f(x_0)|}_{<\sfrac \epsilon3}\\
							 &< \epsilon
\end{align*}
Zeigt die $\epsilon-\delta$-Stetigkeit der Grenzfunktion $f$ am Punkt $X_0$.

\paragraph{Bemerkung}
Für Reihen von Funktionen
\begin{equation*}
	x\mapsto \sum\limits_{n=0}^\infty f_n(x)
\end{equation*}
definieren wir gleichmäßige bzw. punktweise Konvergenz so, dass die Folge der Partialsummen
\begin{equation*}
	g_k(x)=\sum\limits_{n=0}^k f_n(x)
\end{equation*}
gleichmäßig beziehungsweise punktweise konvergiert.

\begin{satz}{Kriterien für gleichmäßige Konvergenz von Funktionenfolgen}
	\begin{itemize}
		\item \textbf{Cauchy-Kriterium:} falls es zu jedem $\epsilon>0$ einen Index $n_0\in\N$ gibt, sodass für alle $x\in D$ gilt:
		\begin{equation*}
			|f_n(x)-f_m(x)|<\epsilon
		\end{equation*}
		für alle $m,n\geq n_0$, dann exisitert ein $f: D\rightarrow\R$ mit
		\begin{equation*}
			f\overset{glm}=\lim\limits_{n\to\infty}f_n \rightsquigarrow f_n\overset{glm}\longrightarrow f
		\end{equation*}
		\item \textbf{Weierstraß-Majorantenkriterium:} falls für eine Reihe von Funktionen gilt:
		\begin{equation*}
			|f_n(x)|\leq c_n \text{ und } \sum\limits_{n=1}^\infty c_n \text{ ist konvergent}
		\end{equation*}
		dann konvergieren die Reihen $\sum\limits_{n=1}^\infty f_n$ und auch $\sum\limits_{n=1}^\infty |f_n|$ gleichmäßig.
	\end{itemize}
\end{satz}

Wir kommen nun zu einer der wichtigsten Anwendung der Funktionenfolgen und -reihen:
\section{Potenzreihen}
Eine wichtige Rolle in der gesamten Mathematik spielen die Potenzreihen, dur die eine große Klasse von Funktionen beschrieben wird.

\begin{definition}{Potenzreihe}
	Sei $z\in\C, (a_k)_{k\in\N_0}$ eine Folge komplexer Zahlen.
	Dann heißt
	\begin{equation*}
		z\mapsto \sum\limits_{k=0}^\infty a_k(z-z_0)^k
	\end{equation*}
	eine \emph{Potenzreihe} um $z_0$ mit den Koeffizienten $a_k$. Sie ist für alle $z,z_0\in\C$ definiert, für die die Reihe konvergiert.
	Den Punkt $z_0\in\C$ nennt man den Entwicklungspunkt der Potenzreihe.

	Falls $z,z_0\in\R$ und alle $a_k\in\R$ sagt man auch reelle Potenzreihe.
\end{definition}
\paragraph{Beispiele:}
\begin{itemize}
	\item Die Exponentialreihe: $\exp(z)=\sum\limits_{k=0}^\infty \frac{z^k}{k!}$ konvergiert für alle $z\in\C$
	\item Die geometrische Reihe: $\sum\limits_{k=0}^\infty z^k$
	\item $\sin(z)\coloneqq\sum\limits_{k=0}^\infty \frac{(-1)^{k}}{(2k+1)!}z^{2k+1}$ konvergiert für alle $z\in\C$
	\item $\cos(z)\coloneqq\sum\limits_{k=0}^\infty \frac{(-1)^{k}}{(2k)!}z^{2k}$ konvergiert für alle $z\in\C$
	\item $\ln(1+x)\coloneqq\sum\limits_{k=0}^\infty \frac{(-1)^{k-1}}{k}x^{k}$ konvergiert für $-1<x\leq 1$
	\item \textbf{Spezialfall:} Polynome vom Grad $n$ sind Potenzreihen, mit $a_k=0$ für $k>n$.
	Man kann sich eine Potenzreihe so vorstellen, dass eine Funktion (immer genauer) durch Polynome angenähert wird.
\end{itemize}
Wir wollen nun das Konvergenzverhalten von Potenzreihen untersuchen. Bemerkenswert ist, dass der Bereich in dem eine komplexe Potenzreihe konvergiert stets eine Kreisscheibe um den Entwicklungspunkt $z_o$ ist. Der Radius dieser Kreisscheibe heißt Konvergenzradius der Potenzreihe. Dieser Kann gleich $+\infty$ sein, was bedeutet, dass die Potenzreihe auf ganz $\C$ konvergiert. Die Potenzreihe divergiert für Werte außerhalb der Scheibe. Auf dem Kreis kann keine eindeutige Aussage gemacht werden, es ist beides möglich.

\begin{satz}{Aussagen zu Potenzreihen}
	Es sei $\sum\limits_{k=0}^\infty a_k(z-z_0)^k$ eine Potenzreihe, dann gelten die folgenden Aussagen:
	\begin{enumerate}
		\item Es gibt einen eindeutig bestimmten Konvergenzradius $R\in[0,\infty)\cup\simpleset\infty$ so dass gilt:

		Die Reihe $\sum\limits_{k=0}^\infty a_k(z-z_0)^k$ ist
		\begin{itemize}
			\item absolut konvergent für $|z-z_0|<R$
			\item divergent für $|z-z_0|>R$
		\end{itemize}
		\item Falls
		\begin{equation*}
			r=\lim\limits_{n\to\infty}\frac{|a_n|}{|a_{n+1}|} \text{ oder }r=\lim\limits_{n\to\infty}\frac{1}{\sqrt[n]{|a_n|}}
		\end{equation*}
		(d.h. insbesondere, dass dieser Grenzwert existiert), dann ist $r=R$.
		\item Falls $R>0$ und $0<\delta<R$, dann konvergiert die Funktionenfolge der Partialsummen mit $z\in\C$ als Variable gleichmäßig für alle $|z-z_0|<\delta$. Insbesondere ist die durch den Grenzwert der Potenzreihe definierte Funktion $K_R(z_0)\rightarrow \C, z\mapsto\sum_{k=0}^n a_k(z-z_0)^k$ stetig.
	\end{enumerate}
\end{satz}
