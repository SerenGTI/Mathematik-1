\chapter{Konvergenz in metrischen Räumen}
\section{Metrische Räume}
Um Konvergenz (beliebig genaue Approximation) beschreiben zu können, benötigen wir den Begriff des Abstands.
\begin{definition}{Metrik, metrischer Raum}
	Sei $X$ eine Menge. Eine Abbildung $\varrho:X\times X\rightarrow \R$ heißt Metrik (auch Abstandsfunktion), wenn sie für alle $x,y,z\in X$ folgende Eigenschaften hat.
	\begin{description}
		\item[M1] $\varrho(x,y)\geq 0$ und es gilt $\varrho(x,y)=0$ gdw. $x=y$
		\item[M2] $\varrho(x,y)=\varrho(y,x)$, d.h. $\varrho$ ist eine symmetrische Funktion
		\item[M3] $\varrho(x,z)\leq\varrho(x,y)+\varrho(y,z)$ (Dreiecksungleichung)
	\end{description}
	Eine Menge $X$ versehen mit einer Metrik nennen wir metrischen Raum.
\end{definition}


\paragraph{Beispiele:}
\begin{itemize}
	\item $X=\R$, $\varrho(x,y)=|x-y|$
	\item $X=\R^n$, in $\R^n$ ist der \emph{euklidische Abstand} gegeben durch
	\begin{align*}
		\varrho(x,y)&=\varrho\vector{x_1\\x_2\\\vdots\\x_n},\vector{y_1\\y_2\\\vdots\\y_n}\\
								&=\sqrt{(x_1-y_1)^2+(x_2-y_2)^2+\ldots+(x_n-y_n)^2}\\
								&=\sqrt{\sum\limits^n_{i=1}(x_i-y_i)^2}
	\end{align*}
	vgl. dem Satz von \textsc{Pythagoras}
	\item Auf jeder nichtleeren Menge kann man die \emph{diskrete Metrik} einführen:
	\begin{equation*}
		\mathrm{d}(x,y)=\begin{cases}
			0, \text{ falls }x=y\\
			1, \text{ falls }x\neq y
		\end{cases}
	\end{equation*}
	\item Ist $V$ ein \emph{euklidischer Vektorraum}, dann ist durch $|\!|v|\!|=\sqrt{<\!x,y\!>}$ eine \emph{Norm} gegeben,falls für jede Norm $|\!|\cdot|\!|$ liefert $\varrho(x,y)\coloneqq |\!|x-y|\!|$ eine Metrik. Mit anderen Worten, jeder normierte Vektorraum ist ein metrischer Raum.
	\item In der \emph{Codierungstheorie} führt man auf der Menge der n-stelligen Binärwörter
	\begin{equation*}
		X=\set{(x_1,x_2,\ldots,x_n)}{x_i\in\simpleset{0,1}}
	\end{equation*}
	den \emph{Hemmingabstand} ein:
	\begin{equation*}
		\varrho(x,y) =\text{ Anzahl von Stellen an denen sich $x$ und $y$ unterscheiden.}
	\end{equation*}
	Z.B. $\varrho((0,0,1,1),(0,0,1,0))=1$. Anwendung: Fehlerkorrigierende Codes.
\end{itemize}


Mit Hilfe der Metrik führen wir den Begriff der Kugelumgebung eines Punktes in einem metrischen Raum ein.
\begin{definition}{Kugelumgebung}
	Sei ein Punkt $x_0\in X$ und $\epsilon>0$ eine reelle Zahl. Unter der Kugelumgebung von $x_0$ mit Radius $\epsilon$ um den Mittelpunkt $x_0$ versteht man die Menge
	\begin{equation*}
		K_\epsilon(x_0)\coloneqq\set{x\in X}{\varrho(x,x_0)<\epsilon}
	\end{equation*}
\end{definition}
\paragraph{Beispiel:} Im $\R^2$ mit der euklidischen Metrik ist $K_\epsilon(x_0)$ die \emph{offene Kreisscheibe} (das Innere der Kreisscheibe) um $x_0$ mit Radius $\epsilon$. (Punkte auf dem Kreis sind nicht in $K_\epsilon$!)

\begin{definition}{Offene und abgeschlossene Mengen}
	Sei $X$ ein metrischer Raum. Eine Teilmenge $U\subseteq X$ heißt \emph{offen}, falls zu jedem $x_0\in U$ eine Kugelumgebung mit $\epsilon>0$ existiert, die ganz in $U$ enthalten ist.

	Eine Teilmenge $A\subseteq X$ heißt abgeschlossen, falls ihr Komplement $X\setminus A$ offen ist.
\end{definition}
\paragraph{Beispiel:}
Sei $X=\R$ und $\varrho(x,y)=|x-y|$.

Dann ist das Intervall
\begin{equation*}
	(a,b)=\set{x\in \R}{a<x<b}
\end{equation*}
im obigen Sinne offen.

Das Intervall
\begin{equation*}
	[a,b]=\set{x\in\R}{a\leq x\leq b}
\end{equation*}
ist abgeschlossen.

Das Intervall
\begin{equation*}
	(a,b]=\set{x\in \R}{a<x\leq b}
\end{equation*}
ist weder abgeschlossen noch offen.
