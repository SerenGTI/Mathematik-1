\chapter{Mehrdimensionale Differentiation}
Für viele Anwendungen in der Physik, Technik und auch für innermathematische Fragestellungen benötigt man Funktionen, die eventuell von mehreren Variablen oder Parametern abhängen. Wir wollen uns überlegen, wie man solche Funktionen mit Mitteln der Differentialrechnung studieren kann. Außerdem betrachten wir auch Abbildungen, deren Wertebereich in einem $\R^m$ liegt.  Wir studieren also Abbildungen
\begin{equation*}
	f:D\rightarrow \R^m, D\subseteq \R^n
\end{equation*}
Abbildungen, deren Wertebereich ein- oder zweidimensional ist werden eher als Funktionen bezeichnet, $f$ deren Wertebereich höherdimensional ist eher als Abbildungen obwohl beide Begriffe strenggenommen synonym sind.

Wir bezeichnen $n$ als die Anzahl der Variablen und $m$ als die Anzahl der Komponenten der Abbildung.

Jede Abbildung $f:D\rightarrow\R^m, D\subseteq \R^n$ ist von der Form
\begin{equation*}
	f(x)=f(x_1,\ldots,x_n)=\vector{f_0(x_1,\ldots,x_n)\\\vdots\\f_m(x_1,\ldots,x_n)}\in\R^m
\end{equation*}
Die Funktionen $f_i:\R^n\rightarrow\R^m$ heißen die Komponentenfunktionen von $f$. Man kann also eine Abbildung $f$ auch als ein $m$-Tupel von Funktionen $f_i:D\rightarrow\R$ betrachten. Deswegen ist es für manche Fragestellenungen ausreichend, den Fall $m=1$ zu betrachten.

\paragraph{Beispiele:}
\begin{itemize}
	\item Eine Funktion in zwei Variablen, sei $D\subseteq[-1,1]\times[-1,1]$ und $m=1$.
	\begin{equation*}
		f:D\rightarrow\R,(x,y)\mapsto2-x^2-y^2
	\end{equation*}
	Der Graph der Funktion
	\begin{equation*}
		\Gamma_f=\set{(x,y,z)\in [-1,1]\times[-1,1]\times\R}{z=f(x,y)}
	\end{equation*}
	ist eine Teilmenge des $\R^3$.

	\begin{center}
		\begin{tikzpicture}

			\coordinate (frontleft) at (-1,-6/4);
			\coordinate (frontmid) at (3,-6/4);
			\coordinate (frontright) at (7,-6/4);
			\coordinate (midleft) at (1,0);
			\coordinate (midmid) at (5,0);
			\coordinate (midright) at (9,0);
			\coordinate (backleft) at (3,6/4);
			\coordinate (backmid) at (7,6/4);
			\coordinate (backright) at (11,6/4);



			\filldraw[draw=green, line width=0.2mm, fill=green!20!white]
			(frontleft) -- (frontright) -- (backright) -- (backleft) -- cycle;
			\node[label={right:\color{green}$D$}] at (7,-6/4) {};

			\draw[->] (0,0) -- (10,0) node[right] {$x_1$};
			\draw[->] (8,9/4) -- (2,-2.25) node[below] {$x_2$};
			\draw[->] (5,0) -- (5,5.4) node[above] {$x_3$};
			%\makegrid

			\draw (midleft) node (a) [draw, rectangle,minimum size=0pt,inner sep = 0pt, minimum height=4pt, label={below:$-1$}] {};
			\draw (midright) node (a) [draw, rectangle,minimum size=0pt,inner sep = 0pt, minimum height=4pt, label={below:$1$}] {};
			\draw (frontmid) node (a) [draw, rectangle,minimum size=0pt,inner sep = 0pt, minimum width=4pt, label={below right:$1$}] {};
			\draw (backmid) node (a) [draw, rectangle,minimum size=0pt,inner sep = 0pt, minimum width=4pt, label={below right:$-1$}] {};
			\draw (5,2) node (a) [draw, rectangle,minimum size=0pt,inner sep = 0pt, minimum width=4pt, label={right:$1$}] {};
			\draw (5,4) node (a) [draw, rectangle,minimum size=0pt,inner sep = 0pt, minimum width=4pt, label={above right:$2$}] {};



			\coordinate (overfrontmid) at (3,-6/4+2);
			\coordinate (overmidleft) at (1,2);
			\coordinate (overbackmid) at (7,6/4+2);
			\coordinate (overmidright) at (9,2);

			\draw[dotted]
			(frontmid) to (overfrontmid)
			(midleft) to (overmidleft)
			(backmid) to (overbackmid)
			(midright) to (overmidright);


			\draw[red, line width=0.5mm]
			(frontleft) .. controls ($(overfrontmid)+(-0.5,0.7)$) and ($(overfrontmid)+(0.5,0.7)$) .. (frontright);
			\draw[red, line width=0.2mm]
			(backleft) .. controls ($(overbackmid)+(-0.5,0.7)$) and ($(overbackmid)+(0.5,0.7)$) ..  (backright);


			\draw[red, line width=0.5mm]
			(frontleft) .. controls (0.2,2.5) and (1.3,2.6) .. (backleft);
			\draw[red, line width=0.5mm]
			(frontright) .. controls (8.2,2.5) and (9.3,2.6) .. (backright);


			\draw[red, line width=0.2mm]
			($(overmidleft)+(-0.01,0)$) .. controls (4.5,4.7) and (5.5,4.7) .. (overmidright);

			\draw[red, line width=0.5mm]
			($(overmidleft)+(-0.08,-0.02)$) .. controls (4.5,5) and (7,4.7) .. (backright);

			% \filldraw[fill=blue, draw=red]
			% 	(frontleft) .. controls ($(overfrontmid)+(-0.5,0.7)$) and ($(overfrontmid)+(0.5,0.7)$) .. (frontright)
			% 	-- (frontright) .. controls (8.2,2.5) and (9.3,2.6) .. (backright)
			% 	-- (backright) .. controls (7,4.7) and (4.5,5) .. ($(overmidleft)+(-0.08,-0.02)$)
			% 	-- (frontleft) .. controls (0.2,2.5) and (1.3,2.6) .. (backleft)
			% 	-- (frontleft);

			\foreach \x/\y in {1/0.73,2/1.4,3/1.85,4/2,5/1.85,6/1.4,7/0.73}
				\draw[red, line width=0.1mm] ($(frontleft)+(\x,\y)$) .. controls ($(0.2,2.5)+(\x,\y)$) and ($(1.3,2.6)+(\x,\y)$) .. ($(backleft)+(\x,\y)$);

			\foreach \x/\y in {0.5/1.4,1/2.5,1.51/3.2,2.5/3.57}
				\draw[red, line width=0.1mm] ($(frontleft)+(\x,\y)$) .. controls ($(overfrontmid)+(-0.5+\x,0.7+\y)$) and ($(overfrontmid)+(0.5+\x,0.7+\y)$) .. ($(frontright)+(\x,\y)$);

		\end{tikzpicture}
	\end{center}

	\item Eine Abbildung von $D=[0,\pi]$ nach $\R^2$ (auch Kurve genannt),
	\begin{equation*}
		c(t)=\vector{\cos(t)\\\sin(t)}
	\end{equation*}
\end{itemize}
