\chapter{Mehrdimensionale Differentiation}
Für viele Anwendungen in der Physik, Technik und auch für innermathematische Fragestellungen benötigt man Funktionen, die eventuell von mehreren Variablen oder Parametern abhängen. Wir wollen uns überlegen, wie man solche Funktionen mit Mitteln der Differentialrechnung studieren kann. Außerdem betrachten wir auch Abbildungen, deren Wertebereich in einem $\R^m$ liegt.  Wir studieren also Abbildungen
\begin{equation*}
	f:D\rightarrow \R^m, D\subseteq \R^n
\end{equation*}
Abbildungen, deren Wertebereich ein- oder zweidimensional ist werden eher als Funktionen bezeichnet, $f$ deren Wertebereich höherdimensional ist eher als Abbildungen obwohl beide Begriffe strenggenommen synonym sind.

Wir bezeichnen $n$ als die Anzahl der Variablen und $m$ als die Anzahl der Komponenten der Abbildung.

Jede Abbildung $f:D\rightarrow\R^m, D\subseteq \R^n$ ist von der Form
\begin{equation*}
	f(x)=f(x_1,\ldots,x_n)=\vector{f_0(x_1,\ldots,x_n)\\\vdots\\f_m(x_1,\ldots,x_n)}\in\R^m
\end{equation*}
Die Funktionen $f_i:\R^n\rightarrow\R^m$ heißen die Komponentenfunktionen von $f$. Man kann also eine Abbildung $f$ auch als ein $m$-Tupel von Funktionen $f_i:D\rightarrow\R$ betrachten. Deswegen ist es für manche Fragestellenungen ausreichend, den Fall $m=1$ zu betrachten.

\paragraph{Beispiele:}
\begin{itemize}
	\item Eine Funktion in zwei Variablen, sei $D\subseteq[-1,1]\times[-1,1]$ und $m=1$.
	\begin{equation*}
		f:D\rightarrow\R,(x,y)\mapsto2-x^2-y^2
	\end{equation*}
	Der Graph der Funktion
	\begin{equation*}
		\Gamma_f=\set{(x,y,z)\in [-1,1]\times[-1,1]\times\R}{z=f(x,y)}
	\end{equation*}
	ist eine Teilmenge des $\R^3$.

	\begin{center}
		\begin{tikzpicture}

			\coordinate (frontleft) at (-1,-6/4);
			\coordinate (frontmid) at (3,-6/4);
			\coordinate (frontright) at (7,-6/4);
			\coordinate (midleft) at (1,0);
			\coordinate (midmid) at (5,0);
			\coordinate (midright) at (9,0);
			\coordinate (backleft) at (3,6/4);
			\coordinate (backmid) at (7,6/4);
			\coordinate (backright) at (11,6/4);



			\filldraw[draw=green, line width=0.2mm, fill=green!20!white]
			(frontleft) -- (frontright) -- (backright) -- (backleft) -- cycle;
			\node[label={right:\color{green}$D$}] at (7,-6/4) {};

			\draw[->] (0,0) -- (10,0) node[right] {$x_1$};
			\draw[->] (8,9/4) -- (2,-2.25) node[below] {$x_2$};
			\draw[->] (5,0) -- (5,5.4) node[above] {$x_3$};
			%\makegrid

			\draw (midleft) node (a) [draw, rectangle,minimum size=0pt,inner sep = 0pt, minimum height=4pt, label={below:$-1$}] {};
			\draw (midright) node (a) [draw, rectangle,minimum size=0pt,inner sep = 0pt, minimum height=4pt, label={below:$1$}] {};
			\draw (frontmid) node (a) [draw, rectangle,minimum size=0pt,inner sep = 0pt, minimum width=4pt, label={below right:$1$}] {};
			\draw (backmid) node (a) [draw, rectangle,minimum size=0pt,inner sep = 0pt, minimum width=4pt, label={below right:$-1$}] {};
			\draw (5,2) node (a) [draw, rectangle,minimum size=0pt,inner sep = 0pt, minimum width=4pt, label={right:$1$}] {};
			\draw (5,4) node (a) [draw, rectangle,minimum size=0pt,inner sep = 0pt, minimum width=4pt, label={above right:$2$}] {};



			\coordinate (overfrontmid) at (3,-6/4+2);
			\coordinate (overmidleft) at (1,2);
			\coordinate (overbackmid) at (7,6/4+2);
			\coordinate (overmidright) at (9,2);

			\draw[dotted]
			(frontmid) to (overfrontmid)
			(midleft) to (overmidleft)
			(backmid) to (overbackmid)
			(midright) to (overmidright);


			\draw[red, line width=0.5mm]
			(frontleft) .. controls ($(overfrontmid)+(-0.5,0.7)$) and ($(overfrontmid)+(0.5,0.7)$) .. (frontright);
			\draw[red, line width=0.2mm]
			(backleft) .. controls ($(overbackmid)+(-0.5,0.7)$) and ($(overbackmid)+(0.5,0.7)$) ..  (backright);


			\draw[red, line width=0.5mm]
			(frontleft) .. controls (0.2,2.5) and (1.3,2.6) .. (backleft);
			\draw[red, line width=0.5mm]
			(frontright) .. controls (8.2,2.5) and (9.3,2.6) .. (backright);


			\draw[red, line width=0.2mm]
			($(overmidleft)+(-0.01,0)$) .. controls (4.5,4.7) and (5.5,4.7) .. (overmidright);

			\draw[red, line width=0.5mm]
			($(overmidleft)+(-0.08,-0.02)$) .. controls (4.5,5) and (7,4.7) .. (backright);

			\foreach \x/\y in {1/0.73,2/1.4,3/1.85,4/2,5/1.85,6/1.4,7/0.73}
				\draw[red, line width=0.1mm] ($(frontleft)+(\x,\y)$) .. controls ($(0.2,2.5)+(\x,\y)$) and ($(1.3,2.6)+(\x,\y)$) .. ($(backleft)+(\x,\y)$);

			\foreach \x/\y in {0.5/1.4,1/2.5,1.51/3.2,2.5/3.57}
				\draw[red, line width=0.1mm] ($(frontleft)+(\x,\y)$) .. controls ($(overfrontmid)+(-0.5+\x,0.7+\y)$) and ($(overfrontmid)+(0.5+\x,0.7+\y)$) .. ($(frontright)+(\x,\y)$);

		\end{tikzpicture}
	\end{center}

	\item Eine Abbildung von $D=[0,\pi]$ nach $\R^2$ (auch Kurve genannt),
	\begin{equation*}
		c(t)=\vector{\cos(t)\\\sin(t)}
	\end{equation*}

	Die Spur (das Bild) der Kurve ist der Halbkreis mit Radius $1$ um $(0,0)$ oberhalb der $x$-Achse. Der Graph ist eine Schraubenlinie.
	\begin{center}
		\begin{tikzpicture}
			\draw[dotted] (4,0) to (4.5,0);
			\draw[black] (0,0) to (4,0);

			\draw (0,0) node[minimum size=0pt, minimum height=4pt, inner sep=0pt, draw=black, label={below:$0$}] {};
			\draw (4,0) node[minimum size=0pt, minimum height=4pt, inner sep=0pt, draw=black, label={below:$\pi$}] {};

			\draw[black, ->] (6,0) to (10,0);
			\draw[black, ->] (8,0) to (8,2);

			\draw[black, thick, ->, bend left] (5,0.75) to (5.5,0.75);

			\draw (6.5,0) node[draw, minimum size=0pt, inner sep=0pt, minimum height=4pt, label={below:{$-1$}}] {};
			\draw (9.5,0) node[draw, minimum size=0pt, inner sep=0pt, minimum height=4pt, label={below:{$1$}}] {};

			\draw (8,1.5) node[draw, minimum size=0pt, inner sep=0pt, minimum width=4pt, label={above right:{$1$}}] {};

			\draw[line width=0.5mm, scale=1, domain=6.5:9.5, smooth, variable=\x, red] plot ({\x},{sqrt(2.25-(\x-8)*(\x-8))}) node[below right] {};
		\end{tikzpicture}
	\end{center}

	\item Die sogenannte Rotationsmatrix ist eine Abbildung $f:\R^2\rightarrow\R^2$ für einen Winkel $\varphi\in\R$:
	\begin{equation*}
		f\vector{x\\y}=
		\matrix{\cos\varphi & -\sin\varphi\\\sin\varphi & \cos\varphi} \vector{x\\y}=
		\vector{\cos\varphi*x-\sin\varphi*y\\\sin\varphi*x+\cos\varphi*y}
	\end{equation*}
	Diese Abbildung bewirkt eine Drehung um den Winkel $\varphi$ um den Ursprung.
\end{itemize}

Wie kann man solche Abbildungen $f:D\rightarrow\R^m,D\subseteq\R^n$ differenzieren?
Wir beschränken uns auf den Fall $m=1$, da man mehrere Komponenten durch mehrere Komponentenfunktionen darstellen kann.
Was nicht funktioniert, ist für Funktionen mehrerer Variablen einen Differentialquotienten zu schreiben, denn dieser ist i.A. nicht definiert.

Die richtige Idee zu mehrdimensionaler Differentiation ist, die Funktion durch eine affin-lineare Abbildung zu ersetzen. Gesucht ist das totale Differential.

\section{Richtungsableitung}
\begin{definition}{Partielle Ableitung}
	Sei $f:D\rightarrow\R^m,D\subseteq\R^n$ und $x=(x_1,\ldots,x_n)\in D$ ein fester Punkt im Definitionsbereich. Die Funktion $f$ heißt im Punkt $x$ partiell nach der $k$ten Variable differenzierbar, falls der Grenzwert
	\begin{align*}
		\frac{\partial f}{\partial x_k} (x_1,\ldots,x_n)
		&\coloneqq\lim\limits_{h\to 0}\frac{f(x_1,\ldots,x_{k-1},x_k+h,x_{k+1},\ldots,x_n)-f(x_1,\ldots,x_n)}{h}\\
		&\coloneqq f_{x_k}(x_1,\ldots,x_n) = f_{x_k}(x)
	\end{align*}
	existiert. In diesem Fall heißt der obige Term die \emph{partielle Ableitung} von $f$ nach der $k$ten Variable $x_k$ im Punkt $x$.
\end{definition}
Die partielle Ableitung ist also die Steigung eines eindimensionalen Schnitts durch den Graphen. Mit anderen Worten ist die partielle Ableitung die gewöhnliche, eindimensionale Ableitung der Funktion.

\begin{definition}{Richtungsableitung}
	Sei $a\in\R^n$ ein Vektor und die Funktion $f:D\rightarrow\R,D\subseteq\R^n$ mit $a\in D$. Die Richtungsableitung von $f$ bei $x\in D$ in Richtung $a$ ist der Grenzwert
	\begin{align*}
		\frac{\partial f}{\partial a}(x)
		&\coloneqq \lim\limits_{h\to 0}\frac{f(x+h*a)-f(x)}{h}\\
		&=\frac{\diff}{\diff t}f(x+t*a)\Big|_{t=0}\\
	\end{align*}
\end{definition}
Es gilt außerdem, dass die partielle Ableitung ein Sonderfall der Richtungsableitung ist, das heißt
\begin{equation*}
	\frac{\partial f}{\partial e_k}(x)=\frac{\partial f}{\partial x_k}(x)
\end{equation*}

Die Begriffe der partiellen Ableitung und der Richtungsableitung lassen sich auch auf Abbildungen $f:D\rightarrow\R^m,D\subseteq \R^n$ verallgemeinern, in dem man die Komponentenfunktionen entsprechend ableitet.
\begin{equation*}
	\frac{\partial f}{\partial x_k}(x)\coloneqq
	\vector{
		\frac{\partial f_1}{\partial x_k}(x)\\
		\vdots\\
		\frac{\partial f_m}{\partial x_k}(x)
	}
\end{equation*}

\section{Totales Differential}
Wir wollen nun einen sinnvollen Begriff von Differenzierbarkeit für Funktinen oder Abbildungen die von mehreren Variablen abhängen definieren. Wir werden sehen, dass es nicht genügt, dass die partiellen Ableitungen existieren.

Was sich verallgemeinern lässt ist die Idee, eine Funktion durch eine affin-lineare Abbildung anzunähern, nämlich durch ihr erstes Taylorpolynom.

\begin{definition}{Totale Differenzierbarkeit}
	Eine Abbildung $f:E\rightarrow\R^m,E\subseteq \R^n$ heißt total differenzierbar in $x\in D$, falls eine Matrix $f'(x)=Df(x)\in M(m,n,\R)$ existiert, so dass
	\begin{equation*}
		f(x+h)=f(x)+f'(x)*h+o(h)
	\end{equation*}
	für alle $h\in\R^n, (x+h)\in D$ gilt. Hierbei muss die Restfunktion $o(h)\in\R^m$ die Bedingungen
	\begin{equation*}
		o(h)=r(x,h)*h\text{ mit }\lim\limits_{h\to 0} r(x,h)=0
	\end{equation*}
	erfüllen. Falls ein solches $f'(x)=Df(x)$ existiert, heißt es auch \emph{das totale Differential} von $f$ bei $x$.
\end{definition}

\paragraph{Beispiele:}
\begin{itemize}
	\item Es sei $\alpha=(\alpha_1,\ldots,\alpha_n)\in\R^n$ und sei $f:\R^n\rightarrow\R$ definiert durch
	\begin{equation*}
		f(x)=\alpha x+b = (\alpha_1,\ldots,\alpha_n)*\vector{x_1\\\vdots\\x_n}+b
	\end{equation*}
	Dies ist eine affin-lineare Abbildung. Dann ist $f'(x)=\alpha$ und zwar für alle $x\in\R^n$ gleichzeitig.
	Es gilt
	\begin{equation*}
		f(x+h)=\alpha (x+h)+b=\alpha x+\alpha h+b=f(x)+\alpha*h
	\end{equation*}
	das heißt wir lesen ab, dass $\alpha\in M(1,n,\R)$ das totale Differential $Df(x)$ für alle $x$ ist. Eine affin-lineare Abbildung hat konstante Ableitung.

	\item Funktionert auch mit $f:\R^n\rightarrow\R^m, f(x)=Ax+b,A\in M(m,n,\R), b\in\R^m$ hierbei ist $Df(x)=A$ konstant.

	\item $f:\R^n\rightarrow\R, f(x)=x^T*x$ das heißt
	\begin{align*}
		f(x)&=(x_1,\ldots,x_n)\vector{x_1\\\vdots\\x_n}\\
		&=x_1^2+\ldots+x_n^2
		\intertext{wir erhalten}
		f(x+h)&=(x+h)^T*(x+h)\\
		&=x^Tx+x^Th+h^Tx+h^Th\\
		&=f(x)+2x^Th+h^Th
	\end{align*}
	also ist $2x^T\in M(1,n,\R)$ das totale Differential von $f$ bei $x$.
\end{itemize}
