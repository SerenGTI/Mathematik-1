\section{Satz von Taylor}
Wir sehen uns noch einmal die Formel aus dem Mittelwertsatz der Differentialrechnung an.
\begin{equation*}
	f(b)-f(a)=(b-a)*f'(\xi)
\end{equation*}
Wir setzen jetzt $b=x$ und $a=x_0$ und erhalten damit:
\begin{equation*}
	f(x)=f(x_0)+(x-x_0)*f'(\xi)
\end{equation*}
Um eine Approximation von $f$ in der Nähe von $x_0$ zu erhalten, ersetzen wir in der obigen Formel $\xi$ durch $x_0$. Wir erhalten dann:
\begin{equation*}
	f(x)\approx f(x_0)+(x-x_0)*f'(x_0)
\end{equation*}
Wir ersetzen also näherungsweise die Funktion $f$ durch das Lineare Polynom, dessen Graph die Tangente an den Graph von $f$ bei $x_0$ ist.

Diese Funktion $T_1(x)=f(x_0)+(x-x_0)*f'(x_0)$ wird als das \emph{Taylorpolynom} ersten Grades von $f$ bei $x_0$ bezeichnet. Die Funktion $T_1$ ist dadurch gekennzeichnet, dass ihre erste und nullte Ableitung mit der ersten und nullten Ableitung von $f$ bei $x_0$ übereinstimmen. Das heißt:
\begin{align*}
	T_1(x_0)&=f(x_0)\\
	T_1'(x_0)&=f'(x_=)
\end{align*}
Wir führen nun das $n$te Taylorpolynom ein, indem wir fordern, dass es am Punkt $x_0$ mit der Funktion $f$ in der nullten bis zur $n$ten Ableitung übereinstimmt.

Nun stellt sich die Frage nach der Güte der Approximation.

\begin{lemma}{Taylorpolynom}
	Sei $f$ $n$-mal differenzierbar im Intervall $(a,b)$ und $x_0\in(a,b)$. Dann gibt es genau ein Polynom $n$ten Grades $T_n$, so dass gilt
	\begin{equation}\label{eq:taylorpolynom}
		T_n^{(k)}(x_0)=f^{(k)}(x_0) \quad\forall k\in[0,n]\in\N
	\end{equation}
	Dieses Polynom wird Taylorpolynom $n$ten Grades von $f$ um den Entwicklungspunkt $x_0$ genannt.
	Dabei ist
	\begin{align*}
		T_n(x_0)&=\sum\limits_{j=0}^n \frac{f^{(j)}(x_0)}{j!}(x-x_0)^j\\
		&=\underbrace{f(x_0)+f'(x_0)*(x-x_0)}_{=T_1}+\ldots+\frac{f^{(k)}(x_0)}{n!}(x-x_0)^n
	\end{align*}
\end{lemma}
\beweis
Wir zeigen zunächst die Eindeutigkeit. Seien $P,Q$ Polynome, die die Eigenschaften aus \autoref{eq:taylorpolynom} besitzen.
Für die Differenz $D=P-Q$ gilt:
\begin{equation*}
	D(x)=P(x)-Q(x)=b_0+b_1*x+\ldots+b_n*x^n
\end{equation*}
Wobei $D^{(k)}(x_0)=0$ für alle $0\leq k\leq n$. Es folgt also
\begin{equation*}
	D^{(n)}(x_0)=n!*b_n=0\enspace \rightsquigarrow b_n=0
\end{equation*}
Im nächsten Schritt sehen wir
\begin{equation*}
	D^{(n-1)}(x_0)=(n-1)!*b_{n-1}=0\enspace \rightsquigarrow b_{n-1}=0
\end{equation*}
und so weiter.
Am Ende sehen wir, dass alle $b_k=0$ sind, dies zeigt die Eindeutigkeit.

Man rechnet leicht nach, dass $T_n^{(k)}(x_0)=f^{(k)}(x_0)$ gilt.\hfill$\Box$

\paragraph{Bemerkung:}
Die Potenzreihe
\begin{equation*}
	\sum\limits_{n=0}^\infty \frac{f^{(n)}(x_0)}{n!}(x-x_0)^n
\end{equation*}
für eine beliebig oft differenzierbare Funktion $f$ wird auch die Taylorreihe von $f$ um $x_0$ genannt.

\paragraph{Beispiele:}
\begin{itemize}
	\item Sei $f$ die Exponentialfunktion $f=e^x$, dann gilt $f^{(k)}(x_0)=e^x$ für alle $k$. Damit gilt für das $n$te Taylorpolynom von $e^x$ bei $x_0=0$: $f^{(k)}(0)=e^0=1$.
	\begin{equation*}
		T_n(x)=1+\frac{1}{1!}x+\frac{1}{2!}x^2+\frac{1}{3!}x^3+\ldots +\frac{1}{n!}x^n
	\end{equation*}
	Das gilt also, dass die Taylorreihe von $e^x$ bei $x_0=0$ mit der Exponentialreihe übereinstimmt.
	\item Sei $f(x)=\sin(x)$ und $x_0=0$. Es gilt:
	\begin{align*}
		f^{(0)}(0)&=\sin(0)=0 & f^{(1)}(0)&=\cos(0)=1\\
		f^{(2)}(0)&=-\sin(0)=0 & f^{(3)}(0)&=-\cos(0)=-1
	\end{align*}
	Also folgt $f^{(2k)}(0)=0$ und $f^{(2k+1)}(0)=(-1)^k$.
	Damit erhalten wir für das Taylorpolynom:
	\begin{equation*}
		T_{2n+1}=\underbrace{x-\frac{1}{3!}x^3+\frac{1}{5!}x^5-\frac{1}{7!}x^7}_{T_3(x)}\pm\ldots+\frac{1}{(2n+1)!}x^{2n+1}
	\end{equation*}
	Hieraus folgt, dass die Taylorreihe von $\sin(x)$ bei $x_0=0$ genau die Potenzreihe von $x\mapsto\sin(x)$ ist, mit der wir den Sinus definiert haben.

	Bereits das Taylorpolynom dritten Grades nähert den Sinus im Intervall $[-\pi,\pi]$ gut an:
	\begin{center}
		\begin{easyfunction}{-5}{5}{-1}{2.5}{1}
			\draw[->] (-5.2,0) -- (5.2,0) node[right] {$x$};
			\draw[->] (0,-2) -- (0,2) node[above] {$f(x)$};
			%\makegrid
			\draw (3.1415,0) node (a) [fill = white,rectangle,inner sep = 0pt,minimum size = 0pt,minimum height=4pt,draw, label={below:$\pi$}] {};
			\draw (1.5707,0) node (a) [fill = white,rectangle,inner sep = 0pt,minimum size = 0pt,minimum height=4pt,draw, label={below:$\frac\pi2$}] {};
			\draw (-3.1415,0) node (b) [fill = white,rectangle,inner sep = 0pt,minimum size = 0pt,minimum height=4pt,draw, label={below:$-\pi$}] {};
			\draw (-1.5707,0) node (a) [fill = white,rectangle,inner sep = 0pt,minimum size = 0pt,minimum height=4pt,draw, label={below:$-\frac\pi2$}] {};

			\begin{scope}
				\draw[line width=0.5mm,scale=1,domain=-5:5,smooth,variable=\x,blue] plot ({\x},{sin(deg(\x))})
					node[right] {$\sin(x)$};
				\draw[line width=0.3mm,scale=1,domain=-3.9:3.9,smooth,variable=\x,red] plot ({\x},{\x-1/6*\x*\x*\x+1/120*\x*\x*\x*\x*\x-1/5040*\x*\x*\x*\x*\x*\x*\x})
					node[below] {$T_3(x)$};
			\end{scope}
		\end{easyfunction}
	\end{center}
\end{itemize}

Um die Güte der Approximation durch Taylorpolynome abschätzen zu können, benutzen wir den folgenden Satz:
\begin{satz}{Satz von Taylor}
Sei $f$ eine $n+1$ mal stetig differenzierbare Funktion auf $(a,b)$ und seien $x,x_0\in(a,b)$. Dann existiert ein $\xi\in(x,x_0)$ beziehungsweise ein $\xi\in(x_0,x)$, so dass gilt:
\begin{equation*}
	f(x)=T_n(x)+\underbrace{\frac{f^{(n+1)}(\xi)}{(n+1!)}(x-x_0)^{n+1}}_{\text{LaGrange'sches Restglied}}
\end{equation*}
\end{satz}
\beweis
(ausstehend.)



\paragraph{Anwedungsbeispiele:}
Sei $f(x)=e^x$. Dann ist in $[0,1]$ für ein $\xi\in(0,1)$
\begin{equation*}
	e^x=f(x)=T_n(x)+\frac{f^{(n+1)}(\xi)}{(n+1!)}x^{n+1}
\end{equation*}
Und es gilt also
\begin{equation*}
	|f(x)-T_n(x)|\leq \frac{e^1}{(n+1)!}<\frac{3}{(n+1)!}
\end{equation*}
Um $e^x$ auf $[0,1]$ mit einem Fehler von höchstens $10^{-5}$ berechnen zu können, berechnen wir
\begin{equation*}
	(n+1)!\geq 30000 \enspace\rightsquigarrow8!=40320
\end{equation*}
Man sieht also für $n=7$ ist der Fehler auf diesem Intervall kleiner als $10^{-5}$.
