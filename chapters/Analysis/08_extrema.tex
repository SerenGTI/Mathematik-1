\chapter{Extrema und Mittelwertsätze}
Wir definieren zunächst lokale und globale Extrema.

\begin{definition}{Lokale Extrema}
	Sei $D\subseteq\R$ eine offene Teilmenge.
	Die Funktion $f:D\rightarrow\R$ hat bei $x_0\in D$ ein lokales Minimum genau dann, wenn gilt:
	\begin{equation*}
		\exists \epsilon>0 \enspace\forall x\in K_\epsilon(x_0):f(x)\geq f(x_0)
	\end{equation*}
	Die Funktion hat bei $x_0$ ein lokales Minimum, falls gilt:
	\begin{equation*}
		\exists \epsilon>0 \enspace\forall x\in K_\epsilon(x_0):f(x)\leq f(x_0)
	\end{equation*}
	Das lokale Extremum heißt strikt oder isoliert, falls in der Definition die strikte Ungleichung gilt.
\end{definition}
\begin{definition}{Globale Extrema}
	Falls für alle $x\in D$ gilt $f(x)\geq f(x_0)$ bzw. $f(x)\leq f(x_0)$, liegt ein globales Minimum bzw. Maximum bei $x_0$ vor.
\end{definition}

Der folgende Satz gibt den Zusammenhang zwischen lokalen Extrema und Differentiation an:
\begin{satz}{}
	Sei $D\subseteq\R$ eine offene Teilmenge und sei die Funktion $f:D\rightarrow\R$ bei $x_0\in D$ differenzierbar. Hat $f$ bei $x_0$ ein lokales Extremum, dann ist $\frac{\diff f}{\diff x}(x_0)=0$.
\end{satz}
\begin{beweis}
	Sei $\epsilon$ wie in der Definition. Dann gilt für $0<h\leq\epsilon$, dass
	\begin{align*}
		&\text{bei einem Maximum:} &&\text{bei einem Minimum}\\
		&0\leq\frac{f(x_0-h)-f(x_0)}{-h} &&0\geq\frac{f(x_0-h)-f(x_0)}{-h}\\
		&0\geq\frac{f(x_0+h)-f(x_0)}{h} &&0\leq\frac{f(x_0+h)-f(x_0)}{h}
	\end{align*}
	Da $f$ bei $x_0$ differenzierbar ist, sind links- und rechtsseitiger Grenzwert gleich, dieser muss damit $0$ sein.
	\hfill$\Box$
\end{beweis}

\begin{satz}{Satz von Rolle}
Sei $f$ stetig in $[a,b]$ und differenzierbar in $(a,b)$. Und es gelte $f(a)=f(b)$. Dann gibt es mindestens eine Stelle $\xi \in(a,b)$ mit $f'(\xi)=0$.
\end{satz}

\begin{multicols}{2}
\beweis
Stetige Funktionen auf einem abgeschlossenen Intervall $[a,b]$ nehmen dort ihr Maximum und Minimum an. D.h. es existieren ein $\nu,\xi\in[a,b]$ mit $f(\nu)=\min f([a,b])$ und $f(\xi)=\max f([a,b])$.
Wir unterscheiden die folgenden Fälle:
\begin{enumerate}
	\item $f(\nu)=f(xi)$ dann ist $f$ auf $[a,b]$ konstant, damit gilt $f'(x)=0\enspace\forall x\in[a,b]$
	\item $f(\nu)<f(\xi)$ und $f(a)=f(b)<f(\xi)$, dann gilt $\xi\in(a,b)$ und $f'(\xi)=0$
	\item $f(\nu)<f(\xi)$ und $f(a)=f(b)>f(\xi)$, dann gilt $\nu\in(a,b)$ und $f'(\nu)=0$
\end{enumerate}
\hfill$\Box$

	\columnbreak
	\begin{center}
		\begin{easyfunction}{0}{5}{0}{4}{1}
			\draw[->] (0,0) -- (5.2,0) node[right] {$x$};
			\draw[->] (0,0) -- (0,4.2) node[above] {$f(x)$};
			\makegrid

			\begin{scope}
				\clip(0,0) rectangle (5,4);

				\draw[line width=0.5mm,scale=1,domain=1:4.36,smooth,variable=\x,blue] plot ({\x},{0.7*(\x-2.55)*(\x-2.55)*(\x-2.55)-2*(\x-2.55)+2.5})
					node[below right] {};
				\draw[line width=0.5mm,scale=1,domain=2.5:4.5,smooth,variable=\x,red] plot ({\x},{1.2})
					node[below right] {};
			\end{scope}
			\draw (1,3) node (fa) [fill = white,circle,inner sep = 0pt,minimum size = 4pt,draw, label={left:$f(a)$}] {};
			\draw (1,0) node (a) [fill = white,rectangle,inner sep = 0pt,minimum size = 0pt,minimum height=4pt,draw, label={below:$a$}] {};

			\draw (4.36,0) node (b) [fill = white,rectangle,inner sep = 0pt,minimum size = 0pt,minimum height=4pt,draw, label={below:$b$}] {};
			\draw (4.36,3) node (fb) [fill = white,circle,inner sep = 0pt,minimum size = 4pt,draw, label={right:$f(b)$}] {};

			\draw (3.5,0) node (xi) [fill = white,rectangle,inner sep = 0pt,minimum size = 0pt,minimum height=4pt,draw, label={below:$\xi$}] {};
			\draw (3.5,1.2) node (fxi) [fill = white,circle,inner sep = 0pt,minimum size = 4pt,draw] {};

			\draw[dashed, blue]
			(a) -- (fa) (b) -- (fb);
			\draw[dashed, red]
			(xi) -- (fxi) (fa) -- (fb);
		\end{easyfunction}
	\end{center}
\end{multicols}

\begin{satz}{Mittelwertsatz der Differentialrechnung}
	Sei $f$ stetig in $[a,b]$ und differenzierbar auf $(a,b)$. Dann gibt es mindestens eine Stelle $\xi\in (a,b)$ für die gilt:
	\begin{align*}
		f(b)-f(a)&=(b-a)*f*(\xi)\\
		\frac{f(b)-f(a)}{b-a}=f'(\xi)
	\end{align*}
	Die Steigung der Sekante durch $a,b$ wird durch eine Tangente an $f$ in der Stelle $\xi$ im Inneren des Intervalls angenommen.
\end{satz}
\begin{multicols}{2}
	Man wende den Satz von Rolle auf die Funktion
	\begin{equation*}
		F(x)=f(x)-\frac{f(b)-f(a)}{b-a}*(x-a)
	\end{equation*}
	an. Die Funktion $F$ genügt den Voraussetzungen des Satzes von Rolle und außerdem gilt
	\begin{align*}
		F(a)&=f(a)-0\\
		F(b)&=f(a)-\frac{f(b)-f(a)}{b-a}*(b-a)=f(a)
	\end{align*}
	Damit existiert ein $\xi\in(a,b)$ sodass
	\begin{align*}
		F'(\xi)&=0\\
					&=f'(\xi)-\frac{f(b)-f(a)}{b-a}*1\\
					f'(\xi)&=\frac{f(b)-f(a)}{b-a}
	\end{align*}
	\hfill$\Box$

	\columnbreak
	\begin{center}
		\begin{easyfunction}{0}{5}{0}{4}{1}
			\draw[->] (0,0) -- (5.2,0) node[right] {$x$};
			\draw[->] (0,0) -- (0,4.2) node[above] {$f(x)$};
			\makegrid

			\begin{scope}
				\clip(0,0) rectangle (5,4);

				%0.6(x-2)^3-(x-2)+1.5
				%1.8(x-2)^2-1 0.8
				\draw[line width=0.5mm,scale=1,domain=1:6,smooth,variable=\x,blue] plot ({\x},{0.6*(\x-2.5)*(\x-2.5)*(\x-2.5)-(\x-2.5)+1.5})
					node[below right] {};
				\draw[line width=0.5mm,scale=1,domain=0.5:2.5,smooth,variable=\x,red] plot ({\x},{0.8*(\x-1.5)+1.9})
					node[below right] {};
			\end{scope}
			\draw (1,1) node (fa) [fill = white,circle,inner sep = 0pt,minimum size = 4pt,draw, label={below left:$f(a)$}] {};
			\draw (1,0) node (a) [fill = white,rectangle,inner sep = 0pt,minimum size = 0pt,minimum height=4pt,draw, label={below:$a$}] {};

			\draw (4.4,0) node (b) [fill = white,rectangle,inner sep = 0pt,minimum size = 0pt,minimum height=4pt,draw, label={below:$b$}] {};
			\draw (4.4,3.72) node (fb) [fill = white,circle,inner sep = 0pt,minimum size = 4pt,draw, label={right:$f(b)$}] {};

			\draw (1.5,0) node (xi) [fill = white,rectangle,inner sep = 0pt,minimum size = 0pt,minimum height=4pt,draw, label={below:$\xi$}] {};
			\draw (1.5,1.9) node (fxi) [fill = white,circle,inner sep = 0pt,minimum size = 4pt,draw] {};

			\draw[dashed, blue]
			(a) -- (fa) (b) -- (fb);
			\draw[dashed, red]
			(xi) -- (fxi) (fa) -- (fb);
		\end{easyfunction}
	\end{center}
\end{multicols}

\begin{satz}{Verallgemeinerter Mittelwertsatz der Differentialrechnung}
	Seien $f,g$ stetig in $[a,b]$ und differenzierbar in $(a,b)$. Außerdem gelte, dass $g'(x)\neq 0$ für alle $x\in(a,b)$ und $g(a)\neq g(b)$. Dann existiert eine Stelle $\xi\in(a,b)$ mit
	\begin{equation*}
		\frac{f(b)-f(a)}{g(b)-g(a)}=\frac{f'(\xi)}{g'(\xi)}
	\end{equation*}
\end{satz}
\beweis
Man wende den Satz von Rolle auf die Funktion
\begin{equation*}
	F(x)=f(x)-f(a)-\frac{f(b)-f(a)}{b-a}*(g(x)-g(a))
\end{equation*}
an.\hfill$\Box$

\begin{satz}{Regel von L'Hospital}
	Seien $f,g$ in $(a,b]$ differenzierbar und gelte
	\begin{equation*}
		\lim\limits_{x\to a^+}f(x)=\lim\limits_{x\to a^+}g(x)=0
	\end{equation*}
	Weiterhin sei $g'(x)\neq 0$ für $x\in(a,b]$. Existiert der Grenzwert
	\begin{equation*}
		\lim\limits_{x\to a^+}\frac{f'(x)}{g'(x)}=c
	\end{equation*}
	dann ist
	\begin{equation*}
		\lim\limits_{x\to a^+}\frac{f(x)}{g(x)}=\lim\limits_{x\to a^+}\frac{f'(x)}{g'(x)}=c
	\end{equation*}
\end{satz}
\beweis
$f$ und $g$ lassen sich durch die Definition $f(a)=g(a)=0$ stetig auf ganz $[a,b]$ fortsetzen und genügen den Voraussetzungen des verallgemeinerten Mittelwertsatzes der Differentialrechnung. Da $f(a)=g(a)=0$ gilt, gilt für ein $x\in(a,b]$, dass ein $\xi(x)\in(a,x)$ existiert, so dass gilt
\begin{equation*}
	\frac{f(x)-f(a)}{g(x)-g(a)}=\frac{f(x)}{g(x)}=\frac{f'(x)}{g'(x)}
\end{equation*}
Wir erhalten für den rechtsseitigen Grenzwert
\begin{equation*}
	\lim\limits_{x\to a^+}\frac{f(x)}{g(x)}=\lim\limits_{x\to a^+}\frac{f'(x)}{g'(x)}
\end{equation*}
