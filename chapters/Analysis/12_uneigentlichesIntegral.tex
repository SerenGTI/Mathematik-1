\subsection{Uneigentliches Integral}
Uneigentlich bedeutet, dass eine oder beide Integrationsgrenzen $\pm\infty$ sind.
\begin{definition}{Uneigentliches Integral}
	Die Funktion $f$ sei auf $[a,x_0), a<x_0\leq\infty$ definiert und Regelfunktion auf jedem Teilintervall $[a,\nu]\subseteq [a,x_0)$. Dann heißt $f$ uneigentlich integrierbar auf $[a,x_0)$, falls der linksseitige Grenzwert
	\begin{equation*}
		\int_a^{x_0} f(x)\intd x\coloneqq \lim\limits_{\nu\to x_0^-}\int_a^\nu f(x)\intd x
	\end{equation*}
	existiert. Entsprechend für den rechtsseitigen Grenzwert für linkte Integrationsgrenze.
\end{definition}
\paragraph{Bemerkungen:}
\begin{itemize}
	\item Potenzreihen können im Innern des Konvergenzradius gliedweise integriert werden.
	\item Manchmal kann man auch mit Hilfe der Integrationsregeln keine Stammfunktion für gewisse Integranden finden. Auf diese Weise können viele neue Funktionen definiert werden. Zum Beispiel der Integralsinus:
	\begin{equation*}
		\operatorname{Si}(x)\coloneqq \int_0^x \frac{\sin(t)}{t}\intd t
	\end{equation*}
\end{itemize}
