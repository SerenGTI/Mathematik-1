\section{Extremwerte im Mehrdimensionalen}
\begin{definition}{Lokaler Extremwert}
	Sei $f:D\rightarrow \R,D\subseteq \R^n$ offen. Wir sagen $f$ hat im Punkt $a\in D$ ein lokales Maximum, falls ein $\delta>0$ existiert, so dass $K_\delta(a)\subseteq D$ ist und für alle $y\in K_\delta(a)$ gilt, dass $f(y)\leq f(a)$ ist. Falls sogar $f(y)<f(a)$ für alle $y\in K_\delta(a)\setminus\simpleset{a}$ gilt, spricht man von einem isolierten lokalen Maximum.

	Entsprechend für lokale Minima.
\end{definition}
\paragraph{Bemerkung:}
Falls $D$ nicht offen ist, kann am Rand von $D$ ein Extremum auftreten, auch wenn dort kein kritscher Punkt vorliegt.

\par\medskip

Im Eindimensionalen kann man lokale Extrema finden, indem man benutzt, dass die Ableitung dort eine Nullstelle hat (notw. Kriterium). Ist die Funktion zweimal stetig differenzierbar, kann man folgendes hinreichendes Kriterium benutzen: Falls
\begin{equation*}
	f'(x_0)=0 \wedge f''(x_0)<0
\end{equation*}
gilt, dann hat $f$ bei $x_0$ ein isoliertes lokales Maximum. Entsprechend ein isoliertes lokales Minimum, falls $f''(x_0)>0$.

Wir werden sehen, dass sich diese Kriterien ganz analog auf den mehrdimensionalen Fall verallgemeinern. Dabei übernimmt die Hessematrix die Rolle der zweiten Ableitung.

\begin{satz}{Notwendige Bedingung für lokalen Extremwert}
	Sei $f:D\rightarrow \R,D\subseteq \R^n$ stetig differenzierbar und besitze in $a$ einen lokalen Extremwert. Dann gilt
	\begin{equation*}
		Df(a)=f'(a)=0
	\end{equation*}
	das heißt es gilt $\frac{\partial f}{\partial x_i}(a)=0$ für alle $i\in\simpleset{1,\ldots,n}$.
\end{satz}
Dieses Kriterium verallgemeinert also direkt das notwendige Kriterium aus dem Eindimensionalen. Es ist jedoch nicht hinreichend für ein lokales Extremum, denn zum Beispiel
\begin{itemize}
	\item hat $f:\R\rightarrow\R,x\mapsto x^3$ bei $a=0$ eine Nullstelle der Ableitung aber kein Extremum.
	\item hat die Funktion $f(x,y)=xy$ die Jacobimatrix $\jacobi f(x,y)=(y,x)$ also gilt $Df(0,0)=(0,0)$. Aber da $f(x,y)>0$ für $x,y>0$ und $f(x,y)<0$ für $x<0<y$ gibt es in jeder noch so kleinen Kugelumgebung $K_\delta (0,0)$ sowohl negative als auch positive Funktionswerte (Sattelpunkt).
\end{itemize}

\begin{definition}{Sattelpunkt}
	Gilt für $f:D\rightarrow \R,D\subseteq \R^n$, dass in jeder Umgebung von $x_0\in D$ sowohl Punkte $y\in D$ mit $f(y)<f(x_0)$ als auch mit $f(y)>f(x_0)$ existieren, wobei $Df(x_0)=0$, dann sagt man $f$ hat bei $x_0$ einen Sattelpunkt.
\end{definition}

Wie bekommt man nun ein zum Eindimensionalen analoges hinreichendes Kriterium für isolierte lokale Extrema? Die Antwort gibt der Satz von Taylor:

Gilt $Df(a)=0$ so folgt
\begin{align*}
	f(a+h)&=f(a)+\underbrace{Df(a)*h}_{=0}+h^T*\hess f(a)*h+o(|\!|h|\!|^2)\\
	&=f(a)+h^T*\hess f(a)*h+o(|\!|h|\!|^2)
\end{align*}
Wir sehen, $f$ hat ein isoliertes lokales Minimum, falls $h^T*\hess f(a)*h$ für alle $h\in\R^n\setminus\simpleset{0}$ positiv ist.

\begin{definition}{Definitheit}
	Sei $A\in M(n,\R)$ eine symmetrische reelle $n\times n$-Matrix. Dann heißt $A$
	\begin{description}
		\item [positiv definit,] falls $h^T*\hess f(a)*h>0$ für alle $h\in\R^n\setminus\simpleset{0}$
		\item [negativ definit,] falls $h^T*\hess f(a)*h<0$ für alle $h\in\R^n\setminus\simpleset{0}$
		\item [indefinit,] falls es ein $v\in\R^n$ und ein $w\in\R^n$ gibt, so dass $v^T*\hess f(a)*v<0$

		und $w^T*\hess f(a)*w>0$ gilt.

		\item [positiv semidefinit,] falls $h^T*\hess f(a)*h\geq0$ für alle $h\in\R^n$
		\item [negativ semidefinit,] falls $h^T*\hess f(a)*h\leq0$ für alle $h\in\R^n$
	\end{description}
\end{definition}

\begin{satz}{Hinreichende Bedingung für lokalen Extremwert}
	Sei $f:D\rightarrow \R,D\subseteq\R^m$ dreimal stetig partiell differenzierbar und sei $a\in D$ ein kritscher Punkt, das heißt $Df(a)=0$ mit $K_\delta (a)\subseteq D$ für ein $\delta>0$.
	\begin{itemize}
		\item Ist $\hess f(a)$ positiv definit so liegt bei $a$ ein isoliertes lokales Minimum vor.
		\item Ist $\hess f(a)$ negativ definit so liegt bei $a$ ein isoliertes lokales Maximum vor.
		\item Ist $\hess f(a)$ indefinit so liegt bei $a$ ein Sattelpunkt vor.
	\end{itemize}
	Falls beim Versuch das Kriterium anzuwenden $\hess f(a)$ semidefinit ist, ist keine Aussage möglich!
\end{satz}
\begin{beweis}
	Siehe Satz von Taylor. \hfill $\Box$
\end{beweis}

\paragraph{Beispiel:}
Betrachte $f_\pm:\R^2\rightarrow \R, f(x,y)=x^2\pm y^4$. Es gilt $\jacobi f_\pm(x,y)=(2x, \pm4y^3)$ und
\begin{equation*}
	\hess f(x,y)=\matrix{2&0\\0&\pm 12y^2}.
\end{equation*}%
Somit auch $Df(0,0)=0$, es liegt also ein kritischer Punkt bei $(0,0)$ vor und es gilt $\hess f_\pm(0,0)=\matrix{2&0\\0&0}$. Diese symmetrische Matrix ist positiv semidefinit, denn
\begin{equation*}
	(v_1,v_2)\matrix{2&0\\0&0}\vector{v_1\\v_2}=2v_1^2\geq 0.
\end{equation*}
Die Funktion $f_+$ hat ein isoliertes lokales Minimum bei $a=(0,0)$. Die Funktion $f_-$ hat einen Sattelpunkt.


\par\medskip

Es stellt sich also die Frage: Wann ist eine symmetrische Matrix positiv definit, negativ definit beziehungsweise indefinit?

Natürlich gilt, dass falls $A$ positiv definit ist, $-A$ negativ definit ist, denn
\begin{equation*}
	(v_1,\ldots,v_n)*A*\vector{v_1\\\vdots\\v_n}=\sum_{i,j=1}^n v_iv_j a_{ij}
\end{equation*}

\paragraph{Bemerkungen:}
\begin{itemize}
	\item Die Einheitsmatrix ist positiv definit, denn $v^T Ev=v_1^2+\ldots+v_n^2$.
	Allgemeiner gilt, eine Diagonalmatrix mit positiven Diagonalelementen ist positiv definit, denn
	\begin{equation*}
		(v_1,\ldots,v_n)\matrix{\lambda_1&\cdots &0\\\vdots & \ddots & \vdots \\ 0&\cdots &\lambda_n}\vector{v_1\\\vdots\\v_n}=\lambda_1v_1^2+\ldots+\lambda_nv_n^2
	\end{equation*}
	Falls alle $\lambda_i<0$ sind, ist die Matrix negativ definit. Falls es positive und nagative $\lambda_i$ gibt, ist die Matrix indefinit.
	\item Auch wenn alle Einträge einer symmetrischen Matrix positiv sind, ist die Matrix nicht notwendigerweise positiv definit, so zum Beispiel
	\begin{equation*}
		(1,-1)\matrix{1&2\\2&1}\vector{1\\-1}=(1,-1)\vector{-1\\1}=-2.
	\end{equation*}
	\item Eine symmetrische reelle Matrix ist
	\begin{description}
		\item [positiv definit,] falls alle Eigenwerte positiv sind
		\item [negativ definit,] falls alle Eigenwerte negativ sind
		\item [indefinit,] falls es positive und negative Eigenwerte gibt.
	\end{description}
\end{itemize}
Um über Definitheit zu entscheiden gibt es das sogenannte Hurwitz'sche Kriterium.

\begin{definition}{Hauptminoren}
	Sei $A\in M(n,\R)$, dann bezeichnen wir mit $A_1,\ldots,A_n$ die folgenden Teilmatrizen
	\begin{equation*}
		A_1=(a_{11}), A_2=\matrix{a_{11}&a_{12}\\a_{21}&a_{22}},\ldots,A_k=\matrix{a_{11}&\cdots&a_{1k}\\\vdots&\ddots&\vdots\\a_{k1}&\cdots&a_{kk}},\ldots A_n=A
	\end{equation*}
	Die Determinanten $\det A_1, \det A_2,\ldots, \det A_n$ heißen \emph{Hauptminoren} von $A$.
\end{definition}

\begin{satz}{Hurwitz'sches Kriterium}
	$A$ ist positiv definit genau dann, wenn alle Hauptminoren positiv sind. $A$ ist negativ definit genau dann, wenn die Hauptminoren abwechselnd negativ und positiv sind (Das heißt beginnend mit negativ!).
\end{satz}
\paragraph{Warnung:} Das Kriterium sagt nichts über semidefinitheit aus!

\paragraph{Beispiel:}
Wir wollen alle lokalen Extrema der Funktion $f:\R^3\rightarrow \R, f(x,y,z)=x^2+y^2-\cos(z)$ bestimmen. Es gilt
\begin{equation*}
	\jacobi f(x,y,z)=(2x,2y,\sin(z))
\end{equation*}
Somit gilt $Df(x,y,z)=0$ genau dann, wenn $x=y=0, z=k\pi, k\in\Z$. Wir berechnen
\begin{equation*}
	\hess f(x,y,z,)=\matrix{2&0&0\\0&2&0\\0&0&\cos(z)}.
\end{equation*}
Somit gilt
\begin{equation*}
	\hess f(0,0,k\pi)=\matrix{2&0&0\\0&2&0\\0&0&\cos(k\pi)}
\end{equation*}
Fallunterscheidung für gerade und ungerade $k$:
\begin{align*}
	\matrix{2&0&0\\0&2&0\\0&0&\cos(2k\pi)} \text{ ist positiv definit}\\
	\matrix{2&0&0\\0&2&0\\0&0&\cos(2(k+1)\pi)} \text{ ist indefinit}
\end{align*}
Somit liegen sind die lokalen Extrema der Funktion $f$ genau die Punkte $(0,0,2k\pi)$. Dort hat $f$ isolierte lokale Minima. (Bei $(0,0,2(k+1)\pi)$ liegen Sattelpunkte vor.)
