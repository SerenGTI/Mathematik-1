Die quadratischen Matrizen $M(n,K)$ bilden einen im Allgemeinen nicht kommutativen Ring mit der Matrixaddition und -multiplikation.

Es gelten:
\begin{equation*}
	A*(B+C)=A*B+A*C
\end{equation*}
\begin{equation*}
	(A+B)*C=A*C+B*C
\end{equation*}

Das neutrale Element bezüglich der Multiplikation ist die sogenannte $n\times n$-Einheitsmatrix:
\begin{equation*}
	E=E_n=\matrix{
		1 & 0 & \cdots & 0\\
		0 & 1 & \cdots & 0\\
		\vdots & \vdots & \ddots & \vdots\\
		0 & 0 & \cdots & 1
		}
\end{equation*}
mit anderen Worten:
\begin{equation*}
	E=(\delta_{ij})_{1\leq i \leq n} \text{ wobei }
	\delta_{ij}=
	\begin{cases}
		1\text{, falls $i=j$} \\
		0\text{ sonst}
	\end{cases}
\end{equation*}
$\delta_{ij}$ wird auch das \textsc{Kronecker}-Delta genannt.

Die $n\times n$-Einheitsmatrix ist die darstellende Matrix der identischen Abbildung $id_{K^n}$.

\begin{definition}{Inverse Matrix}
	$A\in M(n,K)$ heißt invertierbar, falls es eine Matrix $A^{-1}$ gibt mit $A^{-1}\in M(n,K)$ so, dass $A*A^{-1}=A^{-1}*A=E_n$ gilt.

	In diesem Fall nennt man $A^{-1}$ die inverse Matrix von $A$.
\end{definition}

\begin{satz}{Allgemeine lineare Gruppe}
	Die Menge $\mathrm{GL}(n,K)\coloneqq \set{A\in M(n,K)}{\exists A^{-1} : A*A^{-1} = A^{-1}*A = E_n}$ bildet eine Gruppe mit der Matrixmultiplikation.
\end{satz}

\beweis
\begin{enumerate}
	\item Matrixmultiplikation ist assoziativ, da sie die Abbildungsverkettung darstellt.
	\item $E_n$ ist das neutrale Element.
	\item Außerdem besitzen invertierbare Matrizen natürlich ein Inverses.
\end{enumerate}

$\mathrm{GL}(n,K)$ wird auch als die allgemeine lineare Gruppe vom Grad $n$ über dem Körper $K$ bezeichnet.

\begin{satz}{}
	Eine Matrix $A$ ist invertierbar genau dann, wenn die lineare Abbildung $x\mapsto A*x$ bijektiv ist. Ihre Umkehrabbildung ist durch $x\mapsto A^{-1}*x$ gegeben.
\end{satz}
\beweis
\begin{description}
	\item[\glqq$\Leftarrow$\grqq]
	$f:K^n\rightarrow K^n, f(x)=A*x$ bijektiv, dann gilt für die darstellende Matrix $B$ der Umkehrabbildung $f^{-1}:K^n\rightarrow K^n$, dass $A*B=E_n=B*A$. Das heißt, die darstellende Matrix $B$ ist die Inverse von $A$.
	\item[\glqq$\Rightarrow$\grqq]
	Ist $A$ invertierbar, dann ist durch $x\mapsto A^{-1}*x$ die Umkehrabbildung gegeben, denn $A^{-1}*(A*x)=E*x=x$
\end{description}
