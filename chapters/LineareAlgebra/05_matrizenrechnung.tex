\chapter{Matrizenrechnung}
Sei $M(m,n,K)$ die Menge der $m\times n$-Matrizen mit Einträgen aus K.

Matrizen, deren Zeilenzahl mit der Spaltenzahl übereinstimmen nennt man \emph{quadratisch}. Wir beschreiben sie mit $M(n,K)\coloneqq M(n,n,K)$.

Für eine Matrix $A\in M(n,K)$ schreibt man:

\begin{equation*}
  A=
  \matrix{
  \ma{11} & \ma{12} & \cdots & \ma{1n}\\
  \ma{21} & \ma{22} & \cdots & \ma{2n}\\
  \vdots & \vdots & \ddots & \vdots\\
  \ma{21} & \ma{22} & \cdots & \ma{2n}\\
  }
  = ((a_{ij}))_{\substack{1\leq i\leq m\\1\leq j\leq n}}
\end{equation*}

\begin{definition}{Matrizenaddition}
	Die Addition zweier Matrizen $A=(a_{ij}),B=(b_{ij})\in M(m,n,K)$ gleicher Zeilen- und Spaltenzahl ist komponentenweise definiert:

	$C\coloneqq A+B$ wobei $c_{ij}=(a_{ij})+(b_{ij}) \quad\forall 1\leq i\leq m, 1\leq j\leq n$
\end{definition}

\begin{definition}{Skalare Multiplikation}
	Die skalare Multiplikation einer Matrix $A=(a_{ij}\in M(m,n,K)$ mit $\lambda \in K$ ist definiert durch:

	$\lambda A\coloneqq \lambda(a_{ij}) \quad\forall 1\leq i\leq m, 1\leq j\leq n$ (wiederum komponentenweise)
\end{definition}

\bemerkung
Mit diesen beiden Operationen wird $M(m,n,K)$ zu einem $K$-Vektorraum. Dieser ist isomorph zu $K^{m*n}$. D.h. es gibt einen Vektorraumisomorphismus $M(m,n,K)\rightarrow K^{m*n}$.
\begin{equation*}
	M(m,n,K)\overset \sim = K^{m*n}
\end{equation*}
Deswegen sieht man auch die Bezeichnung $K^{m*n}$ für $M(m,n,K)$.

\begin{definition}{Matrixprodukt}
	Seien $A\in M(l,{\color{red} m},K), B\in M({\color{red} m},n,K)$ d.h. stimmen die Spaltenzahl von $A$ mit der Zeilenzahl von $B$ überein.

	Dann ist das \emph{Matrixprodukt}:

	\begin{align*}
		A*B&=C\in M(l,n,K)
		\intertext{definiert durch:}
		C&=(c_{ij})=\left(\sum\limits_{k=1}^m a_{ik} * a_{kj}\right)
	\end{align*}
\end{definition}

\paragraph{Merkregel} Zeile mal Spalte

\bemerkung
\begin{itemize}
	\item Die Matrixmultiplikation ist \emph{nicht} kommutativ!
	\item Spezialfall: Anwenden einer Matrix auf einen Spaltenvektor: Man fasst Spaltenvektoren aus $K^n$ als $n\times 1$-Matrizen auf.
\end{itemize}

\begin{satz}{}
	Das Matrixprodukt entspricht der Verkettung von linearen Abbildungen.
\end{satz}
Genauer: Seien $U,V,W$ drei $K$-Vektorräume mit den Basen
\begin{align*}
	\mathcal{B}&=\simpleset{u_1,\ldots, u_n},\\
	\mathcal{C}&=\simpleset{v_1,\ldots, v_n},\\
	\mathcal{D}&=\simpleset{w_1,\ldots, w_n}
\end{align*}

\begin{itemize}
	\item $A\in M(l,m,K)$ die darstellende Matrix von $f:V\rightarrow W$ bezüglich $\mathcal{C}$ und $\mathcal{D}$.
	\item $B\in M(m,n,K)$ die darstellende Matrix von $g:U\rightarrow V$ bezüglich $\mathcal{B}$ und $\mathcal{C}$.
\end{itemize}
Dann ist $A*B\in M(l,n,K)$ die darstellende Matrix von $f\circ g = f(g):U\rightarrow W$ bezüglich den Basen $\mathcal{B}$ und $\mathcal{D}$.

\beweis
Es gilt:
\begin{align*}
	g(u_j)&=\sum\limits_{i=1}^m ( b_{ij} * v_i )\\
	\intertext{und somit:}
	f(g(u_j)) &= \sum\limits_{i=1}^m ( b_{ij} * f(v_j) ) = \sum\limits_{i=1}^m b_{ij} * \left( \sum\limits_{p=1}^l a_{pi} * w_p \right) =\sum\limits_{p=1}^l \left( \sum\limits_{i_1}^m a_{pi}*a_{ij} \right) * w_p\\
	&= \underbrace{\sum\limits_{p=1}^l (c_{pj} * w_p)}_\text{Matrixprodukt}
\end{align*}


Die quadratischen Matrizen $M(n,K)$ bilden einen im Allgemeinen nicht kommutativen Ring mit der Matrixaddition und -multiplikation.

Es gelten:
\begin{equation*}
	A*(B+C)=A*B+A*C
\end{equation*}
\begin{equation*}
	(A+B)*C=A*C+B*C
\end{equation*}

Das neutrale Element bezüglich der Multiplikation ist die sogenannte $n\times n$-Einheitsmatrix:
\begin{equation*}
	E=E_n=\matrix{
		1 & 0 & \cdots & 0\\
		0 & 1 & \cdots & 0\\
		\vdots & \vdots & \ddots & \vdots\\
		0 & 0 & \cdots & 1
		}
\end{equation*}
mit anderen Worten:
\begin{equation*}
	E=(\delta_{ij})_{1\leq i \leq n} \text{ wobei }
	\delta_{ij}=
	\begin{cases}
		1\text{, falls $i=j$} \\
		0\text{ sonst}
	\end{cases}
\end{equation*}
$\delta_{ij}$ wird auch das \textsc{Kronecker}-Delta genannt.

Die $n\times n$-Einheitsmatrix ist die darstellende Matrix der identischen Abbildung $id_{K^n}$.

\begin{definition}{Inverse Matrix}
	$A\in M(n,K)$ heißt invertierbar, falls es eine Matrix $A^{-1}$ gibt mit $A^{-1}\in M(n,K)$ so, dass $A*A^{-1}=A^{-1}*A=E_n$ gilt.

	In diesem Fall nennt man $A^{-1}$ die inverse Matrix von $A$.
\end{definition}

\begin{satz}{Allgemeine lineare Gruppe}
	Die Menge $\mathrm{GL}(n,K)\coloneqq \set{A\in M(n,K)}{\exists A^{-1} : A*A^{-1} = A^{-1}*A = E_n}$ bildet eine Gruppe mit der Matrixmultiplikation.
\end{satz}

\beweis
\begin{enumerate}
	\item Matrixmultiplikation ist assoziativ, da sie die Abbildungsverkettung darstellt.
	\item $E_n$ ist das neutrale Element.
	\item Außerdem besitzen invertierbare Matrizen natürlich ein Inverses.
\end{enumerate}

$\mathrm{GL}(n,K)$ wird auch als die allgemeine lineare Gruppe vom Grad $n$ über dem Körper $K$ bezeichnet.

\begin{satz}{}
	Eine Matrix $A$ ist invertierbar genau dann, wenn die lineare Abbildung $x\mapsto A*x$ bijektiv ist. Ihre Umkehrabbildung ist durch $x\mapsto A^{-1}*x$ gegeben.
\end{satz}
\beweis
\begin{description}
	\item[\glqq$\Leftarrow$\grqq]
	$f:K^n\rightarrow K^n, f(x)=A*x$ bijektiv, dann gilt für die darstellende Matrix $B$ der Umkehrabbildung $f^{-1}:K^n\rightarrow K^n$, dass $A*B=E_n=B*A$. Das heißt, die darstellende Matrix $B$ ist die Inverse von $A$.
	\item[\glqq$\Rightarrow$\grqq]
	Ist $A$ invertierbar, dann ist durch $x\mapsto A^{-1}*x$ die Umkehrabbildung gegeben, denn $A^{-1}*(A*x)=E*x=x$
\end{description}
