\chapter{Vektorräume}
\paragraph{Beispiel}
\begin{align*}
  \R^2&=\R\times\R=\set{(x,y)}{x,y\in\R}\\
  \R^3&=\set{(x,y,z)}{x,y,z\in\R}\\
  &\vdots\\
  \R^n&=\set{(x_1,x_2,\ldots,x_n)}{x_1,\ldots,x_n\in\R}\\
\end{align*}
Wir schreiben die Elemente von $\R^n$ auch als sogenannte Spaltenvektoren:
\begin{equation*}
  \begin{pmatrix}
    x_1\\
    x_2\\
    \vdots\\
    x_n
  \end{pmatrix} \text{ anstatt von } (x_1,x_2,\ldots,x_n)
\end{equation*}

Mit der komponentenweisen Addition, der Vektoraddition:
\begin{equation*}
  \begin{pmatrix}
    x_1\\
    \vdots\\
    x_n
  \end{pmatrix}
  +
  \begin{pmatrix}
    y_1\\
    \vdots\\
    y_n
  \end{pmatrix}
  =
  \begin{pmatrix}
    x_1+y_1\\
    \vdots\\
    x_n+y_n
  \end{pmatrix}
\end{equation*}
wird $\R^n$ zu einer abel'schen Gruppe mit dem Nullvektor als neutrales Element und dem negierten Vektor als inverses Element bezüglich der Addition.

In der Vektorrechnung nennt man Zahlen (z.B. Elemente aus $\R,\C,\Q$) \emph{Skalare}, um Zahlen und Vektoren deutlich zu unterscheiden.

Sei $x\coloneqq \begin{pmatrix}
  x_1\\
  \vdots\\
  x_n
\end{pmatrix} \in \R$ und $\lambda\in\R$. Dann ist die \emph{skalare Multiplikation} $x*\lambda$ definiert durch $x*\lambda \coloneqq \begin{pmatrix}
  \lambda* x_1\\
  \vdots\\
  \lambda* x_n
\end{pmatrix}$

Die beiden Operationen Vektoraddition und skalare Multiplikation sind kennzeichnend für einen Vektorraum.


\definition{Vektorraum}
Sei $K$ ein Körper, dessen neutrales Element bezüglich der Multiplikation mit $1_K$ bezeichnet wird. Sei $V$ eine Menge mit einer Verknüpfung $+$, so dass $(V,+)$ eine abel'sche Gruppe bildet.

Sei weiter eine Abbildung, genannt \emph{skalare Multiplikation} $K\times V\rightarrow V$ gegeben, so dass folgende Bedingungen $\forall \alpha,\beta \in K; x,y\in V$ gelten:

\begin{description}
  \item[V 1] $(\alpha*\beta)* x=\alpha*(\beta* x)$ (assoziativ)
  \item[V 2] $1_K* x = x$ (neutrales Element des Körpers ist das neutrale bzgl $*$)
  \item[V 3] $(\alpha+\beta)* x=\alpha * x + \beta* x$ (distributiv 1)
  \item[V 4] $\alpha* (x+y)=\alpha * x + \alpha* y$ (distributiv 2)
\end{description}
Dann ist $V$ ein \emph{Vektorraum} über dem Körper $K$. Kurz auch $K$-Vektorraum. Die Verknüpfung $+$ wird Vektoraddition genannt. Für $K=\R$ bzw. $K=\C$ spricht man auch von einem reellen, bzw. komplexen Vektorraum.

Elemente von $V$ nennt man Vektoren.

\paragraph{Beispiele}
\begin{multicols}{2}
  \begin{itemize}
    \item $\R^2, \R^3, \ldots$
    \item $\C^2$
    \item $\{0\}$ ist ein Vektorraum für jeden Körper $K$.
    \columnbreak
    \item Sei $V=\set{f}{f:\R\rightarrow\R}$ die Menge der reellen Funktionen in einer Variable. Durch die punktweise Addition

    $(f+g)(x)=f(x)+g(x)$

    und die punktweise skalare Multiplikation

    $(\lambda f)(x)=\lambda* f(x)$

    wird $V$ zu einem Vektorraum.
  \end{itemize}
\end{multicols}


\definition{Untervektorraum}
Sei $V$ ein $K$-Vektorraum. Eine nichtleere Teilmenge $U\subseteq V$ heißt Untervektorraum bzw. Teilvektorraum, falls gilt:
\begin{description}
  \item[UV 1] Abschluss unter Vektoraddition:
  \begin{equation*}
    \forall u,v : u,v \in U \Rightarrow u+v\in U
  \end{equation*}
  \item[UV 2] Abschluss unter skalarer Multiplikation:
  \begin{equation*}
    \forall u\in U, \lambda\in K : \lambda* u \in U
  \end{equation*}
\end{description}

\paragraph{Beispiele}
Die folgenden sind Untervektorräume von $\R^2$:
\begin{itemize}
  \item $U_1\coloneqq \set{\begin{pmatrix}x\\0\end{pmatrix}}{x\in\R}$ (die $x$-Achse)
  \item $U_2\coloneqq \set{\begin{pmatrix}x\\x\end{pmatrix}}{x\in\R}$ (die Winkelhalbierende des 1. und 3. Quadranten)
\end{itemize}


\begin{lemma}{}
  Für alle $\lambda \in K, v\in V$ wobei $V$ ein $K$-Vektorraum ist, gilt:
  \begin{enumerate}
    \item $0_K* v = 0_V$
    \item $(-\lambda)* v = -(\lambda* v)$
  \end{enumerate}
\end{lemma}
\beweis
\begin{enumerate}
  \item Es gilt:
  \begin{align*}
    0* v = (0+0)* v &\underset{\text{\textbf{(V 3)}}}{=}0* v+ 0* v\\
    0* v + (-(0* v)) &=(0* v + 0* v)+(-(0*v))\\
    &\underset{\text{\textbf{(V 1)}}}{=} 0*v+(0*v+(-0*v))\\
    0=0*v +0&=0*v
  \end{align*}
\end{enumerate}

\definition{Linearkombination}
Seien $v_1,v_2,\ldots,v_k$ Vektoren aus dem $K$-Vektorraum $V$ und seien $\lambda_1,\lambda_2,\ldots,\lambda_k \in K$. Dann heißt der Vektor
\begin{equation*}
  u=\lambda_1v_1+\lambda_2v_2+\ldots+\lambda_kv_k = \sum\limits_{j=1}^k\lambda_jv_j
\end{equation*}
\emph{Linearkombination} von den Vektoren $v_1,v_2,\ldots,v_k$.
Die Skalare $\lambda_1,\lambda_2,\ldots,\lambda_k$ heißen \emph{Koeffizienten} der Linearkombination.

Sind in der Linearkombination alle Koeffizienten gleich Null, handelt es sich um die \emph{triviale Linearkombination}. Gibt es hingegen mindestens einen Koeffizienten $\lambda_j \neq 0$, handelt es sich um einee \emph{nichttriviale Linearkombination}.


\definition{}
Sei $V$ ein $K$-Vektorraum, $M\subseteq V$ eine Teilmenge. Dann heißt die Menge aller Linearkombinationen
\begin{equation*}
  \set{\lambda_1v_1+\ldots+\lambda_kv_k}{v_1,v_2,\ldots,v_k \in M, \lambda_1,\lambda_2,\ldots,\lambda_k \in K}
\end{equation*}
der \emph{Spann} oder die lineare Hülle von M.

\begin{equation*}
  \spann{M}\coloneqq \set{\sum\limits_{j=1}^k\lambda_jv_j}{\lambda_j \in K, v_j\in M}
\end{equation*}

\paragraph{Beispiele}
\begin{itemize}
  \item $v=\vector{1\\1\\0}$ in $\R^3$ $\leadsto \spann{\{v\}} = \set{\vector{\lambda\\\lambda\\0}}{\lambda \in \R}$
  \item $\spann{\simpleset{\vector{1\\0\\0},\vector{0\\1\\0}}} = \set{\vector{x_1\\x_2\\0}}{x_1,x_2\in\R}$ ($x_1,x_2$-Ebene)
\end{itemize}

\begin{satz}{}
  Sei $V$ ein $K$-Vektorraum und $M\subseteq V$. Dann ist $\spann{M}$ ein Untervektorraum von $V$.
\end{satz}
\beweis
\begin{enumerate}
  \item $\spann{M}$ ist nicht leer, da der Nullvektor als leere Linearkombination mindestens enthalten ist.
  \item Abschluss unter skalarer Multiplikation, sei $\lambda \in K, v\in \spann{M}$:
  \begin{align*}
    v &= \lambda_1v_1+\ldots+\lambda_kv_k \quad\text{wobei } v_1,\ldots,v_k \in M\\
    \lambda v &= \lambda(\lambda_1v_1+\ldots+\lambda_kv_k)\\
    &= \lambda(\lambda_1v_1)+\ldots+\lambda(\lambda_kv_k)\\
    &= (\lambda\lambda_1)v_1+\ldots+(\lambda\lambda_k)v_k\\
  \end{align*}
  \item Abschluss unter Addition:

\end{enumerate}


\definition{Erzeugendensystem}
Gilt $V=\spann{M}$ für einen $K$-Vektorraum $V$ und eine Teilmenge $M\subseteq V$, so sagt man $M$ ist ein \emph{Erzeugendensystem} von $V$.

Interessant ist die minimale Anzahl an Vektoren in einem Erzeugendensystem, bzw. ein \emph{minimales Erzeugendensystem}.

\definition{Lineare Abhängigkeit}\label{satz:lineareUnabhaengigkeit}
Eine Menge von Vektoren $M\subseteq V$ heißt \emph{linear abhängig}, wenn es eine nichttriviale Linearkombination gibt, die den Nullvektor ergibt. Andernfalls heißt $M$ \emph{linear unabhängig}!

\begin{satz}{}
  Eine Menge von Vektoren ist genau dann linear abhängig, wenn einen Vektor $v\in M$ gibt, der sich als Linearkombination mit Vektoren aus $M\setminus\simpleset{v}$ darstellen lässt.
\end{satz}


\begin{description}
  \newcommand{\lv}[1]{\ensuremath\lambda_{#1}v_{#1}}
  \item[\glqq$\Rightarrow$\grqq] Angenommen, $M$ ist linear abhängig. Dann gibt es Vektoren $v_1,\ldots,v_n$ und Koeffizienten $\lambda_1,\ldots,\lambda_n \in K$, so dass die Linearkombination \emph{nichttrivial} den Nullvektor ergibt. Dann folgt:
  \begin{align*}
    \lv{j} &= -\lv{1}-\lv{2}-\ldots-\lv{j-1}-\lv{j+1}-\ldots-\lv{n} \quad |\lambda_j\neq0\\
    v_j &= \frac{1}{\lambda_j} * \left( -\lv{1}-\lv{2}-\ldots-\lv{j-1}-\lv{j+1}-\ldots-\lv{n} \right)
  \end{align*}
  Damit ist $v_j$ als nichttriviale Linearkombination von Vektoren aus $M\setminus\simpleset{v_j}$ dargestellt.


  \item[\glqq$\Leftarrow$\grqq] Angenommen, es gibt einen Vektor $v\in M$ sowie Vektoren $v_1,\ldots,v_n\in M\setminus\simpleset{v}$ und Koeffizienten $\lambda_1,\ldots,\lambda_n \in K$, so dass gilt:
  \begin{align*}
    v &= \lv{1}+\ldots+\lv{n}\\
    0 &= \lv{1}+\ldots+\lv{n} - 1*v
  \end{align*}
  Dies ist eine nichttriviale Linearkombination mit Vektoren aus $M$, die $0$ ergibt.
\end{description}


\definition{Basis}
Eine Teilmenge $B$ eines Vektorraums $V$ heißt \emph{Basis} von $V$ falls $B$ ein linear unabhängiges Erzeugendensystem ist.

\paragraph{Beispiele}
Für jeden Körper $K$ gibt es die Standardbasis bzw. die \emph{kanonische Basis} $\simpleset{e_1,e_2,\ldots,e_n}$ von $K^n$:
\begin{equation*}
  e_1=\vector{1\\0\\\vdots\\ 0}, e_2=\vector{0\\1\\\vdots\\ 0}, \ldots, e_n=\vector{0\\0\\\vdots\\ 1}
\end{equation*}
Diese sind linear unabhängig, nach der Folgerung zu \autoref{satz:lineareUnabhaengigkeit}. Die Standardbasis ist ein Erzeugendensystem, da
\begin{equation*}
  \vector{x_1\\x_2\\\vdots\\x_n}=x_1e_1+x_2e_2+\ldots+x_ne_n
\end{equation*}

Im Allgemeinen gibt es verschiedene Basen von demselben Vektorraum.

\begin{satz}{Charakterisierungen von Basen}
  Für eine Teilmene $B\subseteq V$ eines Vektorraums sind folgene Sätze äquivalent:
  \begin{itemize}
    \item $B$ ist eine Basis
    \item Jeder Vektor in $V$ lässt sich auf genau eine Weise als Linearkombination von Vektoren aus $B$ schreiben.
    \item $B$ ist ein minimales Erzeugendensystem von $V$.
    \item $B$ ist eine maximal linear unabhängige Teilmenge von $V$
  \end{itemize}
\end{satz}

\bemerkung
Jeder Vektorraum besitzt eine Basis, jede Basis hat gleich viele Elemente. (auch $\emptyset$ oder $|B|=\infty$ möglich)

\definition{Dimension}
Die Anzahl der Elemente der Basis $B$ eines Vektorraums $V$ nennt man \emph{Dimension}
\begin{equation*}
  \mathrm{dim}(V)=|B|
\end{equation*}
