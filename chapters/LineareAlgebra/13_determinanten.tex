\begin{satz}{Regel von Sarrus}
	Für die Determinante einer $3\times3$-Matrix gilt:
	\begin{align*}
		&\det\matrix{
		\ma{11} & \ma{12} & \ma{13}\\
		\ma{21} & \ma{22} & \ma{23}\\
		\ma{31} & \ma{32} & \ma{33}
		}\\
		&=\ma{11}\ma{22}\ma{33}+\ma{12}\ma{23}\ma{31}+\ma{13}\ma{21}\ma{32}\\
		&\quad-\ma{31}\ma{22}\ma{13}-\ma{32}\ma{23}\ma{11}-\ma{33}\ma{21}\ma{12}
	\end{align*}
\end{satz}
\paragraph{Merkregel:} \glqq Jägerzaunregel\grqq

\paragraph{Vorsicht!} Verallgemeinert sich nicht auf höhere Dimensionen.

\beweis
Entwickeln nach der ersten Spalte:
\begin{align*}
	\det\matrix{
	\ma{11} & \ma{12} & \ma{13}\\
	\ma{21} & \ma{22} & \ma{23}\\
	\ma{31} & \ma{32} & \ma{33}
	}&=\ma{11}*\detmatrix{\ma{22} & \ma{23}\\\ma{32} & \ma{33}}
	-\ma{21}*\detmatrix{\ma{12} & \ma{13}\\\ma{32} & \ma{33}}
	+\ma{31}*\detmatrix{\ma{12} & \ma{13}\\\ma{22} & \ma{23}}\\
	&=\ma{11}(\ma{22}\ma{33}-\ma{32}\ma{23})-\ma{21}(\ma{12}\ma{33}-\ma{32}\ma{13})\\
	&\quad+\ma{31}(\ma{12}\ma{23}-\ma{22}\ma{13})\\
	&=\ma{11}\ma{22}\ma{33}+\ma{12}\ma{23}\ma{31}+\ma{13}\ma{21}\ma{32}\\
	&\quad-\ma{31}\ma{22}\ma{13}-\ma{32}\ma{23}\ma{11}-\ma{33}\ma{21}\ma{12}
\end{align*}


\begin{satz}{\textsc{Leibniz}'sche Formel}
	Für die Determinante einer $n\times n$-Matrix $A\in M(n,K)$ gilt:
	\begin{equation*}
		\det A=\sum_{\sigma\in \mathrm{Sym}(n)}\mathrm{sgn}(\sigma)*\ma{1\sigma(1)}*\ldots*\ma{n\sigma(n)}
	\end{equation*}
	wobei $\mathrm{Sym}(n)=
	\simpleset{
		\sigma:
		\simpleset{1,\ldots,n}
			\overset{\text{bijektiv}}{\longrightarrow}
		\simpleset{1,\ldots,n}
	}$ die Menge aller Permutationen von $\simpleset{1,\ldots,n}$ (auch die symmetrische Gruppe vom Grad n genannt) ist. Und wobei $\mathrm{sgn}$ das Vorzeichen der Permutation ist, d.h. $\mathrm{sgn}(\sigma)=+1$ bei einer geraden Permutation (Hintereinanderausführung von einer geraden Anzahl an Vertauschungen), $\mathrm{sng}(\sigma)=-1$ sonst.
\end{satz}

\paragraph{Weitere Bemerkungen zur Determinante:}
\begin{itemize}
	\item Für die Transponierte Matrix $A^T=(a_{ji})$ gilt $\det A^T=\det A$.
	Dies folgt direkt aus der Leibniz'schen Formel. Insbesondere kann man mit der LaPlace'schen Formel auch nach einer Zeile entwickeln.
	\item Für zwei Matrizen $A,B\in M(n,K)$ gilt der Determinantenmultiplikationssatz:
	\begin{equation*}
		\det(A*B)=\det A*\det B
	\end{equation*}
	\item Für Blockdiagonal- bzw. Bockdreiecksmatrizen gilt
	\begin{equation*}
		\det\matrix{A&0\\0&B}=\det\matrix{A&C\\0&B}=\det A*\det B
	\end{equation*}
	für $A\in M(n,K), B\in M(m,K), C\in M(n,K)$
	\item Oft schreibt man auch $|A|$ anstelle von $\det A$.
\end{itemize}

\section{Gaußalgorithmus Teil 2:}
Oft hat man das Problem für einen gegebenen (Unter-)Vektorraum, der von einer endlichen Menge von Vektoren aufgespannt wird, eine Basis zu ermitteln. Die beiden folgenden Sätze zeigen, dass man hierfür mit dem Gaußalgorithmus verwenden kann.

\begin{satz}{}
	Elementare Zeilenumformungen ändern nichts am Spann der Zeilenvektoren einer Matrix.
\end{satz}
\begin{satz}{}
	Die von Null verschiedenen Zeilen einer Matrix in Zeilenstufenform sind linear unabhängig.
\end{satz}

\paragraph{Rezept zum Bestimmen einer Basis:}
\begin{enumerate}
	\item Bilde die Matrix, die diese Vektoren als Zeilen hat.
	\item Bringe die Matrix auf Zeilenstufenform.
	\item Die von Null verschiedenen Zeilen bilden eine Basis, insbesondere ist die Dimension gleich der Anzahl der Vektoren.
\end{enumerate}
