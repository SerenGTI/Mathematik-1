\begin{satz}{Regel von Sarrus}
	Für die Determinante einer $3\times3$-Matrix gilt:
	\begin{align*}
		&\det\matrix{
		\ma{11} & \ma{12} & \ma{13}\\
		\ma{21} & \ma{22} & \ma{23}\\
		\ma{31} & \ma{32} & \ma{33}
		}\\
		&=\ma{11}\ma{22}\ma{33}+\ma{12}\ma{23}\ma{31}+\ma{13}\ma{21}\ma{32}\\
		&\quad-\ma{31}\ma{22}\ma{13}-\ma{32}\ma{23}\ma{11}-\ma{33}\ma{21}\ma{12}
	\end{align*}
\end{satz}
\paragraph{Merkregel:} \glqq Jägerzaunregel\grqq

\paragraph{Vorsicht!} Verallgemeinert sich nicht auf höhere Dimensionen.

\beweis
Entwickeln nach der ersten Spalte:
\begin{align*}
	\det\matrix{
	\ma{11} & \ma{12} & \ma{13}\\
	\ma{21} & \ma{22} & \ma{23}\\
	\ma{31} & \ma{32} & \ma{33}
	}&=\ma{11}*\detmatrix{\ma{22} & \ma{23}\\\ma{32} & \ma{33}}
	-\ma{21}*\detmatrix{\ma{12} & \ma{13}\\\ma{32} & \ma{33}}
	+\ma{31}*\detmatrix{\ma{12} & \ma{13}\\\ma{22} & \ma{23}}\\
	&=\ma{11}(\ma{22}\ma{33}-\ma{32}\ma{23})-\ma{21}(\ma{12}\ma{33}-\ma{32}\ma{13})\\
	&\quad+\ma{31}(\ma{12}\ma{23}-\ma{22}\ma{13})\\
	&=\ma{11}\ma{22}\ma{33}+\ma{12}\ma{23}\ma{31}+\ma{13}\ma{21}\ma{32}\\
	&\quad-\ma{31}\ma{22}\ma{13}-\ma{32}\ma{23}\ma{11}-\ma{33}\ma{21}\ma{12}
\end{align*}

\section{Bemerkungen}
\subsection{\textsc{Leibniz}'sche Formel}
\begin{satz}{}
	Für die Determinante einer $n\times n$-Matrix $A\in M(n,K)$ gilt:
	\begin{equation*}
		\det A=\sum_{\sigma\in \mathrm{Sym}(n)}\mathrm{sgn}(\sigma)*\ma{1\sigma(1)}*\ldots*\ma{n\sigma(n)}
	\end{equation*}
	wobei $\mathrm{Sym}(n)=
	\simpleset{
		\sigma:
		\simpleset{1,\ldots,n}
			\overset{\text{bijektiv}}{\longrightarrow}
		\simpleset{1,\ldots,n}
	}$ die Menge aller Permutationen von $\simpleset{1,\ldots,n}$ (auch die symmetrische Gruppe vom Grad n genannt) ist. Und wobei $\mathrm{sgn}$ das Vorzeichen der Permutation ist, d.h. $\mathrm{sgn}(\sigma)=+1$ bei einer geraden Permutation (Hintereinanderausführung von einer geraden Anzahl an Vertauschungen), $\mathrm{sng}(\sigma)=-1$ sonst.
\end{satz}
