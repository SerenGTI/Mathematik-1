\chapter{Determinanten und der Gauß-Algorithmus}
\section{Determinanten}
\begin{definition}{Determinante einer $2\times 2$-Matrix}
	Für eine $2\times2$-Matrix $A=\matrix{a & b\\c & d}\in M(2,K)$ definieren wir die \emph{Determinante} von $A$ durch

	\begin{equation*}
		\det A=\det\matrix{a & b\\c & d}=\detmatrix{a & b\\c & d}=ad-bc
	\end{equation*}
\end{definition}
\begin{satz}{Invertierbarkeit einer $2\times 2$-Matrix}
	Eine $2\times 2 $-Matrix $\matrix{a & b\\c & d}\in M(2,K)$ ist genau dann invertierbar, wenn $ad-bc\neq0$ gilt.
\end{satz}

Wir wollen die Definition auf quadratische Matrizen beliebiger Größe erweitern:
\begin{equation*}
	\det:M(n,K)\rightarrow K
\end{equation*}
$A$ soll genau dann invertierbar sein, wenn $\det A\neq 0$.

Dazu fassen wir eine $n\times n$-Matrix als ein $n$-Tupel von $n$-Zeilenvektoren auf, also als ein Element von
\begin{equation*}
	(K^n)^n= \underbrace{K^n\times K^n\times \ldots \times K^n}_{n\text{ mal}}
\end{equation*}

\begin{itemize}
	\item Eine Abbildung $d:V^n\rightarrow K$ wobei $V$ ein $K$-Vektorraum ist, heißt \emph{multilinear}, wenn
	\begin{equation*}
		V\rightarrow K, x\mapsto d(v_1,\ldots,v_{i-1},x,v_{i+1},\ldots,v_n)
	\end{equation*}
	für jedes $i\in\simpleset{1,\ldots, n}$ und alle $v_k\in V$ eine lineare Abbildung ist. Oder kurz gesagt, wenn sie in allen Argumenten linear ist.
	\item Sie heißt alternierend, wenn sie den Wert $0$ annimmt sobald zwei der Argumente gleich sind.
	\item Sie heißt normiert, falls $d(e_1,e_2,\ldots, e_n)=1$ gilt, sie also auf die Einheitsmatrix angewendet die Zahl Eins ergibt.
\end{itemize}
(Die oben definierte Determinante für $2\times 2$-Matrizen hat diese Eigenschaften)

\begin{definition}{Determinante}
	Es gibt genau eine Abbildung
	\begin{equation*}
		(K^n)^n\rightarrow K
	\end{equation*}
	die multilinear, alternierend und normiert ist.

	Der Wert dieser Abbildung auf die Zeilen einer Matrix $A\in M(n,K)$ angewendet heißt Determinante einer Matrix:
	\begin{equation*}
		\det A
	\end{equation*}
\end{definition}


\subsection{Berechnung der Determinante}
Man kann $\det A$ mithilfe des Gaußalgorithmus berechnen:
\begin{satz}{}
	Sei $A\in M(n,K)$ eine quadratische Matrix, dann ändert sich die Determinante bei elementaren Zeilenumformungen wie folgt:
	\begin{description}
		\item[EU 1] Beim Vertauschen zweier Zeilen multipliziert sich $\det A$ mit $(-1)$.
		\item[EU 2] Wird eine Zeile mit $\lambda \in K$ multipliziert, dann multipliziert sich die Determinante ebenfalls mit $\lambda$, d.h. man muss $\det A$ mit dem Kehrwert multiplizieren um das richtige Ergebnis zu erhalten.
		\item[EU 3] Wird ein Vielfaches einer Zeile zu einer anderen addiert, ändert sich der Wert der Determinante nicht.
	\end{description}
\end{satz}
\beweis
\begin{description}
	\item[EU 1] wegen Multilinearität und alternierend:
	\begin{align*}
		\underbrace{\det(\ldots,v+w,\ldots,v+w,\ldots)}_{=0} &= \det(\ldots,v,\ldots, v+w,\ldots)+\det(\ldots,w,\ldots, v+w,\ldots)\\
																&= \underbrace{\det(\ldots,v,\ldots, v,\ldots)}_{=0}+\det(\ldots,v,\ldots, w,\ldots)+\\
																&\quad +\det(\ldots,w,\ldots, v,\ldots)+\underbrace{\det(\ldots,w,\ldots, w,\ldots)}_{=0}\\
		\det(\ldots,v,\ldots, w,\ldots)		&=-\det(\ldots,w,\ldots, v,\ldots)
	\end{align*}
	\item[EU 2] folgt direkt aus der Multilineariät.
	\item[EU 3] wegen der Multilinearität:
	\begin{equation*}
		\det(\ldots,v,\ldots, w+\lambda*v,\ldots)=\det(\ldots,v,\ldots, w,\ldots)+\lambda*\underbrace{\det(\ldots,v,\ldots, v,\ldots)}_{=0}
	\end{equation*}
\end{description}

\bemerkung
Diese Eigenschaften genügen, um jede Determinante auszurechnen (mit dem Gaußalgorithmus). Entweder entsteht eine Nullzeile oder man formt um bis zur Einheitsmatrix.

\paragraph{Beispiel:}
\begin{equation*}
	\detmatrix{1 & 2\\3 & 4}=\detmatrix{1 & 2\\ 0 & -2}= -2\detmatrix{1 & \\ & 1}=-2=1*4-2*3
\end{equation*}

\begin{lemma}{Determinante von Matrizen in oberer Dreiecksgestalt}
	Für Diagonalmatrizen und allgemeiner, für obere Dreiecksmatrizen gilt:
	\begin{equation*}
		\det\matrix{\lambda_1 & \ast & \cdots & \ast \\&\lambda_2& \cdots & \ast\\&&\ddots&\vdots\\&&&\lambda_n}=\lambda_1*\lambda_2*\ldots*\lambda_n
	\end{equation*}
\end{lemma}
\beweis
Für die Diagonalmatrizen direkt aus der \textbf{(EU 2)} und der Normiertheit der Determinante.

Für die obere Dreiecksgestalt gilt, dass man sie durch \textbf{(EU 3)} auf Diagonalgestalt bringen kann falls alle Elemente ungleich Null sind. Dabei ändert sich nichts am Wert der Determinante.
Ist eines der Diagonalelemente Null, entsteht eine Nullzeile durch den Gaußalgorithmus $\rightsquigarrow \det A =0$.

\begin{satz}{Determinante und Invertierbarkeit}
	Die Determinante einer Matrix ist genau dann von Null verschieden, wenn die Matrix invertierbar ist.
\end{satz}

\beweis
Die Matrix ist genau dann invertierbar, wenn in einer Zeilen-Stufenform keine Nullzeilen vorkommen. Dies ist genau dann der Fall wenn die Determinante von Null verschieden ist.

\bemerkung
Aus dem Satz folgt die Eindeutigkeit der Determinante, denn wir können ihren Wert berechnen.

\begin{satz}{}
	Sie $A\in M(n,K)$, dann bezeichnet für $i,j\in\simpleset{1,\ldots,n}$ $A_{ij}$ die Matrix aus $M(n-1,K)$ die aus Streichen der $i$-ten Zeile und $j$-ten Spalte hervorgeht.
\end{satz}
\paragraph{Beispiel:}
\begin{equation*}
	A=\matrix{1&\color{red}2&3&4\\5&\color{red}6&7&8\\\color{red}9&\color{red}10&\color{red}11&\color{red}12\\13&\color{red}14&15&16}
	\rightsquigarrow A_{32}=\matrix{1&3&4\\5&7&8\\13&15&16}
\end{equation*}

\begin{satz}{La-Place'scher Entwicklungssatz}
	Sie $A\in M(n,K)$ und $j\in{1,\ldots, n}$, dann gilt:
	\begin{equation*}
		\det A=\sum\limits_{i=1}^n(-1)^{i+j}*a_{ji}*\det{A_{ij}}
	\end{equation*}
\end{satz}
\paragraph{Erläuterung:}
Dieses Verfahren wird auch Entwickeln nach der $j$-ten Spalte genannt.

\begin{equation*}
	A=\matrix{1&2&3\\4&5&6\\7&8&9}\quad\text{und }j=1 \text{ (Entwickeln nach der 1. Spalte)}
\end{equation*}

Den Faktor $(-1)^{i+j}$ können wir uns als schachbrettartiges Muster von Vorzeichen denken:
\begin{equation*}
	A=\matrix{+&-&+\\-&+&-\\+&-&+}
\end{equation*}

\begin{align*}
	\det A 	&= 1*\det A_{11}-4*\det A_{12}+7*\det A_{13}\\
					&= \detmatrix{5&6\\8&9}-4*\detmatrix{2&3\\8&9}+7*\detmatrix{2&3\\5&6}\\
					&= 5*9-6*8-4(2*9-3*8)+7(2*6-3*5)\\
					&= 45-48-4(18-24)+7(12-15)\\
					&= -3-4(-6)+7(-3)\\
					&= -3+24-21\\
					&= 0
\end{align*}
