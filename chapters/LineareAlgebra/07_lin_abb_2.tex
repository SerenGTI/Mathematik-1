Lineare Abbildungen sind wegen der Additivität Insbesondere Gruppenhomomorphismen bezüglich der Addition. Analog wie für Gruppenhomomorphismen gilt:

\begin{satz}{}
  Bild und Kern einer linearen Abbildung $f:V\rightarrow W$ sind jeweils Untervektorräume von $V$ bzw. $W$.
\end{satz}

\beweis
$\mathrm{Bild}(f)$ ist ein Untervektorraum von $W$:\\
Wegen $f(0)\in\mathrm{Bild}(f)$ ist $\mathrm{Bild}(f)\neq\emptyset$.\\
Seien $f(u),f(v)\in\mathrm{Bild}(f)$, dann gilt $f(u)+f(v)=f(u+v)\in\mathrm{Bild}(f)$.
Ebenso $\lambda f(v)=f(\lambda*v)\in\mathrm{Bild}(f) \quad \forall\lambda\in K$.

\bigskip

$\mathrm{Kern}(f)$ ist ein Untervektorraum von $V$:\\
Es gilt stets, $f(0)=0$ für jede lineare Abbildung, also ist $\mathrm{Kern}(f)\neq \emptyset$\\
Seien $u,v\in \mathrm{Kern}(f)$, dann folgt $f(u+v)=f(u)+f(v)=0+0=0=\in\mathrm{Kern}(f)$.
Und $f(\lambda *v)=\lambda f(v)=\lambda*0=0\in\mathrm{Kern}(f) \quad\forall \lambda\in K$


\begin{definition}{}
  Eine lineare Abbildung $f: V\rightarrow W$ heißt
  \begin{equation*}
    \text{Vektorraum-}
    \begin{cases}
      \text{Monomorphismus, falls $f$ injektiv ist}\\
      \text{Epimorphismus, falls $f$ surjektiv ist}\\
      \text{Isomorphismus, falls $f$ bijektiv ist}
    \end{cases}
  \end{equation*}

  Eine lineare Abbildung $V\rightarrow V$ heißt \emph{Endomorphismus}.
  Eine bijektive lineare Abbildung $V\rightarrow V$ heißt Vektorraum-\emph{Automorphismus}.


  \bemerkung
  Die Automorphismen $\mathrm{Aut}(V)$ eines Vektorraums $V$ bilden eine Gruppe mit der Verkettung als Verknüpfung.

  Die Menge der Endo- bzw. Automorphismen wird mit $\mathrm{End}(V)$ bzw. $\mathrm{Aut}(V)$ bezeichnet.

  Die Menge der linearen Abbildungen $V\rightarrow W$ mit $\mathrm{Hom}(V,W)$.
\end{definition}


\begin{definition}{Rang einer Abbildung}
  Die Dimension des Bildes einer linearen Abbildung $f$ heißt auch \emph{Rang} von $f$ (engl. rank).
  \begin{equation*}
    \rank{f} \coloneqq \dim{\ker f}
  \end{equation*}
\end{definition}

\begin{satz}{Dimensionsformel für lineare Abbildungen}
  Für lineare Abbildungen $f:V\rightarrow W$ gilt, falls $V$ endlich dimensional ist, die \emph{Dimensionsformel für lineare Abbildungen}:
  \begin{align*}
    \dim{V}&=\rank{f}+\dim{\ker f}\\
    &=\dim{\im f}+\dim{\ker f}
  \end{align*}
\end{satz}

\begin{lemma}{}
  Eine lineare Abbildung $f:V\rightarrow W$ ist genau dann injektiv, wenn ihr Kern trivial ist.
\end{lemma}
