\chapter{Verknüpfungen}
\begin{definition}{Verknüpfung}
	Sei $M$ eine Menge. Eine Abbildung $M\times M \rightarrow M, (a,b)\mapsto a\star b$ nennt man Verknüpfung.
\end{definition}

\begin{enumerate}
  \item Eine Verknüpfung heißt kommutativ, falls $a\star b = b\star a \quad\forall a,b\in M$ gilt.
  \item Sie heißt assoziativ, falls $a\star(b\star c)=(a\star b)\star c \quad\forall a,b,c\in M$ gilt.\\
  Man kann auch $a\star b\star c$ schreiben.
  \item Ein Element $e\in M$ heißt neutrales Element bezüglich der Verknüpfung $\star$,\\
  falls $a\star e = e\star a=a \quad\forall a\in M$ gilt.
\end{enumerate}

\begin{definition}{Invertierbarkeit}
	Sei $M$ eine Menge mit einer Verknüpfung $\star$, die ein neutrales Element $e$ besitzt, ein Element $a\in M$ heißt invertierbar, falls es ein Element $a^{-1}\in M$ gibt, so dass gilt:
	\begin{equation*}
	  a\star a^{-1} = a^{-1} \star a = e
	\end{equation*}
\end{definition}


\begin{definition}{Homomorphismus}
	Seien $(G,\star)$ und $(H,\ast)$ Gruppen. Eine Abbildung $f:G\rightarrow H$ heißt (Gruppen-)Homomorphismus, falls gilt:
	\begin{equation*}
	  f(a\star b)=f(a)\ast f(b)\quad\forall a,b\in G
	\end{equation*}
\end{definition}

\begin{lemma}{}
  Ein Gruppenhomomorphismus $f:G\rightarrow H$ bildet stets das neutrale Element in $G$ auf das neutrale Element in $H$ ab.
\end{lemma}
\beweis
Sei $e$ das neutrale Element in $G$, dann folgt:
\begin{equation*}
  f(e)\ast f(g)=f(e\star g)=f(g)
\end{equation*}
Es folgt dann, dass $f(e)$ das neutrale Element in $H$ ist.
