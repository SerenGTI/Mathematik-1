\chapter{Basiswechsel - Koordinatentransformation}

\paragraph{Erinnerung:}
Sei $V$ ein endlich dimensionaler Vektorraum und $B=\simpleset{v_1,\ldots,v_n}$ eine Basis von $V$. Dann hat jeder Vektor $v\in V$ eine Darstellung bezüglich $B$:
\begin{equation*}
	v=\lambda_1v_1+\ldots+\lambda_nv_n\in K
\end{equation*}
mit eindeutig bestimmten $\lambda_1, \ldots, \lambda_n$.\\
Außerdem ist der Koordinatenvektor von $v$ bezüglich der Basis $B$:
\begin{equation*}
	v_B=\vector{\lambda_1\\\vdots\\\lambda_n} \in K^n
\end{equation*}

Ist $C=\simpleset{w_1,\ldots,w_n}$ eine weitere Basis von $V$, dann hat $v$ im Allgemeinen verschiedene Darstellungen $v_B, v_C$.

\section{Transformationsmatrix}
\paragraph{Beispiel:}
Seien zwei Basen für den Vektorraum $V=\R$ gegeben:
\begin{equation*}
	B=\simpleset{\underset{v_1}{\vector{1\\0}},\underset{v_2}{\vector{1\\1}}}, C=\simpleset{\underset{w_1}{\vector{2\\0}},\underset{w_2}{\vector{1\\-1}}}
\end{equation*}

Dann lassen sich die Basisvektoren in $C$ durch die in $B$ ausdrücken:
\begin{align*}
	w_1&=2v_1\\
	w_2&=2v_1-v_2
\end{align*}
das heißt $w_1$ und $w_2$ haben bezüglich $B$ die Koordinatendarstellungen
\begin{equation*}
	w_{1_B}=\vector{2\\0}, w_{2_B}=\vector{2\\-1}
\end{equation*}

Wir schreiben diese Vektoren jetzt als Spalten in die \emph{Transformationsmatrix}
\begin{equation*}
	T_B^C=\matrix{2 & 2\\ 0 & -1} \quad\text{(Transformation von $C$ nach $B$)}
\end{equation*}
Das Anwenden dieser Matrix auf den Koordinatenvektor $v_C$ eines Vektors $v\in V$ liefert den Koordinatenvektor $v_b$ bezüglich der Basis $B$.

\begin{equation*}
	v_b=T_B^C*v_C
\end{equation*}

\begin{definition}{Transformationsmatrix}
	Seien $B=\simpleset{v_1,\ldots,v_n}$ und $C=\simpleset{w_1,\ldots,w_n}$ zwei Basen eines $K$-Vektorraums gegeben. Und die Matrix $T^B_C\in M(n,K)$ deren Spalten durch Koordinatendarstellungen der Vektoren $v_1,\ldots,v_n$ bezüglich der Basis $C$ gebildet werden, das heißt:
	\begin{equation*}
		T_C^B=\matrix{ |&&|\\ (v_1)_C & \cdots & (v_n)_C   \\ | && |}
	\end{equation*}
	diese heißt \emph{Transformationsmatrix} oder auch \emph{Basiswechselmatrix} von $B$ nach $C$.
\end{definition}


\section{Basiswechsel}
Basiswechsel bei einer darstellenden Matrix einer linearen Abbildung:\\
Sei $f:V\rightarrow W$ eine lineare Abbildung zwischen endlich dimensionalen $K$-Vektorräumen. Beim Übergang von einer Basis in $V$ oder in $W$ ändern sich nicht nur die Koordinatendarstellungen von einzelnen Vektoren, sondern auch die Einträge der darstellenden Matrix von $f$.

\begin{satz}{}
	Seien $V$ und $W$ endlich dimensionale $K$-Vektorräume. $B,C$ Basen von $V$ und $D,E$ Basen von $W$.
	Sei $f_D^B$ die darstellende Matrix einer linearen Abbildung $f:V\rightarrow W$ bezüglich der Basen $B$ und $D$.
	Dann gilt für die darstellende Matrix bezüglich $C$ und $E$:
	\begin{equation*}
		f_E^C=T_E^D*f^B_D*T_B^C
	\end{equation*}
\end{satz}

\begin{beweis}{}
	Sei $v\in V$:
	\begin{equation*}
		T_E^D*f^B_D* \underbrace{T_B^C*v_C}_{v_B}=T_E^D*f_D^B*v_B=T^D_E*\left(f(v)\right)_D = \left(f(_v)\right)_E
	\end{equation*}
\end{beweis}

\merkregel \glqq Kürzen \grqq:
\begin{equation*}
	T_E^D*f^B_D*T_B^{\not C}*v_{\not C}=T_E^D*f_D^{\not B}*v_{\not B}=T^{\not D}_E*\left(f(v)\right)_{\not D} = \left(f(_v)\right)_E
\end{equation*}


\paragraph{Problem:}
Wie findet man geeignete Transformationsmatrizen, um eine lineare Abbildung möglichst einfach darzustellen, idealerweise als eine Diagonalmatrix?
