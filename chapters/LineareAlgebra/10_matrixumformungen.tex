\chapter{Erweiterte Matrixrechnungen}
Um Gleichungssysteme systematisch zu lösen ist es zweckmäßg nicht die Gleichungen, sondern nur die Koeffizientenmatrix und die rechten Seiten zu betrachten.

\begin{definition}{Erweiterte Matrixschreibweise}
	Sei durch $A\in M(m,n,K)$ und $b\in K^m$ das lineare Gleichungssystem $A*x=b$ gegeben, dann ist
	\begin{equation*}
		\ematrix{c|c}{A & b}\coloneqq\ematrix{c c c | c}{
			\ma{11} & \cdots & \ma{1n} & b_1\\
			\vdots & \ddots & \vdots & \vdots\\
			\ma{m1} & \cdots & \ma{mn} & b_m
		}
	\end{equation*}
	die zum System gehörende \emph{erweiterte Matrix}.
\end{definition}

\section{Elementare Zeilenoperationen}
Die folgenden sogenannten \emph{elementaren Zeilenumformungen} ändern nichts an der Lösungsmenge des linearen Gleichungssystems $A*x=b$, wenn sie an der erweiterten Matrix $\ematrix{c|c}{A & b}$ vorgenommen werden.

\begin{description}
	\item[EU1] Vertauschen zweier Zeilen
	\item[EU2] Multiplikation einer Zeile mit einem Skalar ungleich $0$
	\item[EU2] Adition eines Vielfachen einer Zeile zu einer anderen
\end{description}

\begin{definition}{Zeilen-Stufenform}
	Eine Matrix $A\in M(m,n,K)$ liegt in \emph{Zeilen-Stufenform} vor, falls es ein $k\in \simpleset{0,\ldots, m}$ gibt, so dass gilt:
	\begin{itemize}
		\item Die ersten $k$ Zeilen sind von $0$ verschieden und der Spaltenindex des am weitesten links stehenden, von $0$ verschiedenen Eintrags erhöht sich jeweils um mindestens $1$ beim Übergang von einer Zeile zur darunterliegenden innerhalb der ersten $k$ Zeilen.
		\item Die unteren $m-k$ Zeilen sind alle Nullzeilen.
	\end{itemize}
\end{definition}


\begin{satz}{}
	Jede Matrix lässt sich mit endlich vielen elementaren Zeilenumformungen auf Zeilen-Stufenform bringen.
\end{satz}
\begin{beweis}
	\textit{Das hier beschriebene Verfahren ist der sogenannte Gauß-Jordan'sche-Eliminationsalgorithmus!}
	\begin{enumerate}
		\item Sortiere die Zeilen nach dem Auftreten des am weitesten links stehenden von Null verschiedenen Element. Nullzeilen unten einsortieren.
		\item Führe dann Umformungen durch
	\end{enumerate}
\end{beweis}


\begin{definition}{Pivotelemente}
	Die \emph{Pivotelemente} einer Matrix in Zeilen-Stufenform sind die in ihrer Zeile am weitesten links stehenden von Null verschiedenen Elemente, die nicht in einer Nullzeile stehen. Die \emph{Pivotvariablen} sind die zugehörigen Variablen. $x_j$ ist eine Pivotvariable genau dann, wenn in der $j$-ten Spalte von $A$ ein Pivotelement steht.
\end{definition}

Die Anzahl der Pivotvariablen ist gleich $k$ (s.o.).
