\chapter{Gruppen, Ringe und Körper}
\definition{Verknüpfung}
Sei $M$ eine Menge. Eine Abbildung $M\times M \rightarrow M, (a,b)\mapsto a\star b$ nennt man Verknüpfung.

\begin{enumerate}
  \item Eine Verknüpfung heißt kommutativ, falls $a\star b = b\star a \quad\forall a,b\in M$ gilt.
  \item Sie heißt assoziativ, falls $a\star(b\star c)=(a\star b)\star c = \quad\forall a,b,c\in M$ gilt.\\
  Man kann auch $a\star b\star c$ schreiben.
  \item Ein Element $e\in M$ heißt neutrales Element bezüglich der Verknüpfung $\star$,\\
  falls $a\star e = e\star a=a \quad\forall a\in M$ gilt.
\end{enumerate}

\definition{Invertierbarkeit}
Sei $M$ eine Menge mit einer Verknüpfung $\star$, die ein neutrales Element $e$ besitzt, ein Element $a\in M$ heißt invertierbar, falls es ein Element $a^{-1}\in M$ gibt, so dass gilt:
\begin{equation*}
  a\star a^{-1} = a^{-1} \star a = e
\end{equation*}

\definition{Gruppe}
Eine Menge $G$ mit einer Verknüpfung $\star$ heißt \emph{Gruppe}, falls
\begin{description}
  \item[G1] Die Verknüpfung assoziativ ist,\label{gruppe1}
  \item[G2] ein neutrales Element besitzt,
  \item[G3] jedes Element invertierbar ist.\label{gruppe3}
\end{description}
Falls die Verknüpfung zusätzlich kommutativ ist, nennt man die Gruppe eine \emph{abel'sche Gruppe} oder auch kommutative Gruppe.

\definition{Ring}
Sei $M$ eine Menge mit zwei Verknüpfungen $(+,\cdot)$ und den folgenden Eigenschaften:
\begin{description}
  \item[R1] $(M,+)$ ist eine abel'sche Gruppe mit neutralem Element $0$.
  \item[R2] die Verknüpfung $\cdot$ ist assoziativ mit neutralem Element $1$.
  \item[R3] es gelten die Distributivgesetze:
  \begin{align*}
    (a+b)\cdot c&=ac+bc\\
    c\cdot(a+b)&=ca+cb
  \end{align*}
  \item[R4] $0\neq 1$
\end{description}
Dan heißt $M$ ein \emph{Ring} (genauer ein Ring mit Eins - unitärer Ring).

\vspace{1em}
Ist zusätzlich auch die Multiplikation $\cdot$ kommutativ und ist $M\setminus\{0\}$ eine Gruppe bezüglich $\cdot$ (d.h. besitzt jedes Element ein Inverses bzgl. $\cdot$) so heißt $M$ \emph{Körper}.

\satz{Eindeutigkeit der neutralen Elemente}
In einer Gruppe ist das neutrale Element stats eindeutig, d.h. ist $e$ ein neutrales Element und gibt es ein Element:
\begin{equation*}
  a\in G, \forall g\in G : a\star g = g\star a = g
\end{equation*}
Dann ist $a = e$!

\beweis
Gelte $a\star g = g$ für ein $g\in G$. Dann folgt:
\begin{equation*}
  (a\star g)\star g^{-1}=g\star g^{-1}
\end{equation*}
Mit \textbf{G1} und \textbf{G3} gilt:
\begin{equation*}
  a\star (g\star g^{-1})=e
\end{equation*}
Dann folgt mit \textbf{G3}:
\begin{equation*}
  a\star e=e \text{ und damit } a=e \hfill\Box
\end{equation*}

\bemerkung
Ähnlich dazu der Beweis, dass inverse Elemente eindeutig bestimmt sind.


\definition{Homomorphismus}
Seien $(G,\star)$ und $(H,\ast)$ Gruppen. Eine Abbildung $f:G\rightarrow H$ heißt (Gruppen-)Homomorphismus, falls gilt:
\begin{equation*}
  f(a\star b)=f(a)\ast f(b)\quad\forall a,b\in G
\end{equation*}

\lemma{}
Ein Gruppenhomomorphismus $f:G\rightarrow H$ bildet stets das neutrale Element in $G$ auf das neutrale Element in $H$ ab.
\beweis
Sei $e$ das neutrale Element in $G$, dann folgt:
\begin{equation*}
  f(e)\ast f(g)=f(e\star g)=f(g)
\end{equation*}
Es folgt dann, dass $f(e)$ das neutrale Element in $H$ ist.


\definition{}
Sei $G$ eine Gruppe mit Verknüpfung $\star$ und neutralem Element $e$.\\
Eine nichtleere Teilmenge $U\subseteq G$ heißt \emph{Untergruppe} von $G$, falls gilt:
\begin{description}
  \item[UG 1] $\forall a,b\in U : a\star b\in U$ (Abgeschlossenheit)
  \item[UG 2] $\forall a\in U : a^{-1}\in U$
\end{description}

Immer gilt, dass der Kern eines Homomorphismus $f:G \rightarrow H$ d.h. $\mathrm{Kern}(f)=f^{-1}(\{e\})$  eine Untergruppe von $G$ ist.
