\chapter{Mengen, Relationen und Abbildungen}

\section{Mengen}
Eine Menge ist eine wohldefinierte Gesamtheit von Objekten, den Elementen der Menge.
\begin{equation*}
  \text{z.B. }\Q=\set{\frac{p}{q}}{p\in\Z, q\in\N}
\end{equation*}

\definition{Teilmenge}
Eine Menge $M_1$ ist \emph{Teilmenge} von $M$, wenn
\begin{align*}
  &\forall x\in M_1 : x\in M\\
  &\Rightarrow M_1 \subseteq M
\end{align*}
Für jede Menge M gilt $\emptyset \subseteq M$ und $M\subseteq M$.\\
Gilt $M_1\subseteq M$ und $M_1\neq M$ ist $M_1$ eine \emph{echte Teilmenge} von $M$, d.h. $M_1 \subset M$ oder $M_1\subsetneq M$

\paragraph{Potenzmenge}
\begin{equation*}
  \P(M)=\Pot(M) \text{ ist die Menge aller Teilmengen von } M.
\end{equation*}
\paragraph{Schnittmenge}
\begin{equation*}
  M_s = M_1 \cap M_2; \quad M_s \coloneqq \set{m\in M_1}{m\in M_2}
\end{equation*}
Zwei Mengen $M_1$ und $M_2$ heißen \emph{disjunkt}, falls $M_1\cap M_2 = \emptyset$
\paragraph{Vereinigung}
\begin{equation*}
  M_v = M_1 \cup M_2; \quad M_v \coloneqq \set{m}{m\in M_2 \vee m\in M_2}
\end{equation*}
\paragraph{Differenz}
\begin{equation*}
  M_1\setminus M_2 \coloneqq \set{m\in M_1}{m\not\in M_2}
\end{equation*}
\paragraph{Kartesisches Produkt}
\begin{equation*}
  M_1 \times M_2 \coloneqq \set{(m_1, m_2)}{m_1\in M_1 \wedge m_2\in M_2}
\end{equation*}

\section{Relationen}
\definition{Relation}
Eine Relation zwischen zwei Mengen $M$ und $N$ ist eine Teilmenge von $M\times N$.
\begin{equation*}
  R\subseteq M_1\times M_2
\end{equation*}
ist $(x, y) \in R$, steht $x$ mit $y$ in Relation $\rightarrow x\sim y$.

$R \subseteq M\times M$ heißt
\begin{description}
  \item[reflexiv], falls $\forall x\in M : (x,x)\in R$
  \item[symmetrisch], falls $\forall x,y \in M : (x,y)\in M \Rightarrow (y,x) \in R$
  \item[antisymmetrisch], falls $\forall x,y \in R : (x,y)\in M \wedge (y,x)\in R \Rightarrow x=y$
  \item[transitiv], falls $\forall x,y,z \in M : (x,y)\in R \wedge (y,z)\in R \Rightarrow (x,z)\in R$
\end{description}

\definition{Äquivalenzrelation}
Eine Relation heißt Äquivalenzrelation, wenn sie reflexiv, symmetrisch und transitiv ist.

\definition{Ordnungsrelation}
Eine Relation heißt Ordnungsrelation, wenn sie reflexiv, antisymmetrisch und transitiv ist.

\section{Abbildungen}
\definition{Abbildung}
Seien $M$ und $N$ zwei Mengen. Eine Zuordnungsvorschift, die jedem Element $x\in M$ ein Element $f(x)\in N$ zuweist, heißt Abbildung oder Funktion von $M$ nach $N$.
\begin{equation*}
  f:M\rightarrow N, x\mapsto f(x)
\end{equation*}
$M$: Definitionsbereich, $N$: Wertebereich

\definition{}
Sei $f:M\mapsto N$ eine Abbildung. Wir definieren
\begin{itemize}
  \item für $x\in M$ heißt $f(x)\in N$ das \emph{Bild} von $x$
  \item für eine Teilmenge $A\subseteq M$ heißt $f(A)=\set{f(x)}{x\in A}$ das \emph{Bild der Teilmenge  $A$}
  \item für eine Teilmenge $B\subseteq N$ heißt $f^{-1}(B) = \set{x\in M}{f(x)\in B}$ das \emph{Urbild} von $B$
\end{itemize}

\definition{Abbildungseigenschaften}
Sei $f:M\rightarrow N$ eine Abbildung. Dann heißt $f$:
\begin{description}
  \item[injektiv], wenn jedes Element $y\in N$ \emph{höchstens ein Urbild} hat.
  \item[surjektiv], wenn jedes Element $y\in N$ \emph{mindestens ein Urbild} hat. $\forall y\in N\; \exists x\in M : f(x) = y$
  \item[bijektiv], wenn jedes Element $y\in N$ \emph{genau ein Urbild} hat. $\forall y\in N\; \exists! x\in M : f(x) = y$
\end{description}

\bemerkung
\begin{enumerate}
  \item Bijektivität gilt genau dann, wenn es eine Umkehrabbildung $f^{-1}$ gibt:
  \begin{align*}
    f:M\rightarrow N && f^{-1}:N\rightarrow M\\
    f\left(f^{-1}(x)\right) \quad\text{mit }x\in N && f^{-1}\left(f(x)\right)=x \quad\text{mit }x\in M
  \end{align*}
  \item Man kann jede Abbildung surjektiv machen, indem man den Wertebereich durch das Bild von $f$ ersetzt: $N\coloneqq f(M)$
\end{enumerate}

\section{Mächtigkeit von Mengen}
Die Mächtigkeit einer Menge ist die Anzahl ihrer Elemente. Man schreibt $|M|$ für die Mächtigkeit von $M$.

Zwei Mengen $A$ und $B$ sind gleich mächtig, wenn es eine bijektive Abbildung $f:A\rightarrow B$ gibt.

Eine Menge heißt \emph{abzählbar unendlich}, falls $|A|=|\N|$ d.h. falls es eine bijektive Abbildung $f:A\rightarrow \N$ gibt.

Sie heißt \emph{überabzählbar unendlich}, falls $|A|>|\N|$.

Es gilt immer auch für unendliche Mengen, dass $|M| < |\Pot(M)|$.

Für endliche Mengen gilt $|\Pot(M)| = 2^{|M|}$



\section{Zahlenmengen}
\definition{Natürliche Zahlen}
Die natürlichen Zahlen sind eine Menge $\N$, auf der eine Abbildung $f:\N\rightarrow\N$ erklärt ist, die folgende Eigenschaften hat, wobei $f(n)$ der \emph{Nachfolger} von $n$ heißt.
\begin{description}
  \item[$\N 1$] Es gibt genau ein Element in $\N$, das nicht Nachfolger eines anderen Elements ist.
  \item[$\N 2$] $f$ ist injektiv
  \item[$\N 3$] Ist $M\subseteq \N$ eine Teilmenge, die folgende Eigenschaften hat:
  \begin{enumerate}
    \item $1\in M$
    \item Falls $m\in M$ und $f(m)\in M$
  \end{enumerate}
  Dann gilt: $M = \N$

  D.h. $M\subseteq \N : 1\in M \wedge (m\in M \Rightarrow f(m)\in M) \Rightarrow M=\N$
\end{description}

Man kann zeigen, dass die natürlichen Zahlen durch diese Eigenschaften (die \textsc{Peano}-Axiome) gekennzeichnet sind. Das heißt, dass es im wesentlichen nur eine solche Menge mit einer solchen Abbildung $f$ gibt, nämlich $\N$.

Das Axiom $\N 3$ heißt auch Induktionsaxiom. Aus ihm folgt:

\begin{satz}{Vollständige Induktion}
  Sei $A(n)$ für jede natürliche Zahl $n \in\N$ eine Aussage, für die gilt:
  \begin{itemize}
    \item $A(1)$ ist wahr
    \item $\forall n\in\N : A(n) \Rightarrow A(n+1)$
  \end{itemize}
  dann ist $A(n)$ für alle $n\in\N$ wahr.
\end{satz}


\definition{Graph einer Abbildung}
Sei $f:M\rightarrow N$ eine Abbildung. Der Graph von $f$ ist eine Teilmenge $\set{(x,f(x))}{x\in M}\subseteq M\times N$.
Für Funktionen $f:\R\rightarrow\R$ ist der Graph eine Teilmenge der Ebene $\R^2$.

Fasst man eine Funktion als eine Relation auf, so ist der Graph das selbe wie R. $\mathrm{Graph}(f)=R\subseteq\R\times\R$

\definition{Verkettung}
Seien $f:M\rightarrow N$ und $g:N\rightarrow P$ Abbildungen. Dann ist die Verkettung:
\begin{align*}
  &g\circ f:M\rightarrow P\\
  &g\circ f(x)\coloneqq g(f(x))
\end{align*}

\definition{Identität}
Für jede Menge $M$ ist
\begin{equation*}
  \mathrm{id}_M:M\rightarrow M, x\mapsto x
\end{equation*}
die identische Abbildung auf $M$.

\section{Summen- und Produktzeichen}
\begin{equation*}
  \sum\limits_{k=m}^n a_k\coloneqq a_m + a_{m+1} + \ldots + a_n
\end{equation*}
Bei der Summe ist $k$ der Summationsindex, $m$ die untere und $n$ die obere Summationsgrenze
\begin{equation*}
  \prod\limits_{k=m}^n a_k\coloneqq a_m * a_{m+1} * \ldots * a_n
\end{equation*}

\bemerkung
\begin{itemize}
  \item Ist die obere Summationsgrenze kleiner als die untere, so handelt es sich um eine \emph{leere Summe}, ihr Wert ist 0.
  \item Entsprechend ist der Wert des \emph{leeren Produkts} 1.
\end{itemize}

\section{Teilbarkeit und Primzahlen}
\definition{Teilbarkeit}
Seien $n\in\Z, m\in\N$. Die Zahl $m$ heißt \emph{ein Teiler} von $n$, in Zeichen $k* m=n$, wenn es ein $k\in\Z$ gibt, so dass $k* m = n$. In diesem Fall heißt $n$ auch teilbar durch $m$.
Die Zahl $0$ ist durch alle $m\in\Z$ teilbar.

Falls $m|n_1$ und $m|n_2$, dann folgt $m|n_1+n_2$.

\definition{Größter gemeinsamer Teiler}
Sei $a\in\Z$, die Menge aller Teiler von $a$ ist $\mathcal{D}(a)\coloneqq\set{d\in\N}{d|a}$.

Die Menge aller gemeinsamer Teiler von $a$ und $b$ mit $a,b\in\Z\setminus\{ 0\}$ ist $\mathcal{D}(a,b) = \mathcal{D}(a) \cap \mathcal{D}(b)$.

Die Zahl $\mathrm{ggT}(a,b) = \mathrm{max}(\mathcal{D}(a,b))$ heißt größter gemeinsamer Teiler von $a$ und $b$. Da eine ganze Zahl (außer der $0$) nur endlich viele Teiler hat, existiert $ggT(a,b)$.


\definition{Primzahl}
Eine natürliche Zahl heißt \emph{Primzahl}, wenn sie genau zwei Teiler besitzt, nämlich 1 und die Zahl selbst.
\begin{equation*}
  p\in\N \text{ mit } |\mathcal{D}(p)| = 2
\end{equation*}

\begin{satz}{Primfaktorzerlegung}
  Jede natürliche Zahl $n\in\N \wedge n\geq2$ ist ein Produkt aus Primzahlen ($1$ ist das leere Produkt).
\end{satz}

\beweis
$A(n)$ : \glqq Jede natürliche Zahl kleiner oder gleich $n$ ist das Produkt von Primzahlen.\grqq

\begin{description}
  \item[IA] $A(2)$ ist wahr, denn $2$ ist selbst eine Primzahl.
  \item[IS] Fallunterscheidung:
  \begin{enumerate}
    \item $n+1$ ist prim. Dann ist $A(n+1)$ wahr.
    \item $n+1$ ist nicht prim. Dann gibt es natürliche Zahlen $1$ und $m$, sodass $n+1=l* m$, wobei $l,m<n+1$.
  \end{enumerate}
  Nach Induktionsvoraussetzung sind somit $l$ und $m$ Produkte von Primzahlen, somit auch $n+1$.
\end{description}
