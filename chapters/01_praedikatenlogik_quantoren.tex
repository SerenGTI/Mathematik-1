\chapter{Grundlagen}
\section{Logik}
\definition{Aussage}
Eine Aussage ist ein Satz, von dem es Sinn macht, zu fragen, ob er wahr oder falsch ist.

\subsection{Logische Junktoren}
Wir verknüpfen mehrere Aussagen zu größeren aussagelogischen Formeln mithilfe von logischen Junktoren:
\begin{multicols}{3}
  \paragraph{Negation:}
  $\neg A$
  \columnbreak
  \paragraph{Konjuktion:}
  $A \wedge B$
  \columnbreak
  \paragraph{Disjunktion:}
  $A \vee B$
\end{multicols}
Durch verwenden dieser grundlegenden Junkoren kann man alle Verknüpfungen darstellen. Um Schreibarbeit zu sparen gibt es verkürzende Schreibweisen

\paragraph{Implikation:}
$A\Rightarrow B \equiv \neg(A\wedge \neg B)$
\paragraph{Äquivalenz:}
$A\Leftrightarrow B \equiv (A\wedge B)\vee (\neg A\wedge \neg B)$

\vspace{1em}
\begin{center}
  \begin{tabular}{c|c||c|c|c|c|c}
    $A$ & $B$ & $\neg A$ & $A \wedge B$ & $A \vee B$ & $A \Rightarrow B$ & $A \Leftrightarrow B$\\
    \hline  f & f & w & f & f & w & w \\
            f & w & w & f & w & w & f \\
            w & f & f & f & w & f & f \\
            w & w & f & w & w & w & w
  \end{tabular}
\end{center}



\section{Prädikatenlogik und Quantoren}
Ein Prädikat ist ein Ausdruck, der die Form einer Aussage hat, aber Variablen enthält.

$P(m)\coloneqq$\glqq m ist eine gerade Zahl. \grqq

\noindent
Eine Aussage wird daraus erst, wenn wir angeben, für welche $m$ das Prädikat gelten soll.

Sei $M$ eine Menge und $P(m)$ für jedes $m\in M$ eine Aussage. Wir beschreiben die Aussage mit dem \emph{Allquantor}:
\begin{equation*}
  \forall m\in M: P(m)
\end{equation*}
d.h. $P(m)$ soll für \emph{jedes} $m\in M$ gelten.

Mit dem \emph{Existenzquantor} bekommt das Prädikat eine andere Bedeutung:
\begin{equation*}
  \exists m\in M: P(m)
\end{equation*}
d.h. es soll mindestens ein $m\in M$ existieren, für das $P(m)$ gilt.

\paragraph{Beispiel}
$M=\N, P(m)$: \glqq $m$ ist eine gerade Zahl.\grqq

$(\forall m\in M: P(m))$ ist falsch. $(\exists m\in M: P(m))$ ist jedoch wahr.

\bemerkung
\begin{itemize}
  \item Verneinung von quantifizieren Prädikat-Aussagen:
  \glqq Prädikat verneinen und Quantoren tauschen.\grqq
  \begin{equation*}
    \neg(\forall m\in M: P(m)) \equiv  \exists m\in M: \neg P(m)
  \end{equation*}
  \item Bei Quantoren kommt es auf die Reihenfolge an:
  \begin{align*}
    \forall n\in\N \quad\exists m\in \N &: m\geq n \quad\text{ist wahr}\\
    \exists n\in\N \quad\forall m\in \N &: m\geq n \quad\text{ist falsch}\\
  \end{align*}
\end{itemize}


\section{Beweise}
Wir wollen eine Aussage $A\Rightarrow B$ beweisen. Dazu gibt es mehrere Ansätze, diese werden am Beispiel gezeigt:
\begin{align*}
  A &\equiv |x-1|<1\\
  B &\equiv x<2
\end{align*}
\subsection{Direkter Beweis}
$A$ wird als wahr angenommen, und daraus muss $B\equiv x<2$ gefolgert werden.

Fallunterscheidung:
\begin{itemize}
  \item $(x-1)\geq0\leadsto x-1<1 \Leftrightarrow x<2$
  \item $(x-1)<0\leadsto x<1 \hfill\Box$
\end{itemize}
\subsection{Indirekter Beweis (Kontraposition)}
Wir zeigen, dass $\neg B\Rightarrow \neg A$.
Gelte also $\neg B$:
\begin{equation*}
  x\geq2 \leadsto |x-1|=x-1\geq 1 \Leftrightarrow x\geq 2
\end{equation*}
\subsection{Widerspruchsbeweis}
Wir zeigen, dass $\neg(A\Rightarrow B)$ bzw. $A\wedge \neg B$ auf einen Widerspruch führt.
Angenommen, es gelte $|x-1|<1$ und $x\geq 2$ daraus folgt:
\begin{equation*}
  |x-1|=x-1<1\Leftrightarrow x<2 \text{ Widerspruch!}
\end{equation*}
