\chapter{Prädikatenlogik und Quantoren}
Ein Prädikat ist ein Ausdruck, der die Form einer Aussage hat, aber Variablen enthält.

$P(m)\coloneqq$\glqq m ist eine gerade Zahl. \grqq

\noindent
Eine Aussage wird daraus erst, wenn wir angeben, für welche m das Prädikat gelten soll.

Sei $M$ eine Menge und $P(m)$ für jedes $m\in M$ eine Aussage. Wir beschreiben die Aussage mit dem Allquantor:

\begin{equation}
  \forall m\in M: P(m)
\end{equation}
