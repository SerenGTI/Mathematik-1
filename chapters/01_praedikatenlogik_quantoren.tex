\chapter{Grundlagen}
\section{Logik}
\definition{Aussage}
Eine Aussage ist ein Satz, von dem es Sinn macht, zu fragen, ob er wahr oder falsch ist.

\subsection{Logische Junktoren}
Wir verknüpfen mehrere Aussagen zu größeren aussagelogischen Formeln mithilfe von logischen Junktoren:
\begin{multicols}{3}
  \paragraph{Negation:}
  $\neg A$
  \columnbreak
  \paragraph{Konjuktion:}
  $A \wedge B$
  \columnbreak
  \paragraph{Disjunktion:}
  $A \vee B$
\end{multicols}
Durch verwenden dieser grundlegenden Junkoren kann man alle Verknüpfungen darstellen. Um Schreibarbeit zu sparen gibt es verkürzende Schreibweisen

\paragraph{Implikation:}
$A\Rightarrow B \equiv \neg(A\wedge \neg B)$
\paragraph{Äquivalenz:}
$A\Leftrightarrow B \equiv (A\wedge B)\vee (\neg A\wedge \neg B)$

\vspace{1em}
\begin{center}
  \begin{tabular}{c|c||c|c|c|c|c}
    $A$ & $B$ & $\neg A$ & $A \wedge B$ & $A \vee B$ & $A \Rightarrow B$ & $A \Leftrightarrow B$\\
    \hline  f & f & w & f & f & w & w \\
            f & w & w & f & w & w & f \\
            w & f & f & f & w & f & f \\
            w & w & f & w & w & w & w
  \end{tabular}
\end{center}



\section{Prädikatenlogik und Quantoren}
Ein Prädikat ist ein Ausdruck, der die Form einer Aussage hat, aber Variablen enthält.
Eine Aussage wird daraus erst, wenn wir angeben, für welche $m$ das Prädikat gelten soll.

Sei $M$ eine Menge und $P(m)$ für jedes $m\in M$ eine Aussage. Wir beschreiben die Aussage mit dem \emph{Allquantor}:
\begin{equation*}
  \forall m\in M: P(m)
\end{equation*}
d.h. $P(m)$ soll für \emph{jedes} $m\in M$ gelten.
\par\medskip
Mit dem \emph{Existenzquantor} bekommt das Prädikat eine andere Bedeutung:
\begin{equation*}
  \exists m\in M: P(m)
\end{equation*}
d.h. es soll mindestens ein $m\in M$ existieren, für das $P(m)$ gilt.

\paragraph{Beispiel}
$M=\N, P(m)$: \glqq $m$ ist eine gerade Zahl.\grqq

$(\forall m\in M: P(m))$ ist falsch. \\
$(\exists m\in M: P(m))$ ist jedoch wahr.

\subsection{Verneinung von Aussagen}
Verneinung von quantifizieren Prädikat-Aussagen:
\glqq Prädikat verneinen und Quantoren tauschen.\grqq
\begin{equation*}
  \neg(\forall m\in M: P(m)) \equiv  \exists m\in M: \neg P(m)
\end{equation*}
\subsection{Reihenfolge der Quantoren}
Bei Quantoren kommt es auf die Reihenfolge an:
\begin{align*}
  \forall n\in\N \quad\exists m\in \N &: m\geq n \quad\text{ist wahr}\\
  \exists n\in\N \quad\forall m\in \N &: m\geq n \quad\text{ist falsch}\\
\end{align*}


\section{Beweise}
Wir wollen eine Aussage $A\Rightarrow B$ beweisen. Dazu gibt es mehrere Ansätze, diese werden am Beispiel gezeigt:
\begin{align*}
  A &\equiv |x-1|<1\\
  B &\equiv x<2
\end{align*}
\subsection{Direkter Beweis}
$A$ wird als wahr angenommen, und daraus muss $B\equiv x<2$ gefolgert werden.

Fallunterscheidung:
\begin{itemize}
  \item $(x-1)\geq0\leadsto x-1<1 \Leftrightarrow x<2$
  \item $(x-1)<0\leadsto x<1 \hfill\Box$
\end{itemize}
\subsection{Indirekter Beweis (Kontraposition)}
Wir zeigen, dass $\neg B\Rightarrow \neg A$.
Gelte also $\neg B$:
\begin{equation*}
  x\geq2 \leadsto |x-1|=x-1\geq 1 \Leftrightarrow x\geq 2
\end{equation*}
\subsection{Widerspruchsbeweis}
Wir zeigen, dass $\neg(A\Rightarrow B)$ bzw. $A\wedge \neg B$ auf einen Widerspruch führt.
Angenommen, es gelte $|x-1|<1$ und $x\geq 2$ daraus folgt:
\begin{equation*}
  |x-1|=x-1<1\Leftrightarrow x<2 \text{ Widerspruch!}
\end{equation*}


\section{Summen- und Produktzeichen}
\begin{equation*}
  \sum\limits_{k=m}^n a_k\coloneqq a_m + a_{m+1} + \ldots + a_n
\end{equation*}
Bei der Summe ist $k$ der Summationsindex, $m$ die untere und $n$ die obere Summationsgrenze
\begin{equation*}
  \prod\limits_{k=m}^n a_k\coloneqq a_m * a_{m+1} * \ldots * a_n
\end{equation*}

\bemerkung
\begin{itemize}
  \item Ist die obere Summationsgrenze kleiner als die untere, so handelt es sich um eine \emph{leere Summe}, ihr Wert ist 0.
  \item Entsprechend ist der Wert des \emph{leeren Produkts} 1.
\end{itemize}


\section{Teilbarkeit und Primzahlen}
\definition{Teilbarkeit}
Seien $n\in\Z, m\in\N$. Die Zahl $m$ heißt \emph{ein Teiler} von $n$, in Zeichen $k* m=n$, wenn es ein $k\in\Z$ gibt, so dass $k* m = n$. In diesem Fall heißt $n$ auch teilbar durch $m$.
Die Zahl $0$ ist durch alle $m\in\Z$ teilbar.

Falls $m|n_1$ und $m|n_2$, dann folgt $m|n_1+n_2$.

\definition{Größter gemeinsamer Teiler}
Sei $a\in\Z$, die Menge aller Teiler von $a$ ist $\mathcal{D}(a)\coloneqq\set{d\in\N}{d|a}$.

Die Menge aller gemeinsamer Teiler von $a$ und $b$ mit $a,b\in\Z\setminus\{ 0\}$ ist $\mathcal{D}(a,b) = \mathcal{D}(a) \cap \mathcal{D}(b)$.

Die Zahl $\mathrm{ggT}(a,b) = \mathrm{max}(\mathcal{D}(a,b))$ heißt größter gemeinsamer Teiler von $a$ und $b$. Da eine ganze Zahl (außer der $0$) nur endlich viele Teiler hat, existiert $ggT(a,b)$.

\begin{satz}{Teilung mit Rest}
  Seien $a,b\in\N$ mit $a>b$. Dann gibt es Zahlen $q\in\N, r\in\N_0$ mit
  \begin{align*}
    &0\leq r<B \quad\text{Rest kleiner als der Teiler}\\
    &a=q*b+r
  \end{align*}
\end{satz}
Mit diesem Satz folgt das Lemma, auf dem der \emph{Euklidische Algorithmus} basiert:
\begin{lemma}{}
  Seien $a,b,q,r\in\N$, so dass $a=q*b+r$. Dann gilt
  \begin{equation*}
    \mathcal{D}(a,b)=\mathcal{D}(b,r)
  \end{equation*}
  Insbesondere gilt:
  \begin{equation*}
    \mathrm{ggT}(a,b)=\mathrm{ggT}(b,r)
  \end{equation*}
\end{lemma}
\beweis
Wir beweisen die Gleichheit der beiden Mengen, indem wir die beiden Inklusionen nachweisen:

\begin{description}
  \item[\glqq$\subseteq$\grqq]
  Sei $d\in\mathcal{D}(a,b)$ d.h. $d|a \wedge d|b$. Wegen $a=q*b+r \Leftrightarrow r=a-q*b$ folgt, dass $d$ auch $r$ teilt.\\
  Es folgt also $d\in \mathcal{D}(b,r)$.
  \item[\glqq$\supseteq$\grqq]
  Sei $d\in\mathcal{D}(b,r)$ d.h. $d|b\wedge d|r$, dann folgt aus $a=q*b+r$, dass $d$ auch $a$ teilt, womit $d\in \mathcal{D}(a,b)$ folgt.
\end{description}
$\mathcal{D}(a,b)=\mathcal{D}(b,r)$

\par\medskip
Dieses Lemma liefert die Idee für einen Algorithmus zur Bestimmung des größten gemeinsamen Teilers zweier natürlicher Zahlen.

Sei $a>b$. Teilt $b$ die Zahl $a$ ohne Rest, so ist $b$ der $\mathrm{ggT}(a,b)$. Ansonsten ermittle den Rest bei der Teilung von $a$ durch $b$ und suche statt $\mathrm{ggT}(a,b)$ den $\mathrm{ggT}(b,r)$.

Nach dem Satz zur Teilung mit Rest sind $b$ und $r$ beide kleiner als $a$, also kommt das Verfahren nach endlich vielen Schritten zum Ende.

\definition{Primzahl}
Eine natürliche Zahl heißt \emph{Primzahl}, wenn sie genau zwei Teiler besitzt, nämlich 1 und die Zahl selbst.
\begin{equation*}
  p\in\N \text{ mit } |\mathcal{D}(p)| = 2
\end{equation*}

\begin{satz}{Primfaktorzerlegung}
  Jede natürliche Zahl $n\in\N \wedge n\geq2$ ist ein Produkt aus Primzahlen ($1$ ist das leere Produkt).
\end{satz}

\beweis
$A(n)$ : \glqq Jede natürliche Zahl kleiner oder gleich $n$ ist das Produkt von Primzahlen.\grqq

\begin{description}
  \item[IA] $A(2)$ ist wahr, denn $2$ ist selbst eine Primzahl.
  \item[IS] Fallunterscheidung:
  \begin{enumerate}
    \item $n+1$ ist prim. Dann ist $A(n+1)$ wahr.
    \item $n+1$ ist nicht prim. Dann gibt es natürliche Zahlen $1$ und $m$, sodass $n+1=l* m$, wobei $l,m<n+1$.
  \end{enumerate}
  Nach Induktionsvoraussetzung sind somit $l$ und $m$ Produkte von Primzahlen, somit auch $n+1$.
\end{description}
